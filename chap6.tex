\chapter{总结与展望}

\section{本文的主要创新点}

空间非合作目标6D姿态估计是在轨服务以及空间中残骸清除中一个关键的技术环节。近年来,有较多的文献开始对相关技术进行预先研究。但这些研究仍然存在一定的问题。模型空间非合作目标6D姿态估计模型为了追求高精度,采用了计算复杂度较高的网络,使得实际运行效率低下,难以实现真正的在轨部署。还有一些研究采用较为简单的端到端的网络实现空间非合作目标的6D姿态估计,然后这样的方法通常精度有限。同时相当多的研究只是局限于静态图像的6D姿态估计,然而实际上空间非合作目标是具有一定的运动状态的,充分利用时序信息可以有效提升6D姿态估计的精度。为了解决上述问题,本文做了在以下三个工作上做出了一定的改进和创新。

	(1)设计了基于注意力机制的空间非合作目标关键点检测网络模型。本文在YOLOv8网络架构的基础上,引入了EfficientViT作为主干网络,结合了级联组注意力机制(CGA),有效提升了模型的全局特征的关注度。同时,在YOLOv8网络的Neck部分集成了TANeck(Triplet Attention Neck),增强了模型对局部特征的捕获能力。实验结果表明,所提出的模型在空间非合作目标关键点检测任务中取得了更高的准确性和稳定性。 针对空间目标的三维特性,本文提出了A3DKS(Approximate 3D Keypoint Similarity)损失函数。该损失函数结合了小孔相机模型,将关键点的深度信息纳入损失计算,更准确地反映了三维空间中的误差。相比传统的OKS损失函数,A3DKS在处理目标具有复杂姿态和深度变化的情况下,能够有效提高关键点检测的精度,增强模型对不同深度关键点的敏感度。实验验证了A3DKS损失函数在提高旋转估计准确性和总体姿态估计性能方面的有效性。
	
	(2)改进了基于重投影误差优化的6D姿态解算方法。在传统PnP问题求解中,旋转和平移参数往往存在耦合,影响了姿态估计的精度。本文提出了RANSAC-TRO SQPnP算法,结合了RANSAC的异常点剔除能力、SQPnP的鲁棒优化能力以及信赖域优化的精细化优化优势。该方法首先利用RANSAC和SQPnP进行初始的姿态估计,然后采用信赖域优化对姿态参数进行精细调整,解决了旋转和平移参数的耦合问题。实验结果显示,RANSAC-TRO SQPnP算法在角度误差、相对位置误差和总评分等指标上均取得了最佳性能,显著提升了6D姿态估计的准确性和鲁棒性。
	
	(3)利用基于关键点观测改进了EKF SE(3) 6D姿态滤波方法。本文进一步将关键点的图像观测直接纳入EKF滤波流程,通过在状态内保留位置、速度、姿态和角速度等完整的刚体运动信息,并利用非线性投影方程对关键点像素坐标进行观测约束,不仅避免了对PnP解算的过度依赖,还可在特征退化或部分遮挡时依然利用冗余信息提高姿态估计的稳健性与精度;同时,通过采用基于李群SE(3)的误差状态表示与指数映射更新,可在旋转部分保持数值稳定并抑制线性化误差的累积,最终实现了更高精度与更高鲁棒性的6D姿态滤波结果。

\section{展望}

尽管本文在空间非合作目标6D姿态估计领域取得了一定的成果,但仍存在一些有待进一步研究和改进的方面。

在数据集方面,本文主要利用仿真渲染的数据集和SPEED+数据集进行实验验证。然而,实际的空间环境复杂多变,光照条件、背景干扰、相机噪声等因素可能对模型的性能产生影响。未来可以考虑采集真实的空间目标图像数据,或者利用更为逼真的模拟环境,进一步提升模型的泛化能力和鲁棒性。

针对关键点检测网络模型,虽然引入了注意力机制和新的损失函数,但在处理更加复杂的目标形状、极端姿态变化和严重遮挡情况下,模型的性能仍有提升空间。未来可以探索更深层次的网络结构,或者引入多任务学习、图神经网络等先进技术,进一步增强模型的特征表示能力。同时,结合数据增强、迁移学习等方法,提升模型在小样本条件下的学习能力。

在6D姿态解算方法方面,虽然提出了 RANSAC-TRO SQPnP 算法,但其精度提升依然有限。对最终结果起决定性作用的,往往还是关键点检测模型本身的精度。

空间非合作目标的6D姿态估计在空间碎片清理、卫星维护、空间探测等领域具有重要的应用价值。未来的研究可以结合具体的应用场景,考虑系统的整体设计,包括硬件平台的选择、算法的优化部署、任务规划与控制等,推动该技术在实际工程中的应用。