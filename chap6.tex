\chapter{总结与展望}

\section{总结}

空间非合作目标 6D 姿态估计是在轨服务以及空间中残骸清除中一个关键的技术环节。近年来,有较多的文献开始对相关技术进行预先研究。但这些研究仍然存在一定的问题。目前较多的空间非合作目标 6D 姿态估计模型为了追求高精度,往往采用了计算复杂度较高的网络,使得实际运行效率低下,难以实现真正的在轨部署。也有一些研究采用较为简单的端到端网络实现空间非合作目标的 6D 姿态估计,但这类方法虽然简单轻量,但是通常精度有限。同时,相当多的研究仅局限于静态图像的 6D 姿态估计,然而实际上空间非合作目标具有一定的运动状态,充分利用时序信息可以有效提升 6D 姿态估计的精度。为了解决上述问题,本文在以下三个方面做出了相关工作并提出了两个创新点:

(1) 设计了基于注意力机制的空间非合作目标关键点检测网络模型,这是本文的第一个创新点。本文在 YOLOv8 网络架构的基础上,引入了EfficientViT作为骨干网络,有效提升了模型对全局特征的关注度。同时,在 YOLOv8 网络的 Neck 部分集成了 TANeck,增强了模型对局部特征的捕获能力。实验结果表明,所提出的模型在空间非合作目标关键点检测任务中取得了更高的准确性和稳定性。针对空间目标的三维特性,本文提出了 A3DKS 损失函数。该损失函数结合了小孔相机模型,将关键点的深度信息纳入损失计算,更准确地反映了三维空间中的误差。相比传统的 OKS 损失函数,A3DKS 提高了关键点 3D 分布的敏感性,有效提高了估计精度。特别是那些目标形状较为细长,容易形成关键点纵深分布的情况下,精度提升更为明显。

(2) 改进了基于重投影误差优化的 6D 姿态解算方法,这是本文的第二个创新点。在传统 PnP 问题求解中,旋转和平移参数往往存在耦合,影响了姿态估计的精度。本文提出了 RANSAC-TRO SQPnP 算法,结合了 RANSAC 的异常点剔除能力、SQPnP 的鲁棒优化能力以及信赖域优化的精细化优化优势。该方法首先利用 RANSAC 和 SQPnP 进行初始的姿态估计,然后采用信赖域优化对姿态参数进行精细调整,解决了旋转和平移参数的耦合问题。实验结果显示,RANSAC-TRO SQPnP 算法在相对位置误差和总评分等指标上取得了最佳性能,显著提升了6D 姿态估计的准确性和鲁棒性。

(3) 利用关键点观测改进了 EKF SE(3) 6D 姿态滤波方法。本文将关键点的图像观测直接纳入EKF滤波流程,通过关键点施加更多的约束,从而减弱了对运动恒速近似所导致的误差。同时,通过采用基于李群 SE(3) 的误差状态表示与指数映射更新,可在旋转部分保持数值稳定并抑制误差的累积,最终实现了更高精度与更高鲁棒性的 6D 姿态滤波结果。

\section{展望}

尽管本文提出了一套有效的空间非合作目标 6D 姿态估计方案,但仍存在一些有待进一步研究和改进的方面:
在数据集方面,本文主要利用自建的仿真渲染数据集和 公开的合成数据集 SPEED+ 进行网络训练和测试,虽然在合成数据集上取得了较好的精度。但实际的空间环境复杂多变,光照条件、背景干扰、相机噪声等因素是合成图像所不能完全反映的,而这些真实的条件也会对模型的性能产生影响。鉴于空间非合作目标真实数据集的采集难度,可考虑采用地面实物模型搭配空间环境光照模拟进行数据集采集。也可以考虑使用真实的样本数据作为补充,鉴于真实的数据集难以通过工具进行精确位姿标注。可以考虑采用无监督学习的方法,在真实数据集上对模型进行训练。或者可以考虑训练从合成数据域到真实数据域的生成式对抗网络。将大量合成数据集映射到真实阈风格,进而提升模型到真实场景的迁移能力。

在 6D 姿态滤波方面,本文设计的基于关键点检测的 EKF SE(3) 滤波器在应对自由翻滚的空间非合作目标运动时,相对于传统EKF体现出了更优的滤波能力。但在角度滤波方面,提升幅度不够明显,这主要受制于运动方程的恒速近似,导致旋转误差改善有限。未来可深入研究欧拉方程,采用更精确的方式近似其运动模型,从而更显著地改善旋转误差。
