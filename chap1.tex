
\chapter{绪论}
\label{chap:intro}

\section{研究背景与意义}
随着人类对太空探索的不断深入,航天器发射活动经历了从无到有、由少变多的历史过程。自1957年苏联成功发射世界上第一颗人造地球卫星斯普特尼克1号以来,人类航天进入快车道。20世纪60年代的美苏太空竞赛时期,大量卫星被送入轨道,用于通信、气象、侦察等用途。此后,随着航天技术进步和发射成本降低,全球每年发射的航天器数量持续攀升。进入21世纪,越来越多国家和商业公司参与航天活动,低成本的小型卫星和“新太空”商业发射兴起,致使在轨航天器的总数屡创新高。据统计,自太空时代开始至今,人类已进行了数千次火箭发射,将上万件航天器送入地球轨道​
sdup.esoc.esa.int
。目前,地球轨道上运行的各类人造卫星数量已达到历史最高水平\cite{prol2022position,zhao2022overview}。这奠定了人类航天利用的基础,但也为轨道环境管理带来了新挑战。

航天器在轨承担的任务高度多样化。最初发射的人造卫星多用于科学试验与技术验证,但很快扩展到通信、导航、遥感和空间科学等诸多领域。例如,通信卫星构建了覆盖全球的卫星通信网络,提供电视广播和互联网连接服务;导航卫星发展出全球卫星导航系统(如美国GPS、中国北斗、欧洲Galileo等),为各国提供高精度的定位、导航和授时支持;遥感卫星用于对地观测,可获取气象预报、环境监测、资源勘测等重要数据;此外,还有空间望远镜和行星探测器用于天文观测与深空探测,载人航天器和空间站用于开展微重力科学实验和在轨技术测试等\cite{zhao2022overview,nozawa2023extent}。航天器功能的不断拓展满足了人类在通信、导航、科研等方面日益增长的需求,同时也使得被送入轨道的航天器数量迅猛增加,轨道资源变得愈发拥挤。

然而,大规模的航天发射活动在惠及人类社会的同时,也带来了严峻的空间碎片问题。正常运行寿命终止后的卫星如果未能及时实施离轨措施,就会变成失去功能的漂浮物长期滞留在轨道上。火箭上面级在完成发射任务后通常也会遗留在轨道环境中,其残余燃料或电池组有可能发生爆炸,产生大量碎片。此外,近年来发生的在轨碰撞和反卫星武器试验事件大幅增加了轨道碎片的数量。例如,2007年某国进行的反卫星导弹试验,以及2009年一颗废弃的俄罗斯卫星与一颗现役商业通信卫星相撞,两次事件就分别产生了数以千计的碎片云。据统计,自20世纪60年代以来,各类航天器爆炸、解体或相撞的事件已超过600起,产生了大量新增碎片\cite{debris1}。这些失控的废弃航天器和碎片随着时间在轨道上越积越多,逐渐演变成对正常航天活动的潜在威胁源。

空间碎片以每秒数公里的高速绕地球运行,即使只有几毫米至几厘米大小,一旦与航天器发生碰撞也会造成严重损坏乃至毁灭性后果。更为严重的是,大碎片相撞会产生更多新的碎片,从而进一步增加未来碰撞的概率,形成恶性循环。这种连锁反应被称为“凯斯勒综合征”,其结果可能使近地轨道出现不可逆的碎片密度增长,最终威胁所有在轨航天器的安全运行\cite{debris2}。目前空间碎片的不断累积已经对航天安全构成了日益严峻的挑战\cite{debris1,debris2}。如果听任碎片数量无节制增长,未来人类对近地轨道的正常利用将面临严重风险。

近几十年的观测数据清晰地表明,空间碎片的数量呈快速增长趋势,轨道拥堵问题日渐凸显。据统计,自1957年进入太空时代以来,人类已进行了约6800次火箭发射,将约2万颗人造卫星送入地球轨道​。截至2023年,仍有大约1.1万颗卫星留存在轨道上,其中约有一半左右处于正常运行状态,其余则已废弃失效\cite{prol2022position,nozawa2023extent}。与此同时,可被各国太空监视网络(如美国空间监视网)跟踪到的在轨物体数量已超过3万个,主要包括废弃卫星、火箭残骸以及较大的碎片​。据模型估计,尺寸大于10厘米的轨道碎片总数约4万个,直径1厘米以上的碎片超过百万个,而更小的毫米级微小碎片数量可能达到上亿级​。这些小碎片目前难以被全面监测,却同样有可能对航天器造成威胁。近年来,随着SpaceX公司“星链”(Starlink)等大型卫星星座计划的实施,单次发射部署数十颗卫星的情况逐渐成为常态,低地轨道的目标物数量陡增,加剧了轨道空间的拥挤程度。基于当前趋势,专家预测如果不采取有效控制措施,未来轨道碰撞事故发生的频率将大幅提高,太空环境的可持续利用前景不容乐观。

针对日益严重的空间碎片威胁,国际社会和各国航天机构已经开展了多方面的应对措施。20世纪末以来,美、俄、欧等主要航天国家相继制定了航天器任务结束后减少残留物的指南和行业规范。例如,国际空间碎片协调委员会(IADC)于2002年提出了著名的“25年规则”,建议近地轨道航天器在退役后25年内脱离轨道,以避免长期滞留太空;联合国和平利用外层空间委员会也于2007年通过了《空间碎片减缓指南》,呼吁各国在航天活动中遵循统一的碎片控制标准。这些措施要求航天器在寿命结束时采取尽快离轨、进入大气层烧毁或移至“墓地轨道”等方式来处置残骸,从源头上减少新的碎片产生。此外,各大航天机构均建立了空间碎片监测与碰撞预警机制,利用地面雷达、光学望远镜和空间监视卫星跟踪在轨物体,并及时发布碰撞风险预警。当预计有碎片可能接近运行航天器时,通常会采取规避机动降低碰撞概率。例如,截至2022年底,国际空间站自1999年以来已累计进行30余次轨道变轨来躲避潜在碎片的威胁,充分说明碎片碰撞风险已成为日常运营中不得不考虑的重要因素。通过上述国际指南约束和在轨碰撞预防措施,新碎片产生的速率在一定程度上得到了控制,但轨道环境并未根本改善,已有碎片的清理和更全面的管理需求愈发迫切。

展望未来,世界各国正积极探索更主动有效的空间碎片治理方案。其中一个重要方向是主动清除(ADR, Active Debris Removal)技术,即通过专门的清除航天器主动捕获并移除在轨碎片,以降低碰撞风险。目前已提出并测试了多种碎片清理技术,包括:在废弃卫星上部署拖曳帆以增加大气阻力、加速其再入大气层烧毁;使用捕获网或鱼叉等装置从近距离捕捉并拖带碎片目标;以及发射携带机械臂的清除卫星,对大型废弃物进行抓取并施加推力使其离轨坠毁等。例如,欧洲航天局计划于2025年执行ClearSpace-1任务,使用一颗配备机械爪的清除卫星捕获并拖离一件废弃火箭上段残骸\cite{zhao2022overview}。又如,日本初创公司Astroscale在2021年实施了ELSA-d在轨演示任务,验证了服务卫星与目标碎片自主会合并利用磁力夹持技术实现抓捕的可行性,为商业清除技术奠定了基础。另一项前沿研究方向是利用定向能技术清理碎片,例如使用地基激光照射小型碎片,使其轨道逐步降低或使碎片汽化烧蚀\cite{nozawa2023extent}。这些主动清理技术有望直接减少在轨碎片数量,被视为遏制碎片增长的有效手段,但在实际应用上仍面临技术复杂、成本高昂和法律责任界定等诸多挑战,需要进一步攻关和国际合作。





随着人类对太空探索的不断深入,越来越多的航天器被送入地球轨道,用于通信、导航、遥感、空间实验等多种任务\cite{prol2022position,zhao2022overview,nozawa2023extent}。与此同时,故障或废弃卫星、火箭上面级以及其他空间碎片在轨数量快速增长,已对正常运行的航天器构成潜在威胁,空间残骸问题日益严峻\cite{debris1,debris2}。为应对这一挑战,在轨服务(On-Orbit Servicing, OOS)\cite{ma2023advances,nwac129,asri2024introductory,wang2023bridging}与在轨残骸清除(Active Debris Removal, ADR)\cite{BARANOV2024100982,CREASER2024481,BAREA20243060,WANG2025}成为研究热点。如图~\ref{fig:OOS_ADR} (a) 所示,该场景为展示了DEOS(German Orbital Servicing Mission,即德国轨道服务任务)的概念示意图,该任务旨在通过一颗装备有七自由度机械臂的服务卫星,展示在轨维护和处置技术,包括对失控卫星的捕获和定点处置,为未来在轨服务及空间碎片处理提供关键技术验证。图~\ref{fig:OOS_ADR} (b)则展示了欧洲航天局与ClearSpace SA合作的ClearSpace-1任务概念图。该任务计划于2025年发射,利用自主导航与机械臂技术捕获并清除轨道上残留的Vespa适配器,


右侧场景则为追踪者航天器捕获火箭引擎残骸进行清理操作。为完成上述两类任务,通常需要首先获取目标航天器或空间物体的6D姿态,进而完成目标捕获过程中的近进和捕获操作。然而,空间环境中的复杂条件\cite{aerospace10120997}、目标在轨的非合作特性\cite{PAULY2023339}以及某些目标特殊的几何外形,使得6D姿态测量面临较大挑战。
\begin{figure}[htbp]
	\centering
	\includegraphics[width=1.0\textwidth]{Img/OOS_ADR.png}
	\caption{OOS与ADR场景的概念图 (a) OOS场景效果图;(b) ADR场景效果图}
	\label{fig:OOS_ADR}
\end{figure}

\section{研究现状}
\subsection{空间非合作目标6D姿态测量设备方案现状}
如图~\ref{fig:6D_equip}所示,从6D姿态测量的设备方案来看,主要有单目相机\cite{PAULY2023339, 9802504,Zhang_2024_CVPR,Liu_2024_CVPR}、双目相机\cite{GXXB202106018, zhang2017optimization,Fan2024}、结构光\cite{laser_stereo,hu2023non,sun2022relative}和激光雷达\cite{10801205,10823741}四类。双目相机、结构光和激光雷达方案能够获取目标的深度信息,理论上更有利于进行目标的6D姿态估计。但从设备本身而言,这些设备的复杂度明显高于单目相机,硬件可靠性相对较差,同时成本也较高。从空间环境干扰角度看,由于极端的光照环境,使得双目相机、结构光和激光雷达等对光照条件要求较为苛刻的测量手段易受干扰\cite{rs15092286,tian2023all}。相比之下,基于单目视觉的方案仅需一台相机,设备简单,可靠性高,成本较低,在航天器上易于部署。同时,针对极端光照条件的单目图像处理算法研究更为丰富。因此,单目视觉6D姿态估计已成为在轨服务领域的重要研究方向。然而,单目相机仅能获取目标的二维图像信息,缺少直接的深度测量。加之空间环境下光照剧变,过明或过暗均会导致目标局部特征弱化,尤其是近处视场无法容纳整个目标导致特征截断损失等复杂因素,进一步降低了特征提取和匹配的稳定性\cite{Hu_2021_CVPR,wang2022revisiting}。如何在保证硬件系统简单的前提下,克服单目视觉对深度信息依赖的先天不足,以及空间环境中光照与纹理不利条件带来的精度衰减,依然是一个亟待解决的难题。
\begin{figure}[htbp]
	\centering
	\includegraphics[width=1.0\textwidth]{Img/6Dpose_measure_equipment.png}
	\caption{6D姿态测量设备方案对比}
	\label{fig:6D_equip}
\end{figure}


\subsection{6D姿态估计算法研究现状}
针对空间非合作目标的6D姿态估计,国内外学者进行了大量研究,提出了从传统方法到基于深度学习的多种方法框架。早期方法多利用基于模型的图像特征提取与配准:从单目图像中提取目标的特征点、边缘或轮廓等,与目标的先验三维模型进行匹配,进而求解姿态。常用的特征点检测算法如SIFT\citep{sift}、SURF\cite{bay2006surf}等在一定程度上具有尺度不变性和光照不变性,可用于卫星外形特征提取,但在极端光照和复杂背景下性能会显著下降。与特征点相比,基于边缘和角点的算法对地球背景干扰表现出更强的鲁棒性;在目标发生部分遮挡时,边缘/角点特征的跟踪稳定性也优于单纯的关键点方法。总体而言,传统基于特征工程的方法实现流程清晰,可解释性强,但面对太空环境的强光照变化和杂乱背景时,稳定性和精度难以满足任务需求。

近年来,借助计算机视觉和深度学习的快速发展,出现了基于深度学习的端到端姿态估计方法。这类方法利用卷积神经网络从图像中自动提取高层特征,学习目标姿态映射关系。一种思路是直接回归目标的姿态参数,但纯粹端到端回归往往难以保证物理合理性,且对训练数据依赖很大。更加普遍的做法是引入关键点检测作为中间步骤:首先通过目标检测网络在图像中定位空间目标的位置,然后采用关键点检测网络提取目标上一组预先定义的特征关键点,最后结合目标3D模型,利用PnP(Perspective-n-Point)算法求解姿态。在欧洲空间局(ESA)等组织举办的卫星姿态估计挑战赛中,多支团队采用了此种两阶段或多阶段方法。例如,UniAdelaide团队利用目标检测框架检测卫星位置,接着使用HRNet网络回归出卫星上的关键点坐标,最后通过PnP计算姿态\cite{chen2019satellite}。该方法充分利用了深度学习提取图像特征的能力,在合成数据上取得了高精度的姿态估计结果。但需要注意的是,多阶段方法往往存在参数量大,计算复杂度高的特点。关键点检测方案还要求目标模型具有足够明显的特征点可供学习和检测;对于表面光滑缺乏特征的航天器,获取稳定的关键点描述变得困难。此外,由于训练数据主要来自仿真,深度模型在真实太空场景下的泛化能力仍有待验证。Pasqualetto等人采用Hourglass卷积网络检测卫星角点特征,虽然在合成图像上成功检测出被光照掩盖的特征,显示出相对于传统算法的潜力,但其在含地球背景的真实图像上鲁棒性尚未充分论证,检测误差对姿态解算的影响也缺乏评估\cite{pclfrg}。

针对上述问题,学者们提出了一些改进方向。例如,将注意力机制引入到姿态估计网络中,以提升对目标显著特征的关注度,抑制背景干扰。注意力机制已被证明能够显著提高6D姿态估计模型的性能和鲁棒性。Rasheed等提出的AttentionPose模型通过像素级特征投票结合注意力模块,在存在遮挡和背景杂乱的情况下仍取得了99\%以上的准确率\cite{RasheedFarhanJasim+2023}。这表明,通过让网络聚焦于关键部位特征,能够增强模型在复杂环境下的抗干扰能力。

在地面场景中,单目6D姿态估计已被广泛研究,并催生了多种成熟算法与应用。近年来,深度学习技术的引入为该领域带来了快速发展:
- 一些工作采用了直接回归或端到端的方式输出目标的6D姿态,如DOPE\cite{Tekin2018DOPE}、DPOD\cite{Li2019DPOD}、GDR-Net\cite{Wang2021GDRNet}与CosyPose\cite{Peng2020CosyPose6D}等。这类方法在一定程度上简化了特征匹配流程,但在面对纹理缺乏或光照变化时,模型往往需要更丰富的训练数据以及复杂的网络结构来保证鲁棒性。
- 另一种流行的思路是关键点检测+PnP方法,通过检测图像关键点并与目标先验模型匹配后,再使用PnP算法(如P3P、EPnP)估计6D姿态。该类方案结合了深度学习对特征的鲁棒提取能力与几何算法的可解释性,可以结合RANSAC等机制有效剔除异常点。在航天应用中,一些研究也在探索如何在弱纹理与强光照干扰下实现准确的关键点检测,如文献\cite{Landrieu2018PointFusion, Simon2018BB8}等在多模态融合与局部特征学习方面做出了探索。
- 融合Transformer和检测网络亦是近来的研究趋势,如端到端的Transformer目标检测\cite{Carion2020EndToEnd}将目标检测过程简化为一个序列预测任务;金字塔视觉Transformer(Pyramid Vision Transformer, PVT)\cite{Wang2021PyramidVisionTransformer}借助金字塔结构缓解了高分辨率特征图带来的大规模计算,并适用于包括目标检测、语义分割等在内的多种视觉任务;对于人脸关键点与空间目标关键点检测,局部块稀疏Transformer\cite{Xia2022SparseLocalPatch}展示了在局部特征聚合和注意力学习方面的潜力。然而,注意力机制的引入也会使模型参数量显著增加,增加计算复杂度,不利于功耗受限设备的部署。

另一方面,在姿态求解阶段融合鲁棒估计算法也是研究的重点。传统PnP问题求解方法众多,包括迭代法(如LHM、PPnP)和非迭代法(如EPnP、DLS等),但迭代法易受初值影响陷入局部极小,非迭代法为提高效率常牺牲部分精度。近年来出现了将PnP转化为凸优化以获得全局最优解的方法(如将PnP构造成二次约束二次规划QCQP问题),提高了解算精度。然而,无论何种PnP算法,在实际应用中都必须考虑野值和异常匹配的干扰。因此,将RANSAC等随机一致性算法与PnP求解相结合已成为常用策略,通过对多假设进行评估来排除错误对应。在空间非合作目标场景中,由于强背景噪声可能导致错误的特征匹配或检测,鲁棒的姿态估计策略尤为重要。

此外,视觉导航滤波技术在空间相对导航中扮演着不可或缺的角色。单帧图像的姿态估计计算复杂,难以支持高频率的姿态连续输出。为了实现对高速旋转或机动目标的实时跟踪,需要将视觉测量结果融入导航滤波器,以预测和更新姿态。常用的滤波方法包括扩展卡尔曼滤波(EKF)、无迹卡尔曼滤波(UKF)以及粒子滤波等。卡尔曼滤波(Kalman Filter, KF)自20世纪60年代由Kalman~\cite{kalman1960new}提出以来,一直是信号处理和状态估计领域的里程碑式算法。最初的线性KF假设系统模型呈线性、高斯噪声分布且统计特性已知,虽然在当时已成功应用于导弹制导、姿态控制等工程项目,但随着日后应用环境的复杂化与非线性化需求迅猛增长,单纯的线性卡尔曼滤波很快在处理强非线性与未知噪声特性等问题时力不从心。为此,研究者开始借助扩展卡尔曼滤波(Extended Kalman Filter, EKF)~\cite{jazwinski2007stochastic}对系统方程进行一阶泰勒展开,从而在一定程度上提升了对非线性系统的适应能力。得益于EKF的出现,卡尔曼滤波在航天器姿态确定方面取得了早期成功,例如NASA在阿波罗登月计划中所使用的导航计算机亦借鉴了KF的思想来处理惯性导航数据。不过,EKF在面对显著非线性或高速翻滚状态时,易受线性化近似误差的影响而出现滤波发散或精度衰减的问题。

为克服EKF在强非线性系统中的局限,Julier与Uhlmann在1990年代提出了无迹变换(Unscented Transform)及无迹卡尔曼滤波(Unscented Kalman Filter, UKF)~\cite{Julier1997}。UKF通过精心选取的Sigma点来捕捉非线性系统中均值与协方差传播的演化过程,相较于EKF无需显式求解雅可比矩阵,且在高翻滚速度或强耦合动力学中能更稳定地保持估计精度。与此同时,粒子滤波(Particle Filter, PF)与容积卡尔曼滤波(Cubature Kalman Filter, CKF)等算法也相继出现,它们或通过基于蒙特卡洛采样的粒子群对任意分布进行近似,或借助容积积分公式提升在多维系统中的精度,进一步丰富了非线性滤波在复杂航天任务场景下的选择~\cite{9272767,AMCCKF_2023}。

近年来,卡尔曼滤波及其改进算法在航天领域(尤其是非合作目标的视觉测量与在轨服务等任务)得到了广泛应用。对于航天器姿态确定这一核心任务,早期方法多采用EKF直接对非线性姿态运动方程进行局部线性化~\cite{Lefferts1982,Shuster1981};然而在强耦合、高翻滚速度以及柔性附件效应下,EKF往往容易在精度或数值稳定性上失效。为提升非线性捕捉能力与鲁棒性,越来越多研究者开始基于UKF或PF等算法进行姿态滤波:相较之下,UKF依托无迹变换能够更准确地捕捉姿态动力学中均值与协方差的传播特征,对高速翻滚或带有柔性部件的航天器具有更高的姿态估计精度和稳定性~\cite{NJHK202201006,1024861534.nh}。另外,诸如自适应策略(噪声协方差在线辨识)、最大相关熵(Maximum Correntropy Criterion)或变分贝叶斯(Variational Bayesian)等思想在卡尔曼滤波基础上的结合,使得算法在存在非高斯噪声、传感器故障以及系统建模误差等恶劣条件下依然能维持较高精度与收敛速度~\cite{GXJM202103017,9272767,qiu2023novel,POURTAKDOUST2022134}。

对于非合作目标的视觉导航与姿态估计,这一问题在当前在轨服务(On-Orbit Servicing, OOS)及主动清除(Active Debris Removal, ADR)任务中尤为突出。由于缺乏先验的外形、惯性参数以及可用的星敏感器或激光角度测距等协作信号,需要完全依赖外部视觉传感器(如激光雷达、双目/多目相机等)进行相对姿态与位置的量测。文献~\cite{BARBIER2023144}在误差状态卡尔曼滤波(ESKF)框架下引入对目标形状和动力学模型的联合估计,并借助高斯过程(Gaussian Processes)来对目标旋转进行建模,成功在合成视觉数据中实现了对未知目标形状与运动的准确跟踪。除此之外,在推力脉冲或陀螺故障的场景下,量测分布往往呈现非高斯性或存在尖峰、拖尾等异常噪声特征,基于高斯混合模型(GMM)、相关熵或实时协方差匹配的自适应滤波方案便可以有效应对短时或突发误差。例如,文献~\cite{2022HO-UKF}提出了运用高阶UKF对多传感器(太阳敏感器、磁强计等)观测融合的策略,降低了航天器姿态确定对单一测量的依赖;而文献~\cite{kim2023gmm,Xiao_2024,AMCCKF_2023}进一步提出针对测量噪声分布随时间变化或存在离群点时的自适应调参与鲁棒优化方法,从而在系统与量测模型不匹配时依然能维持良好性能。与此同时,故障检测与隔离(FDI)机制在滤波流程中的嵌入,也为高可靠度的姿态控制打下基础~\cite{POURTAKDOUST2022134}。值得注意的是,上述部分方法虽非线性拟合能力强,但计算代价相对较高,且参数量大,在实际部署中需权衡计算资源与性能需求。

\section{主要研究内容及其章节安排}
针对当前在轨失效航天器与空间残骸日益增多所带来的潜在威胁以及在轨维修和清理的迫切需求,本文聚焦于获取空间非合作目标的6D姿态信息。本文在单目图像的基础上提出了一种基于关键点检测的完整6D姿态估计算法流程:首先,结合轻量化Transformer(EfficientViT)与Triplet Attention改进YOLOv8-Pose网络,在保证轻量化的同时鲁棒且检测图像关键点;然后,在关键点检测结果的基础上,提出了RANSAC-TRO SQPnP算法,通过排除异常点并采用重投影误差优化与信赖域优化,获得精确的6D姿态解算;最后,针对空间非合作目标6D姿态估计结果的不稳定性,利用基于关键点观测的 SE(3) 扩展卡尔曼滤波(EKF)方法,对多帧数据进行姿态平滑与误差修正。通过上述策略,本文在关键点检测准确性、单帧姿态解算鲁棒性以及多帧姿态估计的稳定性三个层面都进行了有效的探索和改进,可为在轨服务与空间残骸清理等任务提供有效的参考方案。

第一章绪论阐述了空间非合作目标6D姿态估计的背景和意义,介绍了相关研究现状与主要技术路线,并明确提出本文的研究目标与整体思路。

第二章理论基础介绍了6D姿态的概念及其数学表示方法、PnP问题与常见解法,概述了注意力机制与卡尔曼滤波的基础理论,为后续章节提供必要的理论支撑。

第三章到第五章介绍了空间非合作目标6D姿态估计的整体流程,如图\ref{fig:6D_pose_estimation_archieture}所示。对于一张空间非合作目标单目图像,首先输入基于注意力机制的关键点检测网络,得到图像中空间非合作目标的像素坐标;然后结合空间非合作目标标准6D姿态的关键点3D坐标,输入到RANSAC-TRO SQPnP中解算出图像中空间非合作目标的6D姿态。为进一步提升6D姿态估计的精度,将结果输入基于关键点观测的 SE(3) EKF滤波器,得到滤波后的空间非合作目标6D姿态估计结果。

具体而言,第三章依次介绍了空间非合作目标关键点检测数据集的预处理与制作,包括公开数据集SPEED与SPEED+的关键点标注,以及Nauka MLM与Starlink数据集的渲染和关键点标注。随后介绍了在YOLOv8-Pose架构的基础上,采用轻量化的Transformer EfficientViT与Triplet Attention分别改进YOLOv8-Pose的骨干网络和Neck网络,构建出了鲁棒的关键点检测模型并兼顾了模型的轻量化,用以预测单目图像中关键点的像素坐标。其中,EfficientViT与Triplet Attention增强了网络对空间非合作目标全局特征与局部特征的关注度,从而在一定程度上增强了目标截断,光照过暗或者过强,背景干扰等情形的鲁棒性。为增强模型应对3D,在OKS(Object Keypoint Similarity)损失的基础上提出了A3DKS损失,利用该损失有效地训练出更高精度的关键点检测模型。上述改进均通过消融实验验证了其有效性。

第四章介绍了RANSAC-TRO SQPnP算法,利用该算法在已获得的关键点像素坐标基础上进行6D姿态解算,从而得出单目图像中空间非合作目标的6D姿态。同时,将本文提出的方法(第三章到第四章)的估计结果精度与其他单目图像6D姿态估计方法进行了对比,验证了所提方法的有效性。

第五章考虑到空间非合作目标的运动特性,引入6D姿态滤波方法来改善基于单帧估计的精度。分析了EKF滤波器的局限性,引入了基于关键点观测的方法,增加矫正信息的丰富性,同时引入SE(3)来降低四元数更新过程中的误差。

第六章总结与展望总结了全文的研究贡献,指出论文的不足之处,并展望未来在轨目标视觉测量与服务技术的发展方向。

\begin{figure}[htbp]
	\centering
	\includegraphics[width=1.0\textwidth]{Img/chapt2_overview.png}
	\caption{本文的方法架构图}
	\label{fig:6D_pose_estimation_archieture}
\end{figure}
