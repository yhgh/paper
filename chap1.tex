\chapter{绪论}
\label{chap:intro}

\section{研究背景与意义}
随着人类对太空探索的不断深入,航天器发射活动经历了从无到有、由少变多的发展过程。自 1957 年苏联成功发射世界上第一颗人造地球卫星斯普特尼克1号以来,人类航天事业进入快速发展阶段\cite{Alpert2015}。
在 20 世纪 60 年代的美苏太空竞赛时期,大量卫星被送入轨道,用于通信、气象、导航、遥感和空间实验等多种领域。随着航天技术的进步和发射成本降低,全球每年发射的航天器数量持续增加。进入 21 世纪,越来越多的国家和商业公司参与航天活动,低成本的小型卫星以及“新太空”商业发射兴起,导致在轨航天器的总数屡创新高\cite{Kopacz2020}。据统计,自太空时代开始至今,人类已进行了数千次火箭发射,将上万颗航天器送入地球轨道。目前,地球轨道上运行的各类人造卫星数量已达到历史最高水平\cite{Pardini2021}。航天器的功能也从最初的科学实验扩展到通信、导航、遥感和空间科学等领域,满足了人类社会日益增长的需求,同时也使轨道资源变得愈发拥挤。这种大规模的航天活动带来了严重的空间残骸问题\cite{Liou2006}。卫星寿命终止后若未能及时离轨,将长期滞留在轨道上,火箭分离的过程也会产生大量的残骸。如图\ref{fig:spacedebris_distribute}所示,轨道空间中的残骸分布日益密集,空间碎片数量呈快速增长趋势\cite{Giudici2024}。
\begin{figure}[htbp]
	\centering
	\includegraphics[width=0.5\textwidth]{Img/debris.png}
	\caption{空间残骸分布图\cite{yalccin2022class}}
	\label{fig:spacedebris_distribute}
\end{figure}

美国国家航空航天局轨道碎片计划办公室在其轨道碎片季报卫星中指出,截至 2024 年 6 月 9 日,空间监视网已为 12 745 颗航天器(含在役与失效卫星)建立目录\cite{NASAODPO2024};而欧洲空间局 2024 年《年度空间环境报告》给出的在役卫星数量约为 9 100 颗\cite{ESA2024}。由此推算,约 3 600 颗卫星已处于失效状态,除此之外还有更多的空间残骸。

为解决此问题,主动式空间碎片清除(Active Debris Removal,ADR)技术\cite{BARANOV2024100982,CREASER2024481,BAREA20243060,WANG2025}已成为研究热点。如图~\ref{fig:OOS_ADR} (a) 与 (b) 所示,分别展示了对轨道上的火箭箭体残骸与推进引擎残骸进行清除。除了这两种残骸之外,还有其他各种航天任务遗留下的残骸。

\begin{figure}[htbp]
	\centering
	\includegraphics[width=0.7\textwidth]{Img/OOS_ADR.png}
	\caption{ADR 场景的概念图 (a) 火箭箭体残骸清除;(b) 推进引擎残骸清除}
	\label{fig:OOS_ADR}
\end{figure}

同时,随着高价值航天器数量的增加,在轨维修的需求也日益增长。如图~\ref{fig:space_repaire}(a)所示,礼炮七号空间站曾在电子系统失灵后进行非合作状态下的联盟号飞船对接维修。图~\ref{fig:space_repaire}(b)展示了哈勃太空望远镜的多次在轨维修。未来太阳能电站、太空加油站等空间设施也存在类似需求,因此在轨服务(On-Orbit Servicing, OOS)\cite{alizadeh2024comprehensive, wang2025review}成为研究热点。
\begin{figure}[htbp]
	\centering
	\includegraphics[width=0.5\textwidth]{Img/onorbitservicing.png}
	\caption{在轨维修案例 (a) 礼炮七号空间站; (b) 哈勃太空望远镜}
	\label{fig:space_repaire}
\end{figure}

为了完成上述两类任务,通常需要一个追踪者航天器,通过其对这两类任务中的目标的位置和姿态的测量,进而能够完成对目标的进近,以及以一个相对合适的角度抓取目标。其中测量出目标的位置和姿态的参数分别有3个自由度,所以统称为 6D 姿态,对于ADR中的残骸以及 OOS 中的某些目标,由于已经失效,无法正常工作,对自身的姿态不能正常控制,且不能通过主动与追踪者航天器进行交互,传递自身的姿态信息。因此这些目标被称之为空间非合作目标。然而,空间环境中的复杂条件\cite{aerospace10120997}、目标在轨的非合作特性\cite{PAULY2023339}以及某些目标特殊的几何外形,使得6D姿态测量面临较大挑战。


\section{国内外研究现状}
\subsection{空间非合作目标 6D 姿态测量设备方案现状}
如图~\ref{fig:6D_equip}所示,从 6D 姿态测量的设备方案来看,主要有单目相机、双目相机、结构光和激光雷达四类。

单目相机方案只需要一台相机,通过拍摄目标的二维图像并结合几何先验或深度学习方法来估计 6D 姿态,硬件简单、体积小、成本低,易于在航天器上部署。然而,缺少深度测量会使其对光照剧变和纹理弱化更加敏感,尤其当目标较近且视场无法完整容纳目标时,特征截断造成的深度推断不准确会显著降低测量精度。针对这一类别,Long 等人提出了利用多圆特征的单目相对位姿估计方法\cite{9802504},Zhang 等人又提出了结合自迭代优化与运动一致性的单目 6-DoF 姿态估计框架\cite{Zhang_2024_CVPR},而 Liu 与 Yu 借助 RANSAC-EPnP 设计了自监督的域差异消减策略\cite{Liu_2024_CVPR},这些方法均属于单目视觉范畴。

双目相机方案利用两台相机同时获取左右视图,通过立体匹配与三角测量直接恢复深度信息,从而在理论上更稳定地完成 6D 姿态估计,但系统复杂度相应增加,且在极端光照环境下立体匹配精度会受影响。如Zhang 等人基于太阳翼角点提出了优化式双目姿态估计算法\cite{zhang2017optimization},Fan 等人则面向大型航天柱体零件给出了融合先验数据的智能双目测姿方法\cite{Fan2024},二者皆属于双目视觉类别。

结构光方案通过在目标表面投射已知光栅或利用 ToF 主动测距来获取稠密或半稠密深度,再结合几何先验估计姿态,可获得较高精度,但对投射与光照条件要求严格。Peng 等人融合激光雷达与立体视觉实现了适应恶劣光照的姿态与运动估计\cite{laser_stereo},Hu 等人提出了双目-ToF 协同测姿框架\cite{hu2023non},Sun 等人进一步利用 ToF 相机结合注册-映射算法在非合作目标测姿中展现了高精度与实时性\cite{sun2022relative};上述方法均归属主动结构光/ToF 类别。

激光雷达方案基于激光扫描或飞行时间原理直接输出三维点云,再通过点云配准(如 ICP)或深度网络完成 6D 姿态估计,精度高但体积、功耗与成本较大,并且对强反射环境敏感。Renaut 等人提出了针对点云对称性的 CNN-based 姿态估计方法\cite{10801205},Wang 等人则面向小天体交会场景设计了基于激光雷达的旋转状态估计算法\cite{10823741},两者均属于激光雷达类别。


双目相机、结构光和激光雷达方案能够获取目标的深度信息,可以通过点云配准、深度图融合等方式在理论上更有利于 6D 姿态估计,但设备的复杂度、硬件可靠性以及高成本等问题在太空应用中更为突出。而且空间光照条件常呈现过强或过弱等极端状态;在此情况下,结构光等依赖投射光条纹的测量方法对光照变化尤为敏感,稳定性容易受到显著影响 \cite{tian2023all}。相比之下,基于单目视觉的方案仅需一台相机,设备简单、可靠性高、成本较低,更适合在轨服务场景部署。加之针对极端光照条件的单目图像处理算法已有较为深入的研究,因此单目视觉 6D 姿态估计已成为在轨服务领域的重要研究方向。

然而,单目相机只能获取目标的二维图像,缺乏深度测量。且在空间环境中,光照剧变往往导致目标局部特征的弱化或丢失,视场限制带来的特征截断也会进一步降低特征提取和匹配的稳定性\cite{Hu_2021_CVPR,wang2022revisiting,peng2019pvnet}。在保证硬件系统简单可靠的前提下,如何克服单目视觉对深度信息依赖的先天不足,并缓解光照与纹理不利因素带来的精度衰减是一个难题。

\begin{figure}[htbp]
	\centering
	\includegraphics[width=1.0\textwidth]{Img/6Dpose_measure_equipment.png}
	\caption{6D姿态测量设备方案}
	\label{fig:6D_equip}
\end{figure}


\subsection{6D 姿态估计算法研究现状}
如图~\ref{fig:6D_pose_method}所示,6D 姿态估计方法主要可以分为三类:端到端直接回归方法、基于关键点检测与 PnP 求解的方法以及融合深度信息的方法。

\begin{figure}[htbp]
	\centering
	\includegraphics[width=0.8\textwidth]{Img/6Dpose_method_classfication.png}
	\caption{6D 姿态估计方法分类}
	\label{fig:6D_pose_method}
\end{figure}

2017 年 Kehl 等人提出了 SSD-6D 方法\cite{kehl2017ssd},通过扩展 SSD 检测框架直接回归物体的旋转四元数与尺度化平移,实现了端到端的单张 RGB 6D 姿态估计。2018 年 Xiang 等人在提出了PoseCNN\cite{xiang2018posecnnconvolutionalneuralnetwork},将中心投票与四元数回归结合,在复杂遮挡场景下提升了稳健性,同时引入对称物体损失函数,不过依赖后续 ICP 才能获得最高精度。2020 年 Bukschat 与 Vetter 提出 EfficientPose\cite{bukschat2020efficientpose},将 EfficientDet 骨干与单发多实例 6D 回归结结合。由于从单张图像映射到 6D 姿态是一项复杂的任务。导致此类方法精度通常不是很理想。



为了降低直接回归 6D 姿态的难度,不少工作将问题拆解为两步:首先利用神经网络在图像中检测物体的一组特征关键点(通常对应物体 3D 模型上的特定点的 2D 投影),然后通过 PnP 算法求解相机与物体间的 6D 位姿。这种思路通过将学习任务转化为二维定位,并利用PnP确保几何一致性,因而通常比端到端方法达到更高的准确度。如 Tekin 等人提出的单阶段检测器以及 YOLO6D\cite{tekin2018real}等,通过网络输出物体 3D 包围盒顶点和质心等预定义关键点在图像中的坐标,再利用PnP恢复姿态。然而,某些关键点的错误估计也会对最终6D姿态的精度造成较为明显的影响。为了进一步提升关键点的检测进度,Peng 等人提出的 PVNet\cite{peng2019pvnet} 模型并未直接回归关键点坐标,而是预测每个像素指向关键点的单位向量场,再通过 RANSAC 投票得到关键点位置。这种像素级投票机制使 PVNet 能够准确定位被遮挡的关键点,并显著提升了在LINEMOD、Occluded-LINEMOD 和 YCB-Video 等数据集上的姿态估计精度。随后的一些工作进一步利用额外的几何约束提高精度,例如 Song 等人提出的 HybridPose\cite{Song_2020_CVPR},将关键点、边缘向量和对称性约束等多种中间几何表示融合,通过鲁棒回归筛除异常点,增强了对物体遮挡和极端姿态的适应性。总的来说,基于关键点与 PnP 的方法比端到端方法流程稍复杂,但由于将学习任务分解并利用了 PnP 求解,其精度普遍更高,在多个基准数据集上取得了更优异地表现。


基于三维数据的 6D 姿态估计方法则通过融合深度图或点云等 3D 信息,以获得更丰富的几何线索来估计 6D 姿态。只依靠 RGB 相机由于透视投影会丢失物体的深度尺度等空间信息,仅依赖 RGB 往往在光照变化、纹理缺失或遮挡情况下性能受限。而 RGB-D 传感器的出现为 6D 姿态估计提供了额外的深度维度,有效缓解了纯 RGB 方法的劣势。典型方法之一是 Wang 等人提出的 DenseFusion\cite{wang2019densefusion},它采用一个异构网络分别提取 RGB 图像和深度图(转换为点云)特征,再通过像素级的稠密融合模块将颜色和几何信息结合来直接回归物体姿态,并配合端到端的迭代优化进一步提升精度。随后,He 等人提出的 PVN3D\cite{he2020pvn3d}则将第二类方法拓展到 RGB-D 输入:他们设计了深度 Hough 投票网络在点云中检测物体的 3D 关键点,并通过最小二乘解算姿态,从而充分利用深度提供的刚体几何约束,大幅提高了姿态估计的准确性。实验结果表明,相较仅用 RGB 的 2D 关键点方法,融合点云的 3D 关键点方法在多个基准上都有显著优势。需要指出的是,此类融合 3D 信息的方法由于处理额外的 3D 数据,且依赖于更复杂的 3D 视觉测量设备,计算开销相对更高,通常推理速度低于纯 RGB 方法,不利于轻量化的部署。



\subsection{6D 姿态估计算法在空间非合作目标领域的研究现状}
针对空间非合作目标的6D 姿态估计,国内外学者进行了大量研究,提出了从传统方法到基于深度学习的多种方法框架。早期方法多利用基于模型的图像特征提取与配准:从单目图像中提取目标的特征点、边缘或轮廓等,与目标的先验三维模型进行匹配,进而求解姿态。常用的特征点检测算法如 SIFT\cite{sift}、SURF\cite{bay2006surf}等在一定程度上具有尺度不变性和光照不变性,可用于卫星外形特征提取,但在极端光照和复杂背景下性能会显著下降。与特征点相比,基于边缘和角点的算法对地球背景干扰表现出更强的鲁棒性;在目标发生部分遮挡时,边缘/角点特征的跟踪稳定性也优于单纯的关键点方法。总体而言,传统基于特征工程的方法实现流程清晰,可解释性强,但面对太空环境的强光照变化和杂乱背景时,稳定性和精度难以满足任务需求。

因此,近年来出现了更多以深度学习关键点检测为主的相关研究。Li 等人提出了一种结合关键点不确定度预测的 6D 姿态估计方法\cite{aerospace9100592}。该方法首先通过目标检测网络截取卫星区域,然后利用关键点检测网络提取图像中具有明显特征的11个预定义关键点,并预测每个关键点检测的不确定度。接着,算法根据不确定度自动筛选出高精度的关键点,对这些筛选后的关键点采用改进的 EPnP(Efficient Perspective-n-Point)算法计算姿态。

Wang 等人重新审视了单目卫星姿态估计问题,引入 Transformer 架构提升关键点检测和姿态求解性能\cite{s23208633}。相较传统卷积网络,Transformer 能够捕获图像中长距离依赖关系,从而增强对新数据集的泛化能力。实验证明,该 Transformer 方案在姿态估计精度和跨数据集泛化性能上均优于同期方法。Chen 等人提出了一个单阶段的同形矩阵约束方法,将卫星刚体的平面特征转化为单应矩阵求解姿态\cite{chen2024spacecraft}。他们设计了一个 2D 关键点回归网络来预测目标在图像上的四角点等平面特征点,利用这些点构建图像与卫星平面的单应矩阵,并通过解析分解得到初始姿态,再以像素误差为损失函数细化姿态估计。

Zhang 等人提出了一种无模型依赖的通用空间目标姿态估计框架\cite{zhang2025robust}。他们不再假设已知的目标 CAD 模型,而是通过三阶段深度学习模块实现姿态跟踪:首先利用卷积网络提取图像亚像素级关键点;其次设计了多通道关键点匹配网络,通过三重损失函数学习在目标物体坐标系下的 2D-3D 关键点对应关系;最后采用动态关键帧池的位姿图优化算法求解并平滑相对姿态。尽管这些网络在精度上取得了不错的成果,但相对复杂的网络结构(如Transformer)以及热图的密集预测方式使得其计算代价相对较高。这使得这些模型在实际部署到功耗受限平台上时存在一定局限性。

\subsection{空间非合作目标6D姿态滤波的研究现状}
视觉导航滤波技术在空间相对导航的 6D 姿态估计中扮演着不可或缺的角色。单帧图像的姿态估计往往容易出现波动的情况,造成一定的不稳定。但是却可以结合其运动信息联合多帧进行滤波,需要将视觉测量结果融入导航滤波器,以预测和更新姿态。常用的滤波方法包括扩展卡尔曼滤波、无迹卡尔曼滤波(Unscented Kalman Filter, UKF)以及粒子滤波(Particle Filter, PF)等。

卡尔曼滤波(Kalman Filter, KF)自20世纪60年代由 Kalman~\cite{kalman1960new}提出以来,一直是信号处理和状态估计领域的里程碑式算法。最初的线性 KF 假设系统模型呈线性、高斯噪声分布且统计特性已知,虽然在当时已成功应用于导弹制导、姿态控制等工程项目,但随着日后应用环境的复杂化与非线性化需求迅猛增长,单纯的线性卡尔曼滤波很快在处理强非线性与未知噪声特性等问题时力不从心。为此,借助扩展卡尔曼滤波对系统方程进行一阶泰勒展开,从而在一定程度上提升了对非线性系统的适应能力。得益于 EKF 的出现,卡尔曼滤波在航天器姿态确定方面取得了早期成功,例如 NASA 在阿波罗登月计划中所使用的导航计算机亦借鉴了 KF 的思想来处理惯性导航数据。不过,EKF 在面对显著非线性或高速翻滚状态时,易受线性化近似误差的影响而出现滤波发散或精度衰减的问题。

为克服 EKF 在强非线性系统中的局限,Julier 与 Uhlmann 在1990年代提出了无迹变换(Unscented Transform,UT)及无迹卡尔曼滤波~\cite{Julier1997}。UKF 通过精心选取的 Sigma 点来捕捉非线性系统中均值与协方差传播的演化过程,相较于 EKF 无需显式求解雅可比矩阵,且在高翻滚速度或强耦合动力学中能更稳定地保持估计精度。与此同时,粒子滤波与容积卡尔曼滤波(Cubature Kalman Filter, CKF)等算法也相继出现,它们或通过基于蒙特卡洛采样的粒子群对任意分布进行近似,或借助容积积分公式提升在多维系统中的精度,进一步丰富了非线性滤波在复杂航天任务场景下的选择~\cite{9272767,AMCCKF_2023}。但是这些算法在空间非合作目标场景中,由于只有单纯的视觉信息,且这些滤波器往往只有 6D 姿态参数的引入,约束性不够强,使得在空间非合作目标复杂翻滚运动中的滤波效果存在局限。

\section{空间非合作目标 6D 姿态估计面临的问题与挑战}
\label{sec:problems_challenges}

在空间在轨服务(On-Orbit Serving, OOS)与主动碎片移除(Active Deribs Removal, ADR)的应用场景中,精确获取空间非合作目标的 6D 姿态信息具有极其重要的意义。然而,这项任务在实际应用中面临诸多挑战,尤其是在兼顾效率、精度与成本方面难度突出。传统的深度测量手段(如双目相机、结构光和激光雷达)虽然在深度获取与空间分辨率方面具有优势,但其系统复杂度高、功耗大且维护成本高,难以在大规模航天任务中广泛应用。相比之下,单目相机因结构简单、重量轻、功耗低而更具潜力。然而,单目相机只能获取目标的二维图像投影,缺乏直接的深度信息,这对算法端的深度推断和姿态解算提出了更高要求。

在网络模型选择方面,速度与精度之间的权衡同样关键。基于密集预测(热图)的深度学习网络通常能在关键点定位上取得较高精度,但其计算量与推理成本相对较高,不利于在资源受限的空间平台上进行实时部署。与之相比,端到端的直接回归模型虽然推理速度更快、计算资源需求更低,但在关键点定位精度上往往不及密集预测方法。如何在效率与精度之间寻求最佳平衡,并选择合适的网络架构,成为空间非合作目标 6D 姿态估计中需要解决的问题之一。

此外,空间环境和空间非合作目标本身的特性也会导致较高比例的噪声干扰,对关键点检测网络的预测结果及最终的 6D 姿态解算产生显著影响。传统几何求解思路或线性方程组求解方法在面对大量噪声时往往出现较大误差,
如何设计一个对噪声更为鲁棒的 PnP 求解器也是一个需要解决的问题。

同时,空间非合作目标的自由翻滚运动通常具有一定的复杂性。由于目标通常处于失控状态,且与追踪者航天器缺乏直接交互,只能依赖视觉获取的图像序列进行位姿估计。目标可能在三维空间中以复杂且未知的方式翻滚。这些因素显著增加了视觉检测难度,对算法的鲁棒性和跟踪能力提出了更高要求。如何在此类复杂环境下,通过单目图像序列有效融合目标的运动信息,实现深度与姿态的精准重建,是当前研究的一个问题。

综上所述,空间非合作目标 6D 姿态估计在效率、精度、成本与可靠性等多个方面面临挑战。为此,需要从深度学习模型设计、PnP 问题解算方法、以及面向空间自由翻滚复杂运动的时序滤波等多个维度进行深入探索,力求在复杂空间环境中实现高效、精准且可靠的姿态估计。

\section{主要研究内容及其章节安排}
\subsection{研究内容}
本文提出了一种基于单目图像、基于关键点检测的空间非合作目标 6D 姿态估计算法流程。首先,在关键点检测阶段,以 YOLOv8 为基本框架,考虑到空间非合作目标场景的复杂性,传统的 YOLOv8 关键点检测版本 YOLOv8-pose 表现不佳。为此,本文引入轻量化的 Transformer 骨干网络 EfficientViT,并在 Neck 中融入 Triplet Attention,通过全局—局部注意力协同整体增强网络的特征提取能力以及对关键特征的关注,从而进一步提升了网络的泛化性和鲁棒性。同时,构建了 A3DKS 损失函数,利用小孔成像原理将关键点损失转换至近似 3D 尺度,使深度分布信息直接参与网络训练,整体提升了 6D 姿态预测的精度。接着,针对检测得到的关键点,设计了 RANSAC-TRO SQPnP 姿态解算方法:该方法采用基于重投影误差的优化策略。首先使用 RANSAC 剔除异常点,消除异常点的大误差对估计精度的影响;随后通过 SQPnP 将问题建模为重投影误差最小化,并通过构造二次规划进行求解。在 SQPnP 解算得到的 6D 姿态基础上,利用信赖域进行进一步优化,获得了较常用 PnP 解算方法更优的整体精度。最后,为抑制孤立帧预测可能存在的较大误差,本文构建了以关键点观测作为残差的 SE(3) 扩展卡尔曼滤波器,采用李群误差状态的更新策略,充分利用了时序冗余信息,进一步提升了 6D 姿态估计的精度。

本研究的创新体现在以下几个方面:在 YOLOv8 架构上构建了融合全局与局部注意力的轻量化关键点检测网络,在保证网络轻量化的同时兼顾了检测精度;通过引入 TRO 优化,改善了 SQPnP 算法在平移参数求解方面的局限性; RANSAC-TRO 方案不仅在宏观上剔除了误差较大的关键点,而且在姿态精细优化阶段利用 TRO 算法使精度收敛至更高水平。
\subsection{章节安排}
第一章绪论阐述了空间非合作目标 6D 姿态估计的背景和意义,介绍了相关研究现状与主要技术路线,并明确提出本文的研究目标与整体思路。

第二章理论基础介绍了 6D 姿态的基本概念、PnP 问题及其常见解法,概述了注意力机制与卡尔曼滤波的基础理论等,为后续章节提供必要的理论支撑。

第三章到第五章介绍了空间非合作目标 6D 姿态估计的整体流程,如图\ref{fig:6D_pose_estimation_archieture}所示。对于一张空间非合作目标单目图像,首先输入基于注意力机制的关键点检测网络,得到图像中空间非合作目标的像素坐标;然后结合空间非合作目标标准 6D 姿态关键点的 3D 坐标,输入到 RANSAC-TRO SQPnP 中解算出图像中空间非合作目标的 6D 姿态。为进一步提升 6D 姿态估计的精度,将结果输入基于关键点观测的 SE(3) EKF 滤波器,得到滤波后的空间非合作目标 6D 姿态估计结果。

具体而言,第三章依次介绍了在 YOLOv8-Pose 架构的基础上,采用轻量化的 Transformer EfficientViT 与 Triplet Attention 分别改进 YOLOv8-Pose 的骨干网络 和 Neck网络,构建出了鲁棒的关键点检测模型并兼顾了模型的轻量化,用以预测单目图像中关键点的像素坐标。其中,EfficientViT 与 Triplet Attention 增强了网络对空间非合作目标某些关键的全局特征与某些关键的局部特征的关注度,从而在一定程度上增强了目标截断,光照过暗或者过强,背景干扰等情形的鲁棒性。为增强模型应对关键点3D变化的鲁棒性,在 Object Keypoint Similarity (OKS) 损失的基础上提出了 A3DKS 损失,利用该损失有效地训练出更高精度的关键点检测模型。随后介绍了空间非合作目标关键点检测数据集的预处理与制作,包括公开数据集SPEED与SPEED+的关键点标注,以及 Nauka MLM 与 Starlink 数据集的渲染和关键点标注。上述改进均通过消融实验验证了其有效性。

第四章介绍了 RANSAC-TRO SQPnP 算法,利用该算法在已获得的关键点像素坐标基础上进行 6D 姿态解算,从而得出单目图像中空间非合作目标的 6D 姿态。同时,将本文提出的方法(第三章到第四章)的估计结果精度与其他单目图像 6D 姿态估计方法进行了对比,验证了所提方法的有效性。

第五章考虑到空间非合作目标的运动特性,引入 6D 姿态滤波方法来改善基于单帧估计的精度。考虑 EKF 滤波器的局限性,引入了基于关键点观测的方法,增加矫正信息的丰富性,同时引入 SE(3) 来降低四元数更新过程中的误差。

第六章总结与展望总结了全文的研究贡献,指出论文的不足之处,并展望未来在可以做出的相关改进。

\begin{figure}[htbp]
	\centering
	\includegraphics[width=0.90\textwidth]{Img/chapt2_overview.png}
	\caption{本文方法的总体架构图}
	\label{fig:6D_pose_estimation_archieture}
\end{figure}
