\chapter{绪论}

\section{研究背景与意义}
随着人类对太空探索的不断深入,越来越多的航天器被送入地球轨道,用于通信、导航、遥感、空间实验等多种任务\cite{prol2022position,zhao2022overview,nozawa2023extent}。与此同时,故障或废弃卫星、火箭上面级以及其他空间碎片在轨数量快速增长,已对正常运行的航天器构成潜在威胁,空间残骸问题日益严峻\cite{debris1,debris2}。为应对这一问题,在轨服务(On-Orbit Servicing, OOS)\cite{ma2023advances,nwac129,asri2024introductory,wang2023bridging}与在轨残骸清除(Active Debris Removal,ADR)成为研究热点\cite{BARANOV2024100982,CREASER2024481,BAREA20243060,WANG2025}。如图~\ref{fig:OOS_ADR}所示,左边的场景为一个追踪者航天器捕获一个卫星从而进行维修操作,比如在轨零件更换,在轨燃料加注等。而右边的场景为一个追踪者航天器捕获一个火箭引擎残骸进而进行清理操作。为了完成上述两类任务,通常需要首先获取目标航天器或空间物体的6D姿态,进而完成目标捕获过程中的近进和捕获操作。然而,空间环境中的一些复杂条件\cite{aerospace10120997},以及目标在轨的非合作的特性\cite{PAULY2023339},以及某些目标存在较为特殊的几何外形,这些使得6D姿态测量手段面临较大挑战。
\begin{figure}[htbp]
	\centering
	\includegraphics[width=1.0\textwidth]{Img/OOS_ADR.png}
	\caption{OOS与ADR的场景}
	\label{fig:OOS_ADR}
\end{figure}

\section{研究现状}
\subsection{空间非合作目标6D姿态测量方案现状}
如图~\ref{fig:6D_equip}所示,从6D姿态测量的各种设备方案来看主要有单目相机\cite{PAULY2023339, 9802504,Zhang_2024_CVPR,Liu_2024_CVPR}、双目相机\cite{GXXB202106018, zhang2017optimization,Fan2024}、结构光\cite{laser_stereo,hu2023non,sun2022relative}和激光雷达\cite{10801205,10823741}。双目相机,结构光,激光雷达这些方案能够获取目标额外的深度信息,理论上更有利于进行目标的6D姿态估计。但是从设备本身来说,这些设备的复杂度要明显高于单目相机,硬件的可靠性相对较差,同时成本也高于单目相机。从空间的环境干扰来说,由于空间中存在极端的光照环境,使得双目相机,结构光,激光雷达这些对光照条件要求较为苛刻的测量手段会受到干扰\cite{rs15092286,tian2023all}。相比之下,基于单目视觉的方案仅需一台相机,设备简单,可靠性高,成本较低,在航天器上易于部署。同时对于极端光照条件较为鲁棒的单目图像处理算法的研究更为丰富。因此,单目视觉6D姿态估计正成为在轨服务领域的重要研究方向。然而,单目相机仅能获取目标的二维图像信息,缺少直接的深度测量。加之空间环境下光照剧变、弱纹理和高亮背景等复杂因素,往往进一步降低了特征提取和匹配的稳定性\cite{Hu_2021_CVPR,wang2022revisiting}。如何在保证硬件系统简单的前提下,克服单目视觉对深度信息依赖的先天不足,以及空间环境中光照与纹理不利条件带来的精度衰减,依旧是一个亟待解决的难题。
\begin{figure}[htbp]
	\centering
	\includegraphics[width=1.0\textwidth]{Img/6Dpose_measure_equipment.png}
	\caption{6D姿态测量设备方案}
	\label{fig:6D_equip}
\end{figure}

\subsection{空间非合作目标6D姿态估计算法研究现状}





1.基于关键点检测+PnP的(说明把复杂图像映射到6D姿态的过程,分解为关键点检测与PnP估计)
在关键点检测阶段,有传统的特征点检测算法。但是通常是对于那些特征比较明显,甚至是认为设计的易于识别的靶标。若不存在易于检测的特征,那么就需要深度学习网络,这里的深度学习网络分为,神经网络直接回归关键点像素坐标的,还有的是通过热图预测,从热图的概率值来定位关键点像素坐标。神经网络预测关键点由于是从图像到关键点的映射,维度从高纬降到了低维度,网络更轻量化,而基于热图的预测是从图像到热图,是高维到高维的映射,复杂度更高,但是精度更高。但是考虑到空间任务的精度需求,以及空间任务内置计算平台的功耗限制。难以找到既轻量化的且精度高的,适合部署到空间平台上的算法。 哦别忘了综述下PnP



2.端到端回归的(说明其从图像到6D姿态的映射是复杂的,拟合存在困难,导致精度不是那么高,有没有模板匹配你查查文献提一嘴, 一般都是用神经网络来完成)

3.结合三维数据的(我理解的可能有点云,深度图等),但是这些三维数据的产生依赖于更复杂先进的测量设备比如前面提到的xxx。



\subsection{空间非合作6D姿态的滤波算法研究现状}

考虑到空间非合作目标的运动咋咋咋
xxxx
简述下KF,EKF,UKF等滤波器。反正要讲下它们的优缺点,EKF稍微多讲讲吧,倾向于赞赏它吧
