\chapter{理论基础}
\label{chap:theory}

在本章中,将详细介绍与本文研究密切相关的理论基础。首先阐述6D姿态的概念及常见表示方法,重点讨论四元数在姿态描述中的优势与特性。随后,将重点介绍注意力机制在计算机视觉中的应用,包括通道注意力与空间注意力这两大类基于CNN特征图再加权的注意力机制,以及基于Transformer的全局自注意力方法。在此基础上,进一步介绍YOLO系列算法及其在关键点检测中的衍生应用、PnP问题与RANSAC等鲁棒估计方法,最后阐述空间目标自由运动模型与卡尔曼滤波(扩展卡尔曼滤波,EKF)的基本原理。通过对这些理论的阐述,为后续章节中关键点检测网络的设计、6D姿态解算方法的实现以及时序滤波方案的应用奠定坚实的理论基础。

\section{6D姿态的表示}

6D姿态全称6自由度(6 Degree of Freedom)姿态,在许多实际应用中(如航天、机器人、增强现实等领域)\cite{Zuo_2024_CVPR,aerospace11070526,NGUYEN2024103459,choi2025robust},精确描述目标在三维空间中的位置与方向至关重要。6D姿态由三维空间中的位置(3个自由度)以及三维空间中的方向(三个自由度)两个方面组成,共计6个自由度。6D姿态有着不同的表示方式,特别是对于方向的描述上。

\subsection{欧拉角表示}
欧拉角通过将总体旋转分解为绕三个固定轴(或物体自身轴)的连续旋转来描述刚体方向。假设采用常用的 ZYX顺序,即先绕 $z$ 轴旋转 $\psi$(偏航角),再绕 $y$ 轴旋转 $\theta$(俯仰角),最后绕 $x$ 轴旋转 $\phi$(滚转角)。对应的旋转矩阵 $R$ 可写为
\begin{equation}
	R = R_x(\phi) \, R_y(\theta) \, R_z(\psi),
\end{equation}
其中
\begin{equation}
	R_x(\phi) = \begin{pmatrix}
		1 & 0 & 0 \\
		0 & \cos\phi & -\sin\phi \\
		0 & \sin\phi & \cos\phi
	\end{pmatrix},
\end{equation}
\begin{equation}
	R_y(\theta) = \begin{pmatrix}
		\cos\theta & 0 & \sin\theta \\
		0 & 1 & 0 \\
		-\sin\theta & 0 & \cos\theta
	\end{pmatrix},
\end{equation}
\begin{equation}
	R_z(\psi) = \begin{pmatrix}
		\cos\psi & -\sin\psi & 0 \\
		\sin\psi & \cos\psi & 0 \\
		0 & 0 & 1
	\end{pmatrix}.
\end{equation}

这种方法直观易懂,但存在表示不唯一的问题,并且当 $\theta = \pm \frac{\pi}{2}$ 时会出现万向节死锁(Gimbal Lock)现象,从而影响数值计算的稳定。

\subsection{轴–角表示}

轴–角表示方法直接利用旋转轴和旋转角度来描述刚体旋转。设旋转轴为单位向量 
\[
\mathbf{n} = \begin{pmatrix} n_x \\ n_y \\ n_z \end{pmatrix}
\]
和旋转角度 $\theta$,则根据 Rodrigues 公式,旋转矩阵可表示为
\begin{equation}
	R = I \cos\theta + (1-\cos\theta)\, \mathbf{n}\mathbf{n}^T + \sin\theta \, [\mathbf{n}]_\times,
\end{equation}
其中 $I$ 为 $3\times 3$ 单位矩阵,$[\mathbf{n}]_\times$ 为 $\mathbf{n}$ 的反对称矩阵,
\begin{equation}
	[\mathbf{n}]_\times = \begin{pmatrix}
		0 & -n_z & n_y \\
		n_z & 0 & -n_x \\
		-n_y & n_x & 0
	\end{pmatrix}.
\end{equation}

轴–角表示法没有旋转顺序的依赖,因而避免了万向节死锁问题,但在旋转插值和数值优化中,常需转换为其他表示形式处理。

\subsection{四元数表示}

四元数是一种利用4个数字紧凑表示三维旋转的方法。一个单位四元数写作
\begin{equation}
	q = \begin{pmatrix} q_0 \\ q_1 \\ q_2 \\ q_3 \end{pmatrix},
\end{equation}
且必须满足归一化条件
\begin{equation}
	q_0^2 + q_1^2 + q_2^2 + q_3^2 = 1.
\end{equation}
当使用轴–角表示时,其与四元数之间的转换关系为
\begin{equation}
	q_0 = \cos\frac{\theta}{2}, \quad \begin{pmatrix} q_1 \\ q_2 \\ q_3 \end{pmatrix} = \sin\frac{\theta}{2}\, \mathbf{n}.
\end{equation}

由四元数转换得到旋转矩阵的公式为
\begin{equation}
	R = \begin{pmatrix}
		1-2(q_2^2+q_3^2) & 2(q_1q_2 - q_0q_3) & 2(q_0q_2+q_1q_3) \\
		2(q_1q_2+q_0q_3) & 1-2(q_1^2+q_3^2) & 2(q_2q_3 - q_0q_1) \\
		2(q_1q_3 - q_0q_2) & 2(q_0q_1+q_2q_3) & 1-2(q_1^2+q_2^2)
	\end{pmatrix}.
\end{equation}
同时,若以 ZYX 顺序将四元数转换为欧拉角,其公式为
\begin{equation}
	\phi = \arctan2\bigl(2(q_0q_1+q_2q_3),\, 1-2(q_1^2+q_2^2)\bigr),
\end{equation}
\begin{equation}
	\theta = \arcsin\bigl(2(q_0q_2-q_3q_1)\bigr),
\end{equation}
\begin{equation}
	\psi = \arctan2\bigl(2(q_0q_3+q_1q_2),\, 1-2(q_2^2+q_3^2)\bigr).
\end{equation}

需要特别说明的是,虽然使用四元数表示旋转时涉及4个数,但归一化条件限制了它们的自由度为3,与其他旋转表示方法一致。再加上3个平移自由度,构成整体的6D姿态参数。

\subsection{齐次变换}

为了同时描述物体的位置和平移,可以采用齐次变换矩阵同一描述,将旋转矩阵 $R$ 与平移向量 $\mathbf{t}$ 结合,其形式为
\begin{equation}
	H = \begin{pmatrix}
		R & \mathbf{t} \\
		\mathbf{0}^T & 1
	\end{pmatrix} =
	\begin{pmatrix}
		r_{11} & r_{12} & r_{13} & x \\
		r_{21} & r_{22} & r_{23} & y \\
		r_{31} & r_{32} & r_{33} & z \\
		0      & 0      & 0      & 1
	\end{pmatrix},
\end{equation}
其中 $\mathbf{t} = \begin{pmatrix} x \\ y \\ z \end{pmatrix}$ 表示目标在三维空间中的位置,而 $R$ 则描述目标的旋转状态。

总之,本文介绍的各旋转表示方法虽然形式不同,参数的个数也不同,但是由于旋转矩阵与四元数存在额外的约束,使得其自由度为3,与平移向量的3个自由度相加,总共构成6D姿态。


\section{注意力机制}
\label{sec:attention_mechanism}

注意力机制(Attention Mechanism)最初在自然语言处理(NLP)中提出 \cite{vaswani2017},用于强化模型对关键信息的关注。近年来,该机制被成功引入计算机视觉(CV)领域,显著提升了特征表示的精度与鲁棒性。其核心思想是在特征图中为不同通道或不同空间位置分配权重,使得模型倾向于关注更有意义的特征区域或特征通道。

根据实施方式与作用对象的不同,可将视觉领域的注意力机制大致分为以下两大类,并在此基础上衍生出了基于 Transformer 的全局自注意力方法。

\begin{enumerate}
	\item \textbf{CNN 特征域再加权注意力机制}:
	在卷积神经网络(CNN)框架下,对已有的特征图进行加权,典型方法包括通道注意力和空间注意力模块。这类方法通过对特征图维度(如通道或空间)进行显式加权,增强关键信息,抑制无关特征。
	
	\item \textbf{基于 Transformer 的全局自注意力方法}:
	将图像特征序列化(如划分为若干 Patch)并通过多头自注意力机制进行全局特征建模 \cite{dosovitskiy2020image}。这种方法打破了 CNN 特征局部性限制,实现了图像全局依赖关系的高效建模。
\end{enumerate}

\subsection{CNN特征域再加权注意力机制}
此类机制通常以卷积特征图为基础,通过显式加权通道或空间维度来提升特征表达的鲁棒性和区分度。

\subsubsection{通道注意力}
通道注意力通过学习通道级别的权重,对每个通道的特征进行加权,从而突出关键信息、抑制冗余特征。典型通道注意力模如:
\begin{itemize}
	\item \textbf{Squeeze-and-Excitation Networks(SENet)}\cite{senet}:
	利用全局平均池化与全连接层生成通道权重,显著提升了图像分类等任务的性能。
	
	\item \textbf{ECA-Net}\cite{eca-net}:
	通过一维卷积替代全连接层,减少参数量并提高效率,实现更轻量级的通道注意力计算。
	
	\item \textbf{Triplet Attention}\cite{triplet}:
	该机制将通道注意力与其它维度注意力结合在一起,通过跨维度交互(通道、宽度、高度)捕获通道间和空间间的相关性。
\end{itemize}

\subsubsection{空间注意力}
空间注意力关注特征在空间维度上的分布,对特征图中的关键区域进行加权,强调有价值的空间位置。常见实现包括:
\begin{itemize}
	\item \textbf{CBAM(Convolutional Block Attention Module)}\cite{cbam}:
	将通道与空间注意力相结合,对特征图进行双重加权,进一步增强模型的空间特征表达能力。
	
	\item \textbf{BAM(Bottleneck Attention Module)}\cite{bam}:
	在残差网络等结构中增加额外分支,学习空间注意力权重,从而提升图像分类与检测性能。
\end{itemize}

\subsection{基于 Transformer 的全局自注意力方法}
\label{sec:transformer_attn}

在传统的 CNN 注意力机制中,特征的上下文建模往往依赖局部感受野来逐层累积远距离信息,难以高效捕获跨越较大空间范围的依赖关系。与之相比,Transformer 结构\cite{vaswani2017}基于序列建模并以自注意力(Self-Attention)为核心,可以直接对序列各位置之间的全局交互进行建模。自从 Vision Transformer(ViT)\cite{dosovitskiy2020image} 提出后,通过将图像划分为若干 Patch 并以序列形式输入 Transformer 编码器的方式,为视觉任务提供了新的解决思路。

如图~\ref{fig:transformer_arch}所示,基于 Transformer 的图像处理流程通常包含若干关键环节:首先,将输入影像(例如遥感影像)均匀分割为若干 Patch,并对每个 Patch 进行线性投影或小型卷积嵌入,将其映射到固定维度的向量表示,形成一系列 Patch Token。由于 Transformer 自注意力结构本身并不具备位置信息,因此需要对这些 Token 显式添加可学习或固定的位置信息编码,才能在后续处理时保留图像的空间结构。随后,将包含位置信息的 Patch Token 序列输入由多层 Transformer 块堆叠而成的编码器进行特征提取。每个 Transformer 块通常包括多头自注意力、前馈网络、残差连接和层归一化等模块,从而实现对序列中任意两个位置间信息的直接交互。相较于依靠局部卷积逐层扩展感受野的方法,这种全局自注意力机制能够高效建模图像中远距离 Patch 之间的依赖关系。编码器输出的 Token 序列可进一步进入下游任务模块(如分类、检测或分割头),最终完成目标应用的预测或推断。

在 Transformer 中,自注意力(Self-Attention)可表示为对查询(Query)、键(Key)与值(Value)的加权操作。令输入序列的向量集合为 $\mathbf{X}\in \mathbb{R}^{N \times d}$,其中 $N$ 表示序列长度(即 Patch Token 数量),$d$ 表示单个向量的维度。通过可学习的线性变换得到查询矩阵 $\mathbf{Q}$、键矩阵 $\mathbf{K}$ 和值矩阵 $\mathbf{V}$:

\begin{equation}
	\mathbf{Q} = \mathbf{X} \mathbf{W}^Q,\quad
	\mathbf{K} = \mathbf{X} \mathbf{W}^K,\quad
	\mathbf{V} = \mathbf{X} \mathbf{W}^V
\end{equation}
其中 $\mathbf{W}^Q,\ \mathbf{W}^K,\ \mathbf{W}^V\in \mathbb{R}^{d \times d_k}$ 为可学习的参数。随后,自注意力的输出 $\mathrm{Attention}(\mathbf{Q}, \mathbf{K}, \mathbf{V})$ 定义为:

\begin{equation}
	\mathrm{Attention}(\mathbf{Q}, \mathbf{K}, \mathbf{V})
	= \mathrm{softmax}\Bigl(\frac{\mathbf{Q}\mathbf{K}^\top}{\sqrt{d_k}}\Bigr)\mathbf{V}
\end{equation}
通过上述操作,序列中的任何一个位置都可以与其他所有位置进行直接信息交互,从而捕捉全局依赖。


在实际实现中,常使用多头注意力(Multi-Head Attention)来提升表示能力。它将输入分别投影到多个子空间上并进行并行的自注意力计算,最后将结果拼接后再投影回原始维度:

\begin{equation}
	\mathrm{MultiHead}(\mathbf{Q}, \mathbf{K}, \mathbf{V})
	= \mathrm{Concat}\Bigl(\mathrm{head}_1, \ldots, \mathrm{head}_h\Bigr)\mathbf{W}^O
\end{equation}
其中第 $i$ 个注意力头 $\mathrm{head}_i$ 计算如下:

\begin{equation}
	\mathrm{head}_i
	= \mathrm{Attention}\Bigl(\mathbf{Q}\mathbf{W}^Q_i,
	\mathbf{K}\mathbf{W}^K_i,
	\mathbf{V}\mathbf{W}^V_i\Bigr).
\end{equation}


每个 Transformer 块除了多头自注意力之外,还包含一个逐位置(Position-wise)前馈网络(Feed-Forward Network, FFN):

\begin{equation}
	\mathrm{FFN}(\mathbf{x}) = \max(0, \mathbf{x}\mathbf{W}_1 + \mathbf{b}_1)\mathbf{W}_2 + \mathbf{b}_2
\end{equation}
该网络会对每个 Token 向量独立地进行非线性映射,从而增强特征表达能力。与基于卷积的注意力方法相比,Transformer 通过全局自注意力的方式突破了 CNN 主要依赖局部特征的限制,不仅更易捕捉长程依赖,也能在大规模数据上实现更具泛化性的特征学习。在高分辨率、目标尺度多样的视觉场景(例如遥感影像、医学图像等)中,基于 Transformer 的建模方式往往展现出更强的适应性和鲁棒性。随着硬件计算与训练算法的不断发展,越来越多的变体的Transformer结构的出现\cite{sun2023mobilevit,Zhang_2023_CVPR,Chen2024addvit},为视觉任务的研究与应用提供了更丰富的思路和更广阔的空间。针对Transformer计算复杂度过高的问题也有对Transformer轻量化相关的研究,比如EfficientViT\cite{liu2023efficientvit},这个网络也是本文所采用的骨干网络的核心部分。

\begin{figure}[!htb]
	\centering
	\includegraphics[width=1.0\textwidth]{Img/vitillustration.png}
	\caption{Transformer 的结构图示意图}
	\label{fig:transformer_arch}
\end{figure}


\section{YOLO 系列算法与关键点检测}
\label{sec:yolo_keypoint}

\subsection{YOLO 系列算法发展与特点}
YOLO(You Only Look Once)系列算法是近些年目标检测领域最具代表性的单阶段检测器之一,以其实时性高、端到端的优点而被广泛应用 \cite{yolo2016you,yolov3}。其核心思路是将输入图像一次性地通过卷积网络特征提取,并在网络输出层直接回归目标的边界框和类别概率,从而实现高效的检测过程。

随着研究的深入,YOLO 系列版本不断迭代,从 YOLOv2\cite{yolov2}, YOLOv3\cite{yolov3}等到最新的 YOLOv8\cite{yolov8_ultralytics},其结构演化为主干网络(Backbone)、特征整合(Neck)以及检测头(Head)的模式,以适应更复杂的场景与要求并兼顾效率。YOLOv8 不仅对网络结构进行了重新设计,还引入了一系列训练和推理的优化策略,进一步提升了检测精度。

\subsection{关键点检测在 YOLO 框架下的实现}
除了一般的目标检测,YOLO 系列也逐步衍生出在同一网络中同时完成关键点检测的能力,例如检测人体关节点、目标关键特征点等。其思路是在检测头中增加关键点解耦分支,网络在输出目标边界框和类别的同时,也可以回归关键点坐标。

在本研究中,针对空间非合作目标的姿态估计需求,在 YOLOv8 的检测头内增加关键点预测模块,用以回归卫星或碎片表面若干已知结构点的二维位置。一旦在单次前向推理中获得这些关键点,便可通过后端的姿态解算算法快速获得目标的 6D 姿态。

结合注意力机制的特征增强手段,网络在处理弱纹理或高亮对比度背景时,亦能保留关键的结构特征,为后续姿态解算提供高质量、鲁棒的输入数据。
\section{相机模型}
在大多数计算机视觉与图像处理任务中,相机模型通常被简化为针孔模型(Pinhole Camera Model),这是一种常用且有效的理论抽象。设在三维空间中存在一个理想的针孔,用于将外界场景投影到相机的成像平面上。若令世界坐标系中的一点记为 $\mathbf{X} = (X, Y, Z)^\top$,其在图像坐标系中的投影可以在齐次坐标下写作
\begin{equation}
	\begin{pmatrix}
		u \\ v \\ 1
	\end{pmatrix}
	=
	\mathbf{K}\,\begin{bmatrix}
		\mathbf{R} & \mathbf{t}
	\end{bmatrix}
	\begin{pmatrix}
		X \\ Y \\ Z \\ 1
	\end{pmatrix},
\end{equation}
其中 $\mathbf{K}$ 通常称为相机的内参矩阵,用于描述相机的焦距、主点位置等内在性质;$\mathbf{R}$ 与 $\mathbf{t}$ 则为相机的外参,代表相机在世界坐标系中的姿态(旋转与平移)。由于此处使用了齐次坐标系,以上等式在严格意义上是“相等于某个非零标量倍数”——然而在实际推导与实现中,常将其直接视为等式以简化表述。



\section{PnP问题求解概述}
PnP(Perspective-n-Point)问题是在已知若干对空间3D点(空间非合作目标关键点的3D坐标)与其在图像上对应的2D投影点(空间非合作目标关键点的像素坐标)的情况下,求解相机的姿态参数(即位置和平移)的经典问题,如果把相机定义为参考基准则可以通过求解PnP问题得出物体6D姿态。这里通过求解PnP问题就可以解算出空间非合作目标6D姿态。6D姿态解算方法根据求解过程可划分为两大类:基于几何与代数的解法\cite{p3p, EPnP}和基于重投影误差优化的解法\cite{Chen_2022_CVPR,Lipson_2022_CVPR,hu2022perspective},在传统的算法中重投影误差优化的方法通常是在基于几何与代数解法求解出的值的基础上进行进一步地优化,常见的有LM优化,GN优化。前者通过解析几何方法或线性代数直接求解姿态,后者则通过建立重投影误差的非线性优化模型迭代逼近最优解。
\subsection{基于几何与代数的解算方法}
基于几何与代数的方法通过投影过程中的几何关系,构建出相关的方程,通过一定的代数手段求解方程组从而算出目标的6D姿态。
\subsubsection{P3P算法}

P3P(Perspective-Three-Point)算法\cite{p3p}用于在已知三个空间点及其在图像平面上的投影的情况下,求解相机的位姿。该算法基于解析几何,通过建立空间点、相机光心和成像平面之间的几何关系,求解相机的外部参数。

\paragraph{几何原理}

在三维空间中,已知三个不共线的空间点$\mathbf{P}_1$, $\mathbf{P}_2$, $\mathbf{P}_3$,以及它们在图像平面上的投影点$\mathbf{p}_1$, $\mathbf{p}_2$, $\mathbf{p}_3$。相机坐标系下,空间点的坐标与其投影点之间满足透视投影关系:

\begin{equation}
	s_i \mathbf{p}_i = \mathbf{K} [\mathbf{R} | \mathbf{t}] \mathbf{P}_i, \quad i = 1,2,3
\end{equation}


其中,$\mathbf{K}$是相机的内参矩阵,$\mathbf{R}$和$\mathbf{t}$是相机的旋转和平移矩阵,$s_i$是尺度因子。


首先,通过相机的内参矩阵对图像点进行归一化,得到归一化的图像坐标$\mathbf{p}_i'$:

\begin{equation}
	\mathbf{p}_i' = \mathbf{K}^{-1} \mathbf{p}_i
\end{equation}


由于尺度因子$s_i$未知,可以将归一化后的图像坐标看作射线方向的单位向量:

\begin{equation}
	\mathbf{u}_i = \frac{\mathbf{p}_i'}{\|\mathbf{p}_i'\|}
\end{equation}


相机光心$\mathbf{C}$到空间点$\mathbf{P}_i$的距离为$d_i$,则空间点的位置可以表示为:

\begin{equation}
	\mathbf{P}_i = \mathbf{C} + d_i \mathbf{u}_i
\end{equation}

已知空间点之间的距离:

\begin{equation}
	\begin{aligned}
		& a = \|\mathbf{P}_2 - \mathbf{P}_3\| \\
		& b = \|\mathbf{P}_1 - \mathbf{P}_3\| \\
		& c = \|\mathbf{P}_1 - \mathbf{P}_2\|
	\end{aligned}
\end{equation}

将$\mathbf{P}_i$的表达式代入上述距离公式,得到关于$d_1$, $d_2$, $d_3$的方程:

\begin{equation}
	\begin{aligned}
		& a^2 = \left\| d_2 \mathbf{u}_2 - d_3 \mathbf{u}_3 \right\|^2 \\
		& b^2 = \left\| d_1 \mathbf{u}_1 - d_3 \mathbf{u}_3 \right\|^2 \\
		& c^2 = \left\| d_1 \mathbf{u}_1 - d_2 \mathbf{u}_2 \right\|^2
	\end{aligned}
\end{equation}


通过计算各项并利用余弦定理,可以得到:

\begin{equation}
	\begin{aligned}
		& a^2 = d_2^2 + d_3^2 - 2 d_2 d_3 \cos \theta_{23} \\
		& b^2 = d_1^2 + d_3^2 - 2 d_1 d_3 \cos \theta_{13} \\
		& c^2 = d_1^2 + d_2^2 - 2 d_1 d_2 \cos \theta_{12}
	\end{aligned}
\end{equation}


其中,$\cos \theta_{ij}$为射线方向向量之间的夹角的余弦值,计算方法为:

\begin{equation}
	\cos \theta_{ij} = \mathbf{u}_i \cdot \mathbf{u}_j
\end{equation}


现在,得到了三个关于$d_1$, $d_2$, $d_3$的方程。由于只有三个未知数,可以通过消元的方法将其转换为一个关于单个变量的四次多项式方程,通常选择$d_1$为主要未知数。


经过一系列代数运算,可以得到一个标准形式的四次多项式:

\begin{equation}
	A d_1^4 + B d_1^3 + C d_1^2 + D d_1 + E = 0
\end{equation}


使用四次方程的求解公式,可以得到$d_1$的解,然后代入之前的方程求出$d_2$和$d_3$。最终,通过:

\begin{equation}
	\mathbf{C} = \mathbf{P}_i - d_i \mathbf{u}_i内容...
\end{equation}


可以求出相机光心的位置$\mathbf{C}$,相机的旋转矩阵$\mathbf{R}$可以通过对齐空间点和图像点的方向向量来求得。


由于四次方程可能有多达四个实数解,因此P3P算法可能产生多个可能的相机姿态解。为了确定正确的解,需要引入第四个及以上的点,通过验证哪些解符合所有点的投影关系。根据相机和场景的物理特性,排除不合理的解,例如相机不可能位于物体内部。


\subsubsection{EPnP算法}

EPnP(Efficient Perspective-n-Point)算法\cite{EPnP}是一种在已知$n$个空间点及其对应的图像投影点的情况下,快速求解相机位姿的算法。它通过引入虚拟控制点,将非线性问题线性化,从而提高计算效率。

EPnP算法将所有的空间点表示为$M$个虚拟控制点$\{\mathbf{C}_j\}$的线性组合:

\begin{equation}
	\mathbf{P}_i = \sum_{j=1}^{M} \alpha_{ij} \mathbf{C}_j, \quad i = 1,2,\dots,n
\end{equation}

其中,$\alpha_{ij}$是权重系数,满足:

\begin{equation}
	\sum_{j=1}^{M} \alpha_{ij} = 1
\end{equation}

通常,选择$M=4$个控制点,可以是空间点的质心和主方向上的偏移。
在相机坐标系下,控制点的坐标为$\mathbf{C}_j^{c}$,则空间点的相机坐标为:

\begin{equation}
	\mathbf{P}_i^{c} = \sum_{j=1}^{M} \alpha_{ij} \mathbf{C}_j^{c}
\end{equation}

根据透视投影模型,空间点的相机坐标与其图像坐标满足:

\begin{equation}
	s_i \begin{bmatrix}
		u_i \\ v_i \\ 1
	\end{bmatrix} = \mathbf{K} \mathbf{P}_i^{c}
\end{equation}


消除尺度因子$s_i$并使用归一化图像坐标$\mathbf{p}_i'$,得到:

\begin{equation}
	\mathbf{p}_i' \times \mathbf{P}_i^{c} = \mathbf{0}
\end{equation}

将$\mathbf{P}_i^{c}$的表达式代入,得到关于$\mathbf{C}_j^{c}$的线性方程组:

\begin{equation}
	\mathbf{p}_i' \times \left( \sum_{j=1}^{M} \alpha_{ij} \mathbf{C}_j^{c} \right ) = \mathbf{0}, \quad i = 1,2,\dots,n
\end{equation}

展开后,对于每个点得到两个独立的线性方程,总共$2n$个方程,未知数为$3M$个控制点坐标$\mathbf{C}_j^{c}$。当$n \geq 2M$时,可以通过最小二乘法求解。
得到控制点的相机坐标$\mathbf{C}_j^{c}$后,可以通过将控制点的世界坐标$\mathbf{C}_j$和相机坐标$\mathbf{C}_j^{c}$进行刚体变换求解相机的旋转矩阵$\mathbf{R}$和平移向量$\mathbf{t}$。这可以通过求解Kabsch算法或SVD分解实现。

\subsection{基于重投影误差优化的方法}
基于重投影误差优化的方法以重投影误差最小化为目标,通过一定的优化算法来寻找旋转参数和平移参数使得重投影最小化。比如GN\cite{vcolakovic2022hand, zhu2024lodloc}, LM\cite{xu2022rnnpose},等优化算法。

Gauss-Newton 方法是求解非线性最小二乘问题的迭代算法,可视作牛顿法在残差平方和上的一种特例。当最小化目标为 
\begin{equation}
F(\mathbf{x})=\frac{1}{2}\sum_i \bigl\lvert r_i(\mathbf{x})\bigr\rvert^2
\end{equation}
时,其梯度和 Hessian 分别为 
\begin{equation}
\nabla F = J^T \mathbf{r}, 
\end{equation}
\begin{equation}
\nabla^2 F = J^T J + \sum_i r_i \nabla^2 r_i.、
\end{equation}
Gauss-Newton 忽略二阶项 \(\sum_i r_i \nabla^2 r_i\)(假设当残差不大时影响可忽略),以近似 Hessian 为 \(J^T J\)。

因此,更新增量可通过解下面的正则方程得到
\begin{equation}
	J^T J \,\Delta \mathbf{x} \;=\; -\,J^T \mathbf{r}
\end{equation}
\begin{equation}
	\Delta \mathbf{x} \;=\; -\bigl(J^T J\bigr)^{-1} J^T \mathbf{r}
\end{equation}
也可从线性化残差推导:令 
\begin{equation}
	\mathbf{r}(\mathbf{x} + \Delta \mathbf{x}) \approx \mathbf{r}(\mathbf{x}) + J\,\Delta \mathbf{x}
\end{equation}

则目标近似为 \(\bigl\lvert \mathbf{r} + J\,\Delta \mathbf{x}\bigr\rvert^2\),对 \(\Delta \mathbf{x}\) 求极值即可导出上述正则方程。直观来说,Gauss-Newton 每次通过最小二乘解一步到位地逼近当前点的二次近似解。雅可比矩阵 \(J\) 的计算在 6D 姿态问题中与 LM 方法类似,需对旋转和平移的变化求导。由于不引入附加阻尼项,GN 方法每步直接按 \(\bigl(J^T J\bigr)^{-1} J^T \mathbf{r}\) 更新参数。


Levenberg-Marquardt(LM) 算法通过在 Gauss-Newton 法的正则方程中加入阻尼因子来实现稳健更新。给定残差向量 
\(\mathbf{r}(\mathbf{x}) = [r_1(\mathbf{x}), \dots, r_m(\mathbf{x})]^T\) 
(例如每个 \(r_i\) 可以是第 \(i\) 个观测点的重投影误差),以及参数 \(\mathbf{x}\in \mathbb{R}^n\) 表示 6D 姿态(3 维平移 + 3 维旋转参数),定义代价函数为残差平方和 
\begin{equation}
	F(\mathbf{x}) = \tfrac{1}{2}\sum_{i=1}^m \bigl\lvert r_i(\mathbf{x})\bigr\rvert^2
\end{equation}
LM 在第 \(k\) 次迭代时近似地线性化残差:
\begin{equation}
	\mathbf{r}(\mathbf{x}+\Delta \mathbf{x}) \approx \mathbf{r}(\mathbf{x}) + J(\mathbf{x}) \,\Delta \mathbf{x}
\end{equation}

其中 \(J(\mathbf{x})\) 是残差对参数的雅可比矩阵 \(J_{ij} = \partial r_i/\partial x_j\)。为求解 \(\Delta \mathbf{x}\),LM 引入一个阻尼项 \(\lambda\) 改写正则方程为
\begin{equation}
	\bigl(J^T J + \lambda I\bigr)\,\Delta \mathbf{x} \;=\; -\,J^T \mathbf{r}
\end{equation}
\begin{equation}
	\Delta \mathbf{x} 
	= -\bigl(J^{T} J + \lambda I \bigr)^{-1} \, J^{T} \mathbf{r}
\end{equation}
其中 \(J\) 和 \(\mathbf{r}\) 都在当前迭代点 \(\mathbf{x}_k\) 处计算。这里 \(J^T J\) 近似代替了真实 Hessian 矩阵,\(I\) 是单位矩阵,\(\lambda\) 控制了更新步长的调整:当 \(\lambda=0\) 时,退化为 Gauss-Newton;若 \(\lambda\) 很大,则 \(\bigl(J^T J + \lambda I\bigr)^{-1} \approx \tfrac{1}{\lambda}I\),此时 
\(\Delta \mathbf{x} \approx -\tfrac{1}{\lambda} J^T \mathbf{r}\),
相当于沿负梯度方向进行小步长更新(梯度下降)。雅可比矩阵 \(J\) 的具体计算取决于参数化方式,例如 6D 姿态中的旋转部分常用轴角的指数映射(李代数)来计算微小旋转对重投影误差的偏导数。

\section{卡尔曼滤波}
卡尔曼滤波在最小方差意义下对线性高斯系统状态进行最优估计\cite{kalman1960new},通过在每个时刻递推地结合系统状态转移模型与测量模型,得到系统状态的后验估计及其协方差。为了说明其执行过程,可参考图\ref{fig:KalmanFilter}所示。

\begin{figure}[htbp]
	\centering
	\includegraphics[width=1.0\textwidth]{Img/KalmanFilter.pdf}
	\caption{卡尔曼滤波的预测–更新流程示意}
	\label{fig:KalmanFilter}
	\vspace{-3ex}
\end{figure}

假设线性离散系统的状态方程与测量方程分别为
\begin{equation}
	x_n = F_n x_{n-1} + w_{n-1},
\end{equation}
\begin{equation}
	z_n = H_n x_n + v_n,
\end{equation}
其中 \(x_n\) 为系统状态向量,\(z_n\) 为测量向量,\(F_n\) 为状态转移矩阵,\(H_n\) 为测量矩阵,\(w_{n-1}\) 和 \(v_n\) 分别表示过程噪声和测量噪声,均视为零均值高斯白噪声并满足
\begin{equation}
	E[w_{n-1} w_{n-1}^T] = Q_{n-1},
	\quad 
	E[v_n v_n^T] = R_n.
\end{equation}

卡尔曼滤波在每个时刻包含预测与更新两个阶段。预测阶段根据模型对下一时刻的状态和协方差做外推,更新阶段结合新的测量值对状态和协方差做修正。随着时序的不断推进,可获得最小均方误差意义下对系统状态的最优估计。

在处理运动目标时,卡尔曼滤波的“滤波”作用主要体现在对噪声的有效抑制与对目标运动的平滑追踪两个方面。首先,卡尔曼滤波器在预测阶段基于目标的运动学模型(例如匀速或匀加速假设)对下一个时刻的状态做外推,从而在连续时域内保持对目标运动趋势的跟踪。待新的测量值到达后,滤波器再将先验预测与当前测量进行加权融合。卡尔曼增益在融合过程中起到自适应调节的作用:若测量噪声较大,滤波器倾向于对当前测量保持谨慎;若测量较为可靠,则会在更大程度上修正预测偏差。由于这个动态权衡机制,卡尔曼滤波能够在各时刻都以最小均方误差的方式结合历史信息和当前测量,从而在动态跟踪目标的同时抑制瞬时噪声或异常测量引起的跳变,最终实现对运动状态的平滑估计与有效滤波。

在实际应用中,如果系统存在非线性,线性卡尔曼滤波会受到较大限制。对此,通常可采用扩展卡尔曼滤波(Extended Kalman Filter, EKF)或无迹卡尔曼滤波(Unscented Kalman Filter, UKF)对非线性状态进行估计。前者通过对非线性函数进行一阶泰勒展开实现线性化,从而将卡尔曼滤波框架应用于非线性系统;后者则采用无迹变换(Unscented Transform)来对状态和协方差进行更准确的统计线性化,并在一定程度上避免了高阶项截断所带来的误差。这两种滤波器在非线性场景下具有更好的估计性能,是对线性卡尔曼滤波在实践中的有效补充。




\section{本章小结}

本章围绕实现空间非合作目标的6D姿态估计,介绍了所需的理论基础。首先对6D姿态的概念及四种常见旋转表示方法(欧拉角、轴–角、旋转矩阵、四元数)进行了阐述,重点分析了四元数在数值稳定性、计算效率和插值上的优势。然后介绍了注意力机制在计算机视觉中的应用,包括传统CNN框架下的通道和空间注意力以及基于Transformer的全局自注意力方法,并结合当前主流的轻量化Transformer结构讨论了其在大分辨率、高复杂度场景中的潜在应用价值。接下来结合YOLO系列算法的演进历史,阐述了在YOLO框架内实现目标检测与关键点检测的可行思路,为后续利用PnP等经典方法快速解算目标姿态奠定了网络输出层的基础。然后通过相机模型的介绍,说明了从图像平面到三维坐标反演的几何关系及镜头畸变校正对准确测量的重要性;又通过空间目标自由运动假设,阐述了目标在无外力和无外力矩作用下的运动方程,进而为之后采用EKF(扩展卡尔曼滤波)对时序数据进行滤波和融合提供理论支撑。最后,通过对卡尔曼滤波在线性高斯系统中的时序递推原理及其预测–更新过程的说明,为多帧图像数据的综合利用和鲁棒姿态估计奠定了坚实的数学基础。

整体而言,本章从多视角阐述了空间非合作目标6D姿态估计过程中所需的关键理论,包括目标的姿态表示、深度网络中的注意力机制、关键点检测网络方案、PnP与相机几何、自由运动模型及卡尔曼滤波等。上述理论将贯穿于后续的算法设计与实现,为完成高精度、实时性和鲁棒性的在轨目标姿态估计提供了基础理论依据。