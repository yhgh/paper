\chapter{基于关键点重投影误差优化的6D姿态解算方法}
\label{chap:RANSAC-TRO-SQPnP}

\section{引言}
\label{sec:RANSAC-TRO-SQPnP:intro}
在上一章中,已针对单张图像的关键点检测进行了研究,并利用基于注意力机制的关键点检测网络成功获取了非合作目标关键点在图像中的像素坐标。为了从这些像素坐标中精确计算出空间非合作目标的6D姿态,需要将这些2D像素坐标与非合作目标先验的标准6D姿态下的3D关键点结合,通过求解PnP问题进而得出空间非合作目标的完整6D姿态。现有研究中,大多数工作采用P3P\cite{chen2019satellite,s22218541,Guo_2022}或EPnP\cite{li2022learning,huan2020pose,lotti2022investigating}等算法进行求解。
然而,上一章所提出的关键点检测模型预测的关键点像素坐标不可避免地存在一定的噪声,特别是在目标遮挡、光照不足等极端条件下,这种噪声会显著增大。为了更有效地应对PnP求解过程中关键点检测网络预测出的较大噪声,本章采用了 George Terzakis 和 Manolis Lourakis提出的一种基于重投影误差优化的求解方案SQPnP\cite{terzakis2020consistently}。该方法能够在关键点存在较大噪声的情况下,仍然保持较高的姿态估计精度。
进一步考虑到,在极端场景中,关键点检测模型可能产生一定比例的异常点,这些异常点会对估计结果的精度造成显著影响。为了解决这一问题,本章引入了RANSAC算法,通过随机选取关键点子集输入SQPnP,并选择重投影误差最小的子集作为内点(inliers),从而有效排除异常点的干扰,提高姿态估计的鲁棒性。
值得注意的是,SQPnP算法以优化旋转矩阵$R$为核心,而平移向量$t$的估计则是基于$R$求解得出,这导致$t$与$R$之间存在一定程度的耦合,可能会对最终优化收敛产生不利影响。针对这一问题,本章在RANSAC-SQPnP求解的结果基础上,进一步引入信赖域优化(Trust Region Optimization,TRO)技术,通过对旋转和平移参数进行解耦和精细优化,最终使得结果收敛到更高精度。
基于上述分析,本章提出并详细研究了一种结合RANSAC、SQPnP和信赖域优化的PnP求解算法,称为RANSAC-TRO-SQPnP。该算法通过三个关键组件的有机结合,有效应对了实际应用中面临的多重挑战。具体而言,RANSAC组件负责识别并剔除异常点,提高算法的鲁棒性;SQPnP组件提供对关键点噪声的适应能力,保证基础姿态估计的准确性;而TRO组件则对初步估计的6D姿态进行局部精细优化,解决旋转与平移参数的耦合问题。实验结果表明,所提出的RANSAC-TRO SQPnP方法能够显著提高非合作目标6D姿态估计的精度和鲁棒性,特别是在关键点检测存在较大噪声和异常点的复杂场景中表现出色。


\section{基于SQPnP的PnP问题求解}
\label{sec:RANSAC-TRO-SQPnP:SQPnP}
\vspace{1ex}

在空间非合作目标6D姿态测量中,经常使用的算法有EPnP、P3P。然而,由于关键点预测模型得到的关键点通常存在一定噪声,传统几何与代数解法(如EPnP、P3P)在此情况下的鲁棒性相对不足。相比之下,SQPnP算法\cite{terzakis2020consistently}直接以关键点重投影误差最小化为核心思路,通过构建恰当的优化模型并利用序列二次规划进行迭代求解。它在初始值的依赖上要比LM、GN等方法更小,可从任意初始姿态逐步收敛至更准确的解。本节将介绍PnP问题中重投影误差的定义以及针孔相机与畸变模型,用以说明SQPnP算法的基本原理。

\subsection{重投影误差的定义}
给定$n$对匹配的三维关键点$\{\mathbf{X}_i\}_{i=1}^n$和二维图像点$\{\mathbf{p}_i\}_{i=1}^n$,其中$\mathbf{X}_i \in \mathbb{R}^3$为目标坐标系中的三维坐标,$\mathbf{p}_i = [u_i,\;v_i]^\mathsf{T}\in\mathbb{R}^2$为图像平面上的像素坐标。若相机姿态由旋转矩阵$\mathbf{R}$和平移向量$\mathbf{t}$给出,则对第$i$个三维点$\mathbf{X}_i$的投影可分为以下几步:

\begin{enumerate}
	\item 将三维点从目标坐标系变换到相机坐标系:
	\[
	\mathbf{X}_i^{\mathrm{cam}} = \mathbf{R}\,\mathbf{X}_i + \mathbf{t}.
	\]
	\item 在相机坐标系中进行透视投影,得到归一化平面坐标$(x_i, y_i)$:
	\[
	x_i = \frac{X_{i}^{\mathrm{cam}}}{Z_{i}^{\mathrm{cam}}}, \quad
	y_i = \frac{Y_{i}^{\mathrm{cam}}}{Z_{i}^{\mathrm{cam}}}.
	\]
	\item 考虑镜头畸变(包括径向畸变与切向畸变),对$(x_i, y_i)$进行校正得到$(x_{d_i}, y_{d_i})$,具体参见下文畸变模型。
	\item 通过相机内参矩阵$\mathbf{K}$将校正后的归一化坐标映射到像素坐标$(\hat{u}_i, \hat{v}_i)$:
	\[
	\begin{bmatrix}
		\hat{u}_i \\[3pt]
		\hat{v}_i \\[3pt]
		1 
	\end{bmatrix}
	=
	\mathbf{K}
	\begin{bmatrix}
		x_{d_i} \\[3pt]
		y_{d_i} \\[3pt]
		1
	\end{bmatrix}.
	\]
\end{enumerate}

记投影坐标为$\hat{\mathbf{p}}_i = [\hat{u}_i,\;\hat{v}_i]^\mathsf{T}$。\textbf{重投影误差}通常定义为所有关键点投影偏差的均方和,可表示为
\begin{equation}
	E_{\mathrm{reproj}}(\mathbf{R}, \mathbf{t})
	\;=\;
	\frac{1}{n}
	\sum_{i=1}^n 
	\Bigl\|
	\hat{\mathbf{p}}_i - \mathbf{p}_i
	\Bigr\|^2.
	\label{eq:reprojErrDef}
\end{equation}
在6D姿态优化中,往往通过最小化上述重投影误差(或其加权形式),以使估计到的相机姿态尽可能地与真实观测数据吻合。

\subsection{相机模型}
在实际PnP问题中,需要一个准确的成像模型来描述从三维坐标到二维图像平面的投影过程。一般以针孔相机模型为基础,用内参矩阵$\mathbf{K}$和畸变参数$\boldsymbol{\kappa}$来表征相机的成像过程。为描述常见的镜头畸变效应,通常分别考虑径向畸变和切向畸变。

设三维点投影到归一化平面后的坐标为$(x,y)$,其径向距离为
\[
r^2 = x^2 + y^2.
\]
若仅考虑径向畸变,则校正后的坐标$(x_{d},y_{d})$可表示为
\[
\begin{aligned}
	x_{d} &= x\bigl(1 + k_1r^2 + k_2r^4 + k_3r^6\bigr),\\[2pt]
	y_{d} &= y\bigl(1 + k_1r^2 + k_2r^4 + k_3r^6\bigr),
\end{aligned}
\]
其中$k_1, k_2, k_3$为径向畸变系数。

切向畸变则主要由镜头装配或倾斜因素产生,其校正模型为
\[
\begin{aligned}
	x_{d} &= x + 2p_1 xy + p_2\bigl(r^2 + 2x^2\bigr),\\[2pt]
	y_{d} &= y + p_1\bigl(r^2 + 2y^2\bigr) + 2p_2 xy,
\end{aligned}
\]
其中$p_1, p_2$为切向畸变系数。

当径向畸变与切向畸变同时存在时,总的畸变校正可写为
\begin{equation}
	\begin{aligned}
		x_{d} &= x\Bigl(1 + k_1r^2 + k_2r^4 + k_3r^6\Bigr)
		+ 2p_1 xy + p_2\bigl(r^2 + 2x^2\bigr),\\[2pt]
		y_{d} &= y\Bigl(1 + k_1r^2 + k_2r^4 + k_3r^6\Bigr)
		+ p_1\bigl(r^2 + 2y^2\bigr) + 2p_2 xy,
	\end{aligned}
	\label{eq:comprehensive_distortion}
\end{equation}
再通过相机内参$\mathbf{K}$将$(x_{d},y_{d})$映射到最终的像素平面。图~\ref{fig:distortion_patterns}示意了无畸变、仅径向畸变以及同时考虑切向畸变时棋盘格的典型扭曲方式,可见径向畸变主要表现为图像沿半径方向的扩张或收缩,而切向畸变则会导致图像产生局部倾斜或剪切。

\begin{figure}[htbp]
	\centering
	\includegraphics[width=0.8\textwidth]{Img/distortion.png}
	\caption{棋盘格畸变示意图:左图为无畸变情形,中图显示了典型的径向畸变,右图为包含切向畸变的效果。}
	\label{fig:distortion_patterns}
\end{figure}

在上述模型中,具体的相机内参及畸变系数可通过标定获取;当已知这些参数后,PnP问题即通过最小化重投影误差(\ref{eq:reprojErrDef})来估计旋转矩阵$\mathbf{R}$和平移向量$\mathbf{t}$。基于此投影与畸变模型的SQPnP方法,能更好地适应有噪声的关键点观测并在初始值不佳时也能稳健收敛。


\subsection{SQPnP 优化问题建模}
PnP(Perspective-n-Point)问题可表述为:给定 $n$ 对空间点和图像点的对应关系,求解相机的旋转矩阵 $R$ 和平移向量 $t$。这里可以将其转化为一个带约束的优化问题,即在保证 $R$ 为正交矩阵的条件下最小化关键点重投影误差的平方和。令旋转矩阵 $R$ 的9个元素按行(或列)展开组成向量 $x \in \mathbb{R}^9$,那么优化目标可表示为一个关于 $x$ 的二次型,而正交性条件则转化为 $x$ 满足的约束方程组。具体地,消去平移 $t$ 后得到的代价函数可写成:

\begin{equation}
	\min_{x \in \mathbb{R}^9}   x^\top \Omega x
	\quad \text{s.t.} \quad
	h(x) = \mathbf{0}_6
\end{equation}

其中 $\Omega$ 是由观测数据计算得到的 $9 \times 9$ 宽松正定(PSD)矩阵。  
$h(x) = \mathbf{0}_6$ 则表示一组等式约束,使得当 $h(x)=\mathbf{0}$ 时,$x$ 恰好对应一个旋转矩阵。若记
\begin{equation}
	x = 
	\begin{bmatrix}
		r_{1:3} \\[2pt]
		r_{4:6} \\[2pt]
		r_{7:9}
	\end{bmatrix}
	\quad\text{(其中每个 }r_{i:i+2}\text{均为}\mathbb{R}^3\text{向量)},
\end{equation}
则可写出正交性约束:
\begin{equation}
	h(x)  = 
	\begin{bmatrix}
		r_{1:3}^\top r_{1:3}  -  1 \\
		r_{4:6}^\top r_{4:6}  -  1 \\
		r_{7:9}^\top r_{7:9}  -  1 \\
		r_{1:3}^\top r_{4:6} \\
		r_{1:3}^\top r_{7:9} \\
		r_{4:6}^\top r_{7:9}
	\end{bmatrix}
	= 
	\mathbf{0}_6 
\end{equation}
\subsection{SQPnP旋转矩阵 R 的求解}
采用序列二次规划(SQP)方法来求解上述带约束的非线性二次规划问题。SQP 的核心思想是:在迭代的每一步,将当前问题在附近用二次函数近似其目标函数、并用线性函数近似其约束条件,从而形成一个线性约束的二次规划(LCQP)子问题。通过求解该子问题可以得到原问题的一个改进解,如此迭代直至收敛。对于PnP 问题,由于目标函数本身已经是二次型($x^\top \Omega x$),因此在 $R$ 的当前估计值附近,其二次近似就是自身;而约束的线性化则来自对正交性约束 $h(x)=0$ 做一阶泰勒展开。具体来说,设第 $k$ 次迭代的当前解为 $x^{(k)} = r^{(k)}$(对应旋转矩阵 $R^{(k)}$),则令增量 $\delta = x - r^{(k)}$。目标函数关于 $\delta$ 的展开为:
\begin{equation}
	f\bigl(r^{(k)} + \delta\bigr)
	= \bigl(r^{(k)} + \delta\bigr)^{\top} \Omega \bigl(r^{(k)} + \delta\bigr)
	= r^{(k)\top}\Omega r^{(k)}  +  2 r^{(k)\top}\Omega \delta  +  \delta^\top\Omega \delta
\end{equation}
其中常数项 $r^{(k)\top}\Omega r^{(k)}$ 可略去。约束函数的线性化为:
\begin{equation}
	h\bigl(r^{(k)} + \delta\bigr)
	\approx 
	h\bigl(r^{(k)}\bigr)
	+ 
	H_{r^{(k)}} \delta
	= 
	\mathbf{0}_6
\end{equation}
其中 $H_{r^{(k)}} = \frac{\partial h}{\partial x}\big|_{x=r^{(k)}}$ 是在当前点计算的 $6\times 9$ 雅可比矩阵。若当前解 $r^{(k)}$ 已满足约束(即 $h(r^{(k)})=\mathbf{0}$,例如选择初始解时取满足 $R^\top R=I$ 的矩阵),则线性化可简化为
\begin{equation}
	H_{r^{(k)}} \delta  =  \mathbf{0}
\end{equation}
即使 $r^{(k)}$ 初始不完全可行,线性约束
\begin{equation}
	H_{r^{(k)}} \delta 
	=  
	- h(r^{(k)})
\end{equation}
也会逐步将解拉回可行域。这样,第 $k$ 步迭代的局部子问题可表述为:
\begin{equation}
	\min_{\delta \in \mathbb{R}^9} 
	\quad 
	\delta^\top \Omega  \delta 
	+  2 r^{(k)\top} \Omega  \delta
	\quad 
	\text{s.t.} 
	\quad 
	H_{r^{(k)}} \delta 
	=  
	- h(r^{(k)})
\end{equation}

这是一个带线性等式约束的凸二次优化问题。可以通过拉格朗日乘子条件将其转化为线性方程组求解。构建拉格朗日函数
\begin{equation}
	\mathcal{L}(\delta, \lambda) 
	=  
	\delta^\top \Omega  \delta 
	+  
	2 r^{(k)\top}\Omega \delta 
	+ 
	\lambda^\top \bigl(H_{r^{(k)}} \delta + h(r^{(k)})\bigr)
\end{equation}
对 $\delta$ 和拉格朗日乘子向量 $\lambda$ 求导并令其为零,即可得到 KKT 条件所对应的线性方程组:
\begin{equation}
	\begin{pmatrix}
		2 \Omega & H_{r^{(k)}}^\top \\
		H_{r^{(k)}} & \mathbf{0}_{6\times 6}
	\end{pmatrix}
	\begin{pmatrix}
		\delta^* \\
		\lambda^*
	\end{pmatrix}
	= 
	\begin{pmatrix}
		- 2 \Omega r^{(k)} \\
		- h(r^{(k)})
	\end{pmatrix}
\end{equation}

解此线性方程组即可得到优化方向增量 $\delta^*$ 以及对应的拉格朗日乘子 $\lambda^*$。然后将旋转向量更新为
\begin{equation}
	r^{(k+1)} 
	= 
	r^{(k)} + \delta^*
\end{equation}
(必要时可结合步长或信赖域策略保证收敛),并重复上述过程,直到增量范数足够小而收敛。通过这种 SQP 迭代,在每一步都满足(或逐步逼近)旋转矩阵的正交约束,并不断降低目标函数,最终得到满足 $R^\top R=I$ 的最优旋转矩阵解 $R^*$。
\subsection{平移向量$t$的求解}
在确定了旋转矩阵 $R$ 之后,平移向量 $t$ 可以通过最小二乘闭式求解。由于上述代价函数在消去 $t$ 后成为 $x^\top \Omega x$,这实际上等价于:对于任意给定的 $R$,都可直接找到使误差最小的 $t$ 表达式。这由对原始未消元的目标函数关于 $t$ 的偏导为零条件得到。具体而言,利用先前定义的 $A_i$ 和 $Q_i$ 矩阵,令 $\mathbf{r} = \operatorname{vec}(R)$ 为旋转的9维向量,对 $t$ 求导并令梯度为零,可得到如下线性方程:

\begin{equation} 
	\sum_{i=1}^n Q_i \big(A_i \mathbf{r} + t\big) =  \mathbf{0}_3
\end{equation}

其中 $A_i \mathbf{r} = R X_i$ 表示旋转后的第 $i$ 个空间点坐标在相机坐标系下的表示(以向量形式融入等式),$Q_i$ 则与该点的观测有关的权矩阵。将上式整理,可得到关于 $t$ 的线性方程组:

\begin{equation} 
	\Big(\sum_{i=1}^n Q_i\Big)  t = - \sum_{i=1}^n Q_i A_i \mathbf{r}
\end{equation}

在通常情况下,矩阵 $\sum_i Q_i$ 是非奇异的(这相当于所有点的方向约束提供了充分信息)。因此可以直接求解得到

\begin{equation} 
	t = -\Big(\sum_{i=1}^n Q_i\Big)^{-1} \sum_{i=1}^n Q_i A_i \mathbf{r}  
\end{equation}

即 $t$ 关于 $\mathbf{r}$ 为线性关系,可写为 $t = P \mathbf{r}$。其中矩阵 $P = -(\sum_i Q_i)^{-1}(\sum_i Q_i A_i)$ 可以在给定所有观测点后预先计算。当利用上述 SQP 方法得到最终最优旋转 $\mathbf{r}^*$ 时,只需代入此公式即可得到对应的最优平移 $t^*$。由于这一求解过程实质上是对原目标关于 $t$ 的线性最小二乘优化,因而计算高效且解是全局最优的。值得一提的是,在实际实现中,不必每次迭代都重新计算 $t$;通常可以在主迭代外层,当旋转收敛后再一次性求出最终 $t$,或者在需要评估当前解的实际投影误差时临时计算相应的 $t$ 值。

在确定了旋转矩阵 $R$ 之后,平移向量 $t$ 可以通过最小二乘闭式求解。由于上述代价函数在消去 $t$ 后成为 $x^\top \Omega x$,这实际上等价于:对于任意给定的 $R$,都可直接找到使误差最小的 $t$ 表达式。这由对原始未消元的目标函数关于 $t$ 的偏导为零条件得到。具体而言,利用先前定义的 $A_i$ 和 $Q_i$ 矩阵,令 $\mathbf{r} = \operatorname{vec}(R)$ 为旋转的9维向量,对 $t$ 求导并令梯度为零,可得到如下线性方程:

\begin{equation}
	\sum_{i=1}^n Q_i  \big(A_i \mathbf{r} + t\big) = \mathbf{0}_3
\end{equation}

其中 $A_i \mathbf{r} = R X_i$ 表示旋转后的第 $i$ 个空间点坐标在相机坐标系下的表示(以向量形式融入等式),$Q_i$ 则与该点的观测有关的权矩阵。将上式整理,可得到关于 $t$ 的线性方程组:

\begin{equation}
	\Big(\sum_{i=1}^n Q_i\Big) t  = - \sum_{i=1}^n Q_i A_i \mathbf{r}
\end{equation}

在通常情况下,矩阵 $\sum_i Q_i$ 是非奇异的,因此可以直接求解得到

\begin{equation}
	t  = - \Big(\sum_{i=1}^n Q_i\Big)^{-1} \sum_{i=1}^n Q_i A_i \mathbf{r}
\end{equation}

即 $t$ 关于 $\mathbf{r}$ 为线性关系,可写为 $t = P \mathbf{r}$。其中矩阵 $P = -(\sum_i Q_i)^{-1}(\sum_i Q_i A_i)$ 可以在给定所有观测点后预先计算。当利用上述 SQP 方法得到最终最优旋转 $\mathbf{r}^*$ 时,只需代入此公式即可得到对应的最优平移 $t^*$。由于这一求解过程实质上是对原目标关于 $t$ 的线性最小二乘优化,因而计算高效且解是全局最优的。值得一提的是,在实际实现中,不必每次迭代都重新计算 $t$;通常可以在主迭代外层,当旋转收敛后再一次性求出最终 $t$,或者在需要评估当前解的实际投影误差时临时计算相应的 $t$ 值。

然而,$R$ 与 $t$ 之间存在一定的耦合关系。由于 $t$ 是通过最小二乘法根据已确定的 $R$ 求解得到的,这意味着平移向量 $t$ 的优化空间实际上是受到旋转矩阵 $R$ 的影响的。在优化过程中,旋转矩阵 $R$ 的变化直接决定了平移向量 $t$ 的求解方式,这种耦合限制了平移向量的搜索空间,从而影响了 $t$ 的进一步寻优。具体而言,若 $R$ 的求解精度较低,或者在某些特定情况下,平移向量 $t$ 的变化范围被压缩,使得平移的优化空间无法充分探索。后续在求解出的6D姿态上进行的优化算法能够进一步解决这个问题。



\section{RANSAC算法}
\label{sec:ransac_pnp}

RANSAC算法由 Fischler 和 Bolles 于 1981 年提出,用于在含有大量异常值(离群点)的环境下对模型参数进行鲁棒估计。它在数据中反复随机抽取最小数量的样本来拟合模型,并以预先设定的误差阈值判断内点和外点。图~\ref{fig:ransac_flowchart} 展示了本章所采用的 RANSAC 流程:在获取输入数据后,首先对关键变量和计数器进行初始化,包括将迭代次数 \(k\) 置零、内点集合 \(\mathcal{I}_{\text{best}}\) 置空、内点比例 \(\omega_{\text{best}}\) 初始化为 0 以及设置最大迭代次数 \(k_{\max}\) 等。然后在每次迭代中随机选取最小数量的 3D--2D 匹配点(对于 PnP 通常需要 4 或 5 对),求解相机姿态 \(\mathrm{rvec}, \mathrm{tvec}\)。之后计算所有匹配点的关键点重投影误差,并根据阈值 \(\tau\) 判断其是否为内点,从而获得当前迭代的内点集合 \(\mathcal{I}_{i,\text{new}}\)。更新迭代计数后,将该模型与已有最佳模型进行比较:若当前内点数更多(或者内点数相同但总误差更小),则更新最佳模型的内点集合、内点比例与误差等变量,并基于更新后的内点比例 \(\omega_{\text{best}}\) 重新计算满足期望置信度 \(\alpha\) 的最少迭代次数 \(N\)。每次迭代都要检查终止条件,包括迭代次数是否达到或超过 \(k_{\max}\),以及是否超过由内点比例计算得到的迭代次数 \(N\)。若任何条件满足,算法便可停止并输出当前最佳模型及其内点集合,从而减少不必要的迭代。


\begin{figure}[htbp]
	\centering
	\begin{tikzpicture}[node distance=18mm, >=Stealth, semithick, scale=0.8, transform shape]
		% -- 定义样式 --
		\tikzstyle{startstop} = [ellipse, draw, align=center, minimum width=25mm, minimum height=10mm]
		\tikzstyle{process} = [rectangle, draw, align=center, minimum height=12mm, minimum width=45mm]
		\tikzstyle{decision} = [diamond, aspect=2, draw, align=center, inner sep=1pt, minimum width=40mm, minimum height=14mm]
		\tikzstyle{data} = [trapezium, trapezium left angle=70, trapezium right angle=110, minimum width=60mm, minimum height=12mm, draw, align=center]
		
		% 开始
		\node[startstop] (start) {开始};
		
		% 输入框
		\node[data, below of=start, yshift=-5mm] (input) {
			输入
			$\mathcal{I}_{3D}$, $\mathcal{I}_{2D}$
			$\alpha$, $\tau$, $k_{\max}$, $s$
		};
		
		% 初始化:仅列出相关数学变量名称
		\node[process, below of=input, yshift=-2mm] (init) {
			初始化
			$k=0,\; \mathcal{I}_{\text{best}}=\emptyset,\; \omega_{\text{best}}=0,\; E_{\text{best}}=\infty,\; N=k_{\max}$
		};
		
		% 随机选取最小样本,并求解 PnP 得到 $tvec,\, rvec$
		\node[process, below of=init, yshift=-3mm] (sample) {
			随机选取最小样本\\
			求解 PnP得到$\mathrm{rvec}$,$\mathrm{tvec}$
		};
		
		% 根据阈值 $\tau$ 计算重投影误差,估计当前内点集
		\node[process, below of=sample, yshift=-3mm] (model) {
			计算关键点重投影误差\\
			按照阈值$\tau$估计内点集 $\mathcal{I}_{i,\text{new}}$
		};
		
		% 计算误差及统计内点比例,并更新迭代次数计数
		\node[process, below of=model, yshift=-3mm] (evaluate) {
			计算误差并统计 $\omega_i$\\
			$k \leftarrow k + 1$
		};
		
		% 判定是否更新最佳模型
		\node[decision, below of=evaluate, yshift=-14mm, text width=32mm] (better) {
			\scriptsize
			$|I_{i,\text{n}}| > |I_{\text{b}}|$ 
			\scriptsize 或 
			\scriptsize ($|I_{i,\text{n}}| = |I_{\text{b}}|$ 
			\scriptsize 且 $E_{i,\text{n}} < E_{\text{b}}$)
		};
		
		\node[process, right of=better, xshift=55mm, text width=40mm] (update) {
			\scriptsize
			$\mathcal{I}_{\text{best}} = \mathcal{I}_{i,\text{new}}$\\
			\scriptsize
			$\omega_{\text{best}} = \frac{|\mathcal{I}_{\text{best}}|}{N_{\text{kpt}}}$\\
			\scriptsize
			$N = \frac{\ln(1-\alpha)}{\ln(1 - \omega_{\text{best}}^s)}$
		};
		
		% 终止条件判定
		\node[decision, below of=better, yshift=-27mm, text width=30mm] (stop) {
			\scriptsize
			$k \ge k_{\max}$ 或 
			\scriptsize
			$k \ge N$ 或 
			\scriptsize
		};
		
		% 输出框
		\node[data, below of=stop, yshift=-20mm] (output) {
			输出\\
			$\mathrm{rvec}, \mathrm{tvec}$\\
			最佳内点集 $\mathcal{I}_{\text{best}}$
		};
		
		% 结束
		\node[startstop, below of=output, yshift=-8mm] (end) {
			结束
		};
		
		% 连线
		\draw[->] (start) -- (input);
		\draw[->] (input) -- (init);
		\draw[->] (init) -- (sample);
		\draw[->] (sample) -- (model);
		\draw[->] (model) -- (evaluate);
		\draw[->] (evaluate) -- (better);
		\draw[->] (better) -- node[above, xshift=3mm] {是} (update);
		\draw[->] (better) -- node[left] {否} (stop);
		\draw[->] (update) |- (stop);
		\draw[->] (stop) -- node[right] {是} (output);
		\draw[->] (output) -- (end);
		\coordinate (backLoop) at ($(sample.west)+(-6cm,0)$);
		\draw[->] (stop.west) node[above, xshift=-2.5mm]{否} -| (backLoop) -- (sample.west);
	\end{tikzpicture}
	\caption{RANSAC 算法流程图}
	\label{fig:ransac_flowchart}
\end{figure}

\begin{table}[h!]
	\centering
	\caption{RANSAC 中主要参数与变量说明}
	\label{tab:ransac_params}
	\begin{tabular}{cc}
		\toprule
		\textbf{符号} & \textbf{意义}\\
		\midrule
		$\mathcal{I}_{3D}, \mathcal{I}_{2D}$ & 3D--2D 点集合 \\
		$\alpha$ & 期望的置信度\\
		$\tau$ & 关键点重投影误差阈值 \\
		$k_{\max}$ & 最大迭代次数 \\
		$s$ & 最小样本大小\\
		$k$ & 当前迭代次数 \\
		$N$ & 根据 $\alpha$ 和内点比例动态计算得到的最少迭代次数 \\
		$\omega_{\text{best}}$ & 当前最佳模型的内点比例 \\
		$E_{\text{best}}$ & 当前最佳模型的总误差或平均误差 \\
		\bottomrule
	\end{tabular}
\end{table}









\section{信赖域优化算法}

由于关键点噪声的干扰,加上PnP求解算法的一定局限。导致对于解算出的6D姿态会出现更大的偏差,有不少研究基于关键点重投影误差最小化对其进行优化以获得更为精确的解\cite{10297555,guo2024joint}。信赖域优化算法(Trust Region Optimization,TRO)是一种基于信赖域思想的非线性优化方法。信赖域算法通过在每次迭代中限制步长范围,可可靠地求解约束优化问题,确保即使在大规模非线性约束条件下算法也能保持收敛性和高效性​\cite{wen2024augmented}。另外,TRO针对复杂的目标函数优化(包括非凸或多模态目标函数),通过动态调整信赖域半径在全局搜索与局部模型近似之间取得平衡,既避免过大步长导致的发散,又防止过小步长陷入局部极值,从而实现对复杂目标的有效优化​\cite{asmtr}。基于上述特点,信赖域优化方法已成为解决诸如约束规划、非凸优化等难题的有力工具,在学术研究和工程实践中具有重要地位。
考虑到关键点重投影误差优化也是一种非线性优化问题,其目标函数的非凸。这里将TRO用于优化6D姿态。TRO在每次迭代中,它通过构建当前点处目标函数的二次近似模型,并在一个限定的“信赖域”半径范围内求解该模型的极小值,从而获得试探步长。然后,通过计算目标函数实际值降低量与模型预测降低量的比值 $\rho_k$,判断该步长是否可信。如果 $\rho_k$ 值足够大(如达到预设阈值 $\eta$),则认为模型对实际函数的近似较好,接受该步长并更新当前解;否则,拒绝该步长(保持当前解不变)。无论步长是否被接受,算法都会根据 $\rho_k$ 的大小自适应地调整信赖域半径(如果步长效果理想则适当增大信赖域范围,以加快收敛;若步长效果不佳则缩小信赖域范围,以提高下一步迭代的可靠性)。如此迭代进行,直至满足收敛判据(例如梯度范数足够小或步长长度足够小),算法终止并得到最终优化结果。


\begin{figure}[htbp]
	\centering
	\begin{tikzpicture}[
		>=stealth,
		node distance=2.2cm,   % 节点垂直/水平距离
		every node/.style={font=\small},
		startstop/.style={
			ellipse,
			draw,
			align=center,
			minimum width=2.5em,
			minimum height=1.5em
		},
		process/.style={
			rectangle,
			draw,
			align=center,
			minimum width=6em,
			minimum height=2em
		},
		decision/.style={
			diamond,
			aspect=2,
			draw,
			align=center,
			inner sep=1pt,
			text width=9em,
			minimum height=3em
		}
		]
		
		% 开始
		\node (start) [startstop] {开始};
		
		% 初始化
		\node (init) [process, below of=start, yshift=-0.5cm] {
			初始化参数 $x_0$ 及信赖域半径 $\Delta_0$\\
			令 $\mathrm{nfev} = 0$
		};
		
		% 构建子问题
		\node (subprob) [process, below of=init, yshift=-0.5cm] {
			构建二次模型\\
			并求解信赖域子问题\\
			得到试探步长 $\mathbf{p}_k$
		};
		
		% 评估目标函数
		\node (eval) [process, below of=subprob, yshift=-0.5cm] {
			评估目标函数\\
			$\mathrm{nfev} \leftarrow \mathrm{nfev} + 1$\\
			计算实际下降与预测下降比值 $\rho_k$
		};
		
		% 判断步长可接受?
		\node (accept) [decision, below of=eval, yshift=-0.5cm] {
			$\rho_k \ge \eta$
		};
		
		% 接受步长
		\node (updateX) [process, right of=accept, xshift=3.2cm, yshift=-1.5cm] {
			接受步长:\\
			$x_{k+1} = x_k + \mathbf{p}_k$
		};
		
		% 拒绝步长 - 更改此处的xshift参数,从-3.2cm改为-4.5cm
		\node (rejectStep) [process, left of=accept, xshift=-3.5cm, yshift=-1.5cm] {
			拒绝步长:\\
			$x_{k+1} = x_k$
		};
		
		% 调整信赖域半径
		\node (radius) [process, below of=accept, yshift=-0.5cm] {
			根据 $\rho_k$\\
			调整信赖域半径 $\Delta_{k+1}$\\
		};
		
		% 终止条件
		\node (stopcond) [decision, below of=radius, yshift=-1.5cm, text width=10em] {
			$\mathrm{nfev} \ge \mathrm{nfev}_\mathrm{max}$\\
			或 $\|\nabla f\|_\infty \le \texttt{gtol}$\\
			或 $\|x_{k+1} - x_k\| \le \texttt{xtol}$
		};
		
		% 结束
		\node (stop) [startstop, below of=stopcond, yshift=-1.5cm] {
			结束
		};
		
		% 连线
		\draw[->] (start) -- (init);
		\draw[->] (init) -- (subprob);
		\draw[->] (subprob) -- (eval);
		\draw[->] (eval) -- (accept);
		
		\draw[->] (accept) -| node[anchor=south, near start]{是} (updateX);
		\draw[->] (accept) -| node[anchor=south, near start]{否} (rejectStep);
		
		\draw[->] (updateX) |- (radius);
		\draw[->] (rejectStep) |- (radius);
		
		\draw[->] (radius) -- (stopcond);
		
		% 如果满足终止条件则结束
		\draw[->] (stopcond) -- node[right, midway]{是} (stop);
		
		% 修改后的路径,改为-7增加左侧距离
		\draw[->] (stopcond) -- ++(-8,0) node[midway, above]{否} |- (subprob);
		
	\end{tikzpicture}
	\caption{信赖域优化算法流程图}
	\label{fig:TRO_flowchart}
\end{figure}


如图~\ref{fig:TRO_flowchart}所示,信赖域优化算法从初始化步骤开始:在每次迭代中首先构建当前点处目标函数的二次近似模型,并在给定的信赖域半径 $\Delta_k$ 约束下求解该近似子问题以得到候选步长 $\mathbf{p}_k$。然后,计算实际函数值下降量与预测下降量的比值 $\rho_k$ 来评估步长的有效性。如果 $\rho_k \ge \eta$(其中$\eta$为步长接受阈值),则接受该步长(更新 $x_{k+1} = x_k + \mathbf{p}_k$);否则拒绝该步长(令 $x_{k+1} = x_k$ 不变)。接着,根据 $\rho_k$ 的大小动态调整信赖域半径$\Delta_{k+1}$:当步长效果理想($\rho_k$ 较大)时增大信赖域;当步长效果不佳($\rho_k$ 较低)时则缩小信赖域。然后检查终止条件:函数评估次数 $\mathrm{nfev}$ 是否超过最大限制 $\mathrm{nfev_{max}}$,梯度无穷范数 $\|\nabla f\|_\infty$ 是否小于梯度容差 $\texttt{gtol}$,或步长范数 $\|x_{k+1} - x_k\|$ 是否小于步长容差 $\texttt{xtol}$。如果满足任一终止条件则结束;若不满足,则返回到"构建二次模型并求解信赖域子问题"节点,开始下一轮迭代。通过不断修正步长并调整信赖域范围,最终可得到使目标函数达到最优或次优的参数估计值。


相对于经典的 LM(Levenberg-Marquardt)算法,信赖域优化方法在收敛的鲁棒性方面具有一定优势。LM 方法实质上在每次迭代中引入阻尼因子调整步长,但其阻尼因子的选取对收敛速度和稳定性有较大影响,需要经验调节。相比之下,信赖域算法通过内部机制自适应地调整信赖域半径,无需人工干预即可控制每步迭代的步长大小,因此对初始值的敏感性更低,能够在病态问题中提供更稳定可靠的收敛表现。


\section{RANSAC-TRO SQPnP算法}
\label{sec:RANSAC-TRO-SQPnP:Algorithm}

如伪代码~\ref{alg:RANSAC_TRO_SQPnP}所示,RANSAC-TRO SQPnP 的核心流程包括以下步骤:
\begin{itemize}
	\item \textbf{RANSAC 采样}:从原始点集中随机抽取少量 3D--2D 对,应满足 PnP 最小解的需求。计算其位姿并计算内点数量或关键点重投影误差,重复多次,得到当前最优的位姿初值。
	\item \textbf{提取内点}:根据关键点重投影误差阈值 $\tau$,将关键点重投影误差较小的点视为内点,形成更可靠的子集。
	\item \textbf{TRO 细化}:针对上一阶段得到的旋转和平移初解,利用信赖域优化(TRO)策略,在内点集合上进行非线性最小二乘优化,迭代更新位姿直至收敛或达到迭代上限。
\end{itemize}

该伪代码\autoref{alg:RANSAC_TRO_SQPnP}可分为两大循环:外层循环执行 RANSAC 随机采样与初值估计,内层循环则执行 TRO 对初值的精细化。在 TRO 步骤中,每次迭代都会计算当前的残差和雅可比,并依此构造或近似出局部二次模型,然后在受限的"信赖域"半径 $\Delta$ 内寻找步长。如果实际误差改进和预测改进的比率 $\rho$ 较大,则认为方向与步长有效,进而可能扩大信赖域;否则就收缩信赖域,重新搜索更小步长,从而在局部寻优时保证稳定性。信赖域优化的参数如下:
\begin{itemize}
	\item \textbf{$\Delta$}(信赖域半径):
	用于限制当次迭代的步长大小,若步长超过 $\Delta$,则会被截断在该半径内。若某次迭代的实际改进率 $\rho$ 较高,说明模型对目标函数的预测准确、步长有效,则在下一次迭代可适当增大 $\Delta$;反之则减小。
	
	\item {\texttt{ftol}、\texttt{xtol}、\texttt{gtol}}:
	在代码默认设置为 $10^{-8}$,分别控制收敛判据中函数值、参数更新量和梯度范数的容差。具体含义是,若目标函数(关键点重投影误差平方和)的相对变化量小于 \texttt{ftol},则判定收敛;若步长(参数更新量)的范数小于 \texttt{xtol} 与当前参数范数的乘积,则视为收敛;若梯度的无穷范数已足够小,则满足一阶最优性。
	
	\item \texttt{max\_nfev}:
	最大函数评估(关键点重投影误差计算)次数;若超限仍未收敛,则退出,以防在极端情况下陷入过多不收敛迭代。若用户不指定,常以 $100\times\text{参数维度}$ 的经验值做默认限制。
	
	\item $\rho$(改进率):
	反映了目标函数"实际误差下降量"与"二次模型预测下降量"的比值,用于判断当次迭代步长是否可接受,并据此更新 $\Delta$。当 $\rho$ 较大时,说明模型预测与实际误差相符,步长有效;当 $\rho$ 很小甚至为负值时,则表示步长无效,需要缩小信赖域。
	
	\item \texttt{alpha}(阻尼系数):
	在部分实现中会结合高斯牛顿或 LM 思路,引入 \texttt{alpha} 来缓解病态问题。代码里设为很小值,并结合 $\Delta$ 更新做简单调节,但不一定需要像 LM 那样反复迭代调试。
\end{itemize}

将 RANSAC 的内点集合带入到 TRO 的非线性最小二乘迭代中,结合上述各项参数的综合调控,可在局部寻优阶段获得更稳健的步长搜索与数值收敛。若初始估计较好,往往只需较少迭代便可达到满意的旋转和平移精度;若初值不够好或存在噪声与外点,则通过 TRO 优化 RANSAC SQPnP,仍可获得鲁棒稳定的最终解。
%\begin{algorithm}[!htbp]
%	\caption{RANSAC-TRO SQPnP}
%	\label{alg:RANSAC_TRO_SQPnP}
%	\begin{spacing}{0.9}  % 将行间距缩小到 0.9
%		\footnotesize        % 如果需要进一步缩小,可以换成 \scriptsize
%		\begin{algorithmic}[1]
%			\Procedure{RANSAC-TRO SQPnP}{$\mathbf{P},\mathbf{p},\mathbf{K},n,\tau,k_{\max},\epsilon_f,\epsilon_x,\epsilon_g,n_{\max}$}
%			\Comment{\parbox[t]{.85\linewidth}{
%					\textbf{输入:} 
%					$\mathbf{P}$: 3D点集, 
%					$\mathbf{p}$: 2D点集, 
%					$\mathbf{K}$: 相机内参矩阵, 
%					$n$: 每次随机采样的点数, 
%					$\tau$: 误差阈值, 
%					$k_{\max}$: 最大迭代次数, 
%					$\epsilon_f$: 函数容差, 
%					$\epsilon_x$: 参数容差, 
%					$\epsilon_g$: 梯度容差, 
%					$n_{\max}$: 最大函数评估次数.\\[4pt]
%					\textbf{输出:} 
%					$\mathbf{r}^*$: 最优旋转向量, 
%					$\mathbf{t}^*$: 最优平移向量, 
%					$\mathcal{I}^*$: 最佳内点集合
%			}}
%			
%			\State 初始化 $e^* \gets \texttt{MAX\_DOUBLE},\;\mathcal{I}^* \gets \emptyset$
%			
%			\For{$k \gets 1$ to $k_{\max}$}
%			\State 随机采样 $\mathcal{I}_k \gets \texttt{RandomSample}(n, |\mathbf{P}|)$
%			\State 提取对应点集 $\mathbf{P}_k \gets \mathbf{P}[\mathcal{I}_k]$, $\mathbf{p}_k \gets \mathbf{p}[\mathcal{I}_k]$
%			\State 姿态估计 $\mathbf{r}_k, \mathbf{t}_k \gets \texttt{SQPnP}(\mathbf{P}_k, \mathbf{p}_k, \mathbf{K})$
%			\State 计算误差 $e_k \gets \texttt{ReprojectError}(\mathbf{P}[\mathcal{I}_k], \mathbf{p}[\mathcal{I}_k], \mathbf{K}, \mathbf{r}_k, \mathbf{t}_k)$
%			\If{$e_k < e^*$}
%			\State 更新最优值 $e^* \gets e_k,\;\mathcal{I}^* \gets \mathcal{I}_k$
%			\State 存储最优姿态 $\mathbf{r}^* \gets \mathbf{r}_k,\;\mathbf{t}^* \gets \mathbf{t}_k$
%			\EndIf
%			\EndFor
%			
%			\State 使用最佳内点集合 $\mathbf{P}^* \gets \mathbf{P}[\mathcal{I}^*],\;\mathbf{p}^* \gets \mathbf{p}[\mathcal{I}^*]$
%			\State 初始化优化参数 $\mathbf{x} \gets [\mathbf{q}^*, \mathbf{t}^*]$
%			\State 计算初始误差 $\mathbf{f} \gets \texttt{ReprojectError}(\mathbf{P}^*, \mathbf{p}^*, \mathbf{K}, \mathbf{x}[0], \mathbf{x}[1])$
%			\State 计算雅可比矩阵 $\mathbf{J} \gets \texttt{Jacobian}(\mathbf{P}^*, \mathbf{p}^*, \mathbf{K}, \mathbf{x}[0], \mathbf{x}[1])$
%			\State 初始化计数器 $n_f, n_J \gets 1, 1$, $\mathbf{g} \gets \mathbf{J}^\top \mathbf{f}$, $i \gets 0$, $s \gets \texttt{None}$
%			
%			\While{未满足终止条件}
%			\If{$\|\mathbf{g}\|_\infty < \epsilon_g$}
%			\State $s \gets 1$
%			\State \textbf{break}
%			\EndIf
%			\State $\mathbf{g}_h, \mathbf{J}_h, \mathbf{U}, \mathbf{s}, \mathbf{V}, \mathbf{u}_f \gets \texttt{ComputeAproximateProblem}(\mathbf{J}, \mathbf{f})$
%			\While{$\rho \leq 0$ \textbf{ and } $n_f < n_{\max}$}
%			\State $\boldsymbol{\delta}_h, \boldsymbol{\delta}, \mathbf{x}_{\text{new}}, \mathbf{f}_{\text{new}} \gets \texttt{ComputeStep}(\mathbf{g}_h, \mathbf{J}_h, \mathbf{U}, \mathbf{s}, \mathbf{V}, \mathbf{u}_f, \mathbf{x}, \Delta)$
%			\State $n_f \gets n_f + 1$
%			\If{\textbf{not} \texttt{IsFinite}($\mathbf{f}_{\text{new}}$)}
%			\State $\Delta \gets \Delta \cdot 0.5$
%			\State \textbf{continue}
%			\EndIf
%			\State $\rho, \Delta_{\text{new}}, r, \|\boldsymbol{\delta}\| \gets \texttt{ComputeRatio}(\mathbf{f}, \mathbf{f}_{\text{new}}, \boldsymbol{\delta}_h, \Delta)$
%			\If{满足终止条件}
%			\State \textbf{break}
%			\EndIf
%			\State \texttt{Update}($\Delta$, $r$)
%			\EndWhile
%			
%			\If{$\rho > 0$}
%			\State 更新参数 $\mathbf{x}, \mathbf{f} \gets \mathbf{x}_{\text{new}}, \mathbf{f}_{\text{new}}$
%			\State 更新旋转和平移 $\mathbf{r}, \mathbf{t} \gets \mathbf{x}[0],\;\mathbf{x}[1]$
%			\State 重新计算雅可比矩阵和梯度 
%			$\mathbf{J}, \mathbf{g} \gets \texttt{Jacobian}(\mathbf{P}^*, \mathbf{p}^*, \mathbf{K}, \mathbf{r}, \mathbf{t}),\; \mathbf{J}^\top \mathbf{f}$
%			\State $n_J \gets n_J + 1$
%			\EndIf
%			\State $i \gets i + 1$
%			\EndWhile
%			
%			\If{$s$ 为 \texttt{None}}
%			\State $s \gets 0$
%			\EndIf
%			
%			\State 更新最终姿态 $\mathbf{r}^*, \mathbf{t}^* \gets \mathbf{x}[0], \mathbf{x}[1]$
%			\State \textbf{return} $\mathbf{r}^*, \mathbf{t}^*, \mathcal{I}^*$
%			\EndProcedure
%		\end{algorithmic}
%	\end{spacing}
%\end{algorithm}





\section{实验分析}
\label{sec:RANSAC-TRO-SQPnP:ExperimentCompare}
\vspace{1ex}
\subsection{SQPnP的参数设置}
表 \ref{tab:sqpnp-params} 列出了 SQPnP 算法中与迭代收敛和数值稳定性相关的主要参数设置。大多数收敛阈值如 \(\epsilon_{\mathrm{sqp}}\)、\(\epsilon_{\mathrm{orth}}\)、\(\epsilon_{\mathrm{vec}}\)、\(\epsilon_{\mathrm{err}}\) 等,都是为了控制迭代更新、旋转正交性、多解筛选以及关键点重投影误差等方面的精度,确保在满足一定精度的前提下及时停止计算,从而避免无意义的过度迭代。退化判断阈值 \(\epsilon_{\mathrm{var}}\) 用于识别输入数据是否有较大共线、共面或分布极为不平衡等情况,从而在必要时提前终止或进行特殊处理。最大迭代次数 \(K_{\mathrm{max}}\) 则在确保精度的同时,防止算法在极端场景下因数值问题而陷入死循环或过度迭代。通过合理设置这些参数,可以在保证求解精度的同时,提高算法的稳定性和效率。
\begin{table}[htbp]
	\centering
	\caption{SQPnP的主要参数设置}
	\label{tab:sqpnp-params}
	\begin{tabular}{l c p{7.5cm}}
		\toprule
		参数 & 数值 & 解释 \\
		\midrule
		
		$\epsilon_{\mathrm{sqp}}$      & $10^{-10}$ & 
		SQP 迭代更新量 $\|\delta\|^2$ 的收敛阈值,小于此值认为已充分收敛。\\[3pt]
		
		$\epsilon_{\mathrm{orth}}$     & $10^{-8}$  & 
		判断当前旋转矩阵正交误差是否可忽略,例如 $\|R^\top R - I\|$ 是否足够小。\\[3pt]
		
		$\epsilon_{\mathrm{vec}}$      & $10^{-10}$ &
		判断两个旋转向量是否几乎相同,以便在多解筛选或重复解检测时去重。\\[3pt]
		
		$\epsilon_{\mathrm{err}}$      & $10^{-6}$  &
		判断两次关键点重投影误差的平方和是否近似相等,避免无意义的细微迭代。\\[3pt]
		
		$\epsilon_{\mathrm{var}}$      & $10^{-5}$  &
		判断输入点分布或观测矩阵是否退化(如共线、共面);低于此值可能无法得到有效解。\\[3pt]
		
		$K_{\mathrm{max}}$             & 15         &
		SQP 算法最大迭代步数上限,用于防止在极端情况下陷入死循环或过度迭代。\\
		\bottomrule
	\end{tabular}
\end{table}


\subsection{算法参数的设置}
为了求解PnP问题,首先要按照6D姿态解算场景的相机内参进行设置,这里的相机内参采用的是SPEED/SPEED+数据集以及自建数据集渲染时所设置的相机参数,如表~\ref{tab:camera_params}所示。

\begin{table}[htbp]
	\centering
	\caption{PnP求解需要的相机参数}
	\label{tab:camera_params_solvepnp}
	\begin{tabular}{lccc}
		\toprule
		\textbf{参数名称} & \textbf{符号} & \textbf{数值} & \textbf{单位} \\
		\midrule
		焦距(水平) & $f_x$ & \(3.003\times10^{3}\) & pixel \\
		焦距(垂直) & $f_y$ & \(3.003\times10^{3}\) & pixel \\
		主点横坐标 & $c_x$ & 960 & pixel \\
		主点纵坐标 & $c_y$ & 600 & pixel \\
		径向畸变系数1 & $k_1$ & \(-2.238\times10^{-1}\) & -- \\
		径向畸变系数2 & $k_2$ & \(5.141\times10^{-1}\) & -- \\
		切向畸变系数1 & $p_1$ & \(-6.650\times10^{-4}\) & -- \\
		切向畸变系数2 & $p_2$ & \(-2.140\times10^{-4}\) & -- \\
		径向畸变系数3 & $k_3$ & \(-1.312\times10^{-1}\) & -- \\
		\bottomrule
	\end{tabular}
\end{table}





% 示例性内容开始
如图~\ref{fig:ransac_flowchart}所示,RANSAC算法在输入阶段会根据置信度 $\alpha$ 和内点阈值 $\tau$ 对采样结果进行判别。为兼顾算法的精度和效率,需合理设置该算法的参数,例如置信度 $\alpha$ 直接影响抽样成功的概率,而 $\tau$ 决定了判定内点与否的误差门槛。此外,还需设定最大迭代次数 $k_{\max}$,以避免在极端情况下过多的迭代。具体设置如表~\ref{tab:ransac_params_setting} 所示。


\begin{table}[h!]
	\centering
	\caption{RANSAC的参数设置}
	\label{tab:ransac_params_setting}
	\begin{tabular}{ccc}
		\toprule
		参数 & 数值 & 含义 \\
		\midrule
		$\alpha$   & 0.99   & 置信度 \\
		$\tau$     & 8      & 内点阈值 \\
		$k_{\max}$ & 10000  & 最大迭代次数 \\
		\bottomrule
	\end{tabular}
\end{table}


如图~\ref{fig:TRO_flowchart}所示,信赖域优化算法的具体实现包含以下关键参数:在本章的研究内容下,初始参数向量 $x_0$ 是PnP解算产生的6D姿态参数(3维旋转向量和3维平移向量)。我们优化的参数是相机姿态参数,其中3维旋转向量表示相机的方向(使用Rodrigues旋转表示法),3维平移向量表示相机的空间位置坐标。初始信赖域半径 $\Delta_0$ 设置为初始参数向量的范数 $\|x_0\|$,这种动态设置使算法能够自适应地调整搜索范围。在每次迭代中,算法评估试探步长 $\mathbf{p}_k$ 的效果,通过比值 $\rho_k$ 判断(步长接受阈值 $\eta$ 在本实现中采用了0值,表示只要步长能带来任何程度的目标函数减少就接受,同时通过信赖域半径的动态调整确保算法稳定性)。算法的终止条件由三个主要参数控制:最大函数评估次数设为6000、梯度容差 $\texttt{gtol}$设为 $10^{-8}$,步长容差 $\texttt{xtol}$设为 $10^{-8}$。
\begin{table}[htbp]
	\centering
	\caption{信赖域优化算法参数及初始值}
	\label{tab:tr_parameters}
	\begin{tabular}{llp{8cm}}
		\hline
		参数 & 数值 & 说明 \\
		\hline
		$\Delta_0$ & $\|x_0\|$ & 初始信赖域半径\\
		$\eta$ & 0 & 步长接受阈值 \\
		$\text{ftol}$ & $10^{-8}$ & 函数值相对变化的容差 \\
		$\text{xtol}$ & $10^{-8}$ & 参数变化的容差 \\
		$\text{gtol}$ & $10^{-8}$ & 梯度无穷范数的容差 \\
		$\text{nfev}_\text{max}$ & 6000 & 最大函数评估次数 \\
		\hline
	\end{tabular}
\end{table}


\subsection{PnP 方法在 SPEED+数据集上的对比}
\begin{table}[!htbp]
	\centering
	\caption{在SPEED+合成数据集上的PnP算法对比}
	\setlength{\tabcolsep}{4.7mm}{
		\begin{tabular}{lccc}
			\toprule
			方法 & $score_{\text{ort}}^+$ & $score_{\text{pst}}^+$ & $score^+$  \\ \midrule
			EPnP \citep{EPnP} & 0.02539 & 0.01356 & 0.03896 \\
			SQPnP \citep{terzakis2020consistently} & 0.02227 & 0.01247  & 0.03474 \\
			RANSAC SQPnP & 0.01877 & 0.00860  & 0.02737 \\
			LM \citep{lm} SQPnP & 0.02227 & 0.01446 & 0.03673 \\
			RANSAC-LM SQPnP & \textbf{0.01875} & 0.00865 & 0.02739 \\
			TRO \citep{trf} SQPnP & 0.02227 & 0.01251 & 0.03478 \\
			\textbf{Ours (RANSAC-TRO SQPnP)} & 0.01878 & \textbf{0.00850} & \textbf{0.02728} \\ 
			\bottomrule
	\end{tabular}}
	\label{tab:PnPCmp}
\end{table}

从表 \ref{tab:PnPCmp} 可见,与 EPnP 相比,SQPnP 在三个指标上都有明显提升,表明 SQPnP 的优化建模,优化求解手段对应关键点噪声更为鲁棒,使得求解精度明显高于 EPnP;RANSAC + SQPnP 的精度比单纯 SQPnP + LM 或 TRO 更优,说明异常点(外点)的存在确实会显著影响优化求解的过程,使其不能够收敛到理想的精度。如果首先选择去除外点则可以在不进行优化的情况下达到更优的精度。在 RANSAC-SQPnP 的基础上,对初始姿态进行 LM 和 TRO 两种后续优化时,LM 在 $score_{\text{ort}}^+$ 上略好一些,但 $score_{\text{pst}}^+$ 更差,导致最终 $score^+$ 下降;而 TRO 则在 $score_{\text{pst}}^+$ 与整体 $score^+$ 指标上优势更显著;这说明相对于 LM 优化,TRO 通过限制在信赖域上的优化,有利于局部更为精确的寻优从而在整体上达到更高的精度。


\subsection{与其他空间非合作目标6D姿态估计方法的对比}
\label{sec:RANSAC-TRO-SQPnP:OthersCompare}
考虑到大量相关研究在SPEED/SPEED+数据集上进行了测试,因此这里的精度测试也在其进行。为了验证从关键点检测到最终 PnP 解算的完整流程在 6D 姿态估计上的有效性,进一步将本文方法与其他航天器姿态估计方法对比。其中,一些工作直接回归 6D 姿态或者使用关键点回归加上PnP解算。有的方法为了得到更高的精度采用热图的关键点预测方式;而本文的基于全局和局部注意力优化的改进YOLOv8-pose关键点检测网络直接回归关键点,并结合RANSAC-TRO SQPnP算法对关键点进行解算,从而在更小的计算复杂上,达到了相对不错的精度。


\subsection{SPEED数据集上的对比}
如表\ref{tab:SPEED_Comparison}所示,本文所提出的方法的精度尽管不如参数量更大的某些模型,但是本文的参数量最小,并且还优于部分参数量较大的模型。这使得其在部署到资源受限的空间平台上成为可能。
\begin{table}[htbp]
	\centering
	\caption{在SPEED合成数据集上6D航天器姿态估计方法的性能比较}
	\label{tab:SPEED_Comparison}
	\begin{tabular}{lccc}
		\toprule
		方法 & $err_{\text{T}}$ & $err_{\text{ort}}^{\circ}$ & 参数量(M) \\
		\midrule
		Hu et al.\citep{gerard2019segmentation} & 0.0730 & 0.9100 & $\sim$59.1 \\
		Lotti et al.\citep{lotti2022investigating} & \textbf{0.0340} & \textbf{0.5200} & 15.4 \\
		Wang et al.\citep{wang2022revisiting} & 0.0391 & 0.6638 & $\sim$47.8 \\
		Park et al.\citep{park2019towards} & 0.2090 & 2.6200 & 11.17 \\
		Piazza et al.\citep{piazza2021deep} & 0.1036 & 2.2400 & $\sim$36.1 \\
		Huan et al.\citep{huan2020pose} & 0.1823 & 2.8723 & $\sim$63.6 \\
		Ours & 0.1043 & 1.4076 & \textbf{4.3} \\
		\bottomrule
	\end{tabular}
\end{table}

\subsection{SPEED+数据集上的对比}
表\ref{tab:SPEEDplus_Comparison} 展示了在 SPEED+ 上各方法的详细对比。可见本文方法在旋转$score_{\text{ort}}^+$与位置综合评分 $score^+$ 上取得较好结果,本文的small方法能够以4.3M的参数量,10.4GFLOPs的计算复杂度达到旋转误差small级别的模型也能超越部分参数量和FLOPs较大的模型;考虑到SPEED+数据集规模是SPEED数据的4倍,因此本章在小规模参数模型的基础上,增加了medium级别与large级别的参数量模型。随着模型规模增大,精度进一步提升,对于large级别的模型其精度超过了与同类大型网络SPNv2($\phi$=6 GN),而参数量与FLOPs却均小于SPNv2($\phi$=6 GN)。这证明本文的模型在架构上也是具有优越性的。
\begin{table*}[htbp]
	\centering
	\caption{SPEED+合成数据集上6D航天器姿态估计方法的性能比较}
	\label{tab:SPEEDplus_Comparison}
	\setlength{\tabcolsep}{0.5mm}{
		\begin{tabular}{lcccccc}
			\toprule
			方法 & 参数量(M) & FLOPs(G) & $err_{\text{ort}}^{\circ}$ & $score_{\text{ort}}^+$ & $err_{\text{T}}$ & $score^+$ \\
			\midrule
			SPN\cite{sharma2019pose} & - & - & 7.7700 & - & 0.1600 & 0.1600 \\
			KRN\cite{park2019towards} & - & - & 3.6900 & - & 0.1400 & 0.0900 \\
			HigherHRNet \cite{higherhrnet} & $\sim 28.6\text{–}63.8$ & $\sim 74.9\text{–}154.3$ & 1.5100 & - & 0.0500 & 0.0400 \\
			P\'erez-Villar et al. \citep{perez2022spacecraft} & 190.1 & 487.8 & 1.4700 & 0.0256 & - & 0.0355 \\
			SPNv2($\phi$=3 GN) \cite{park2024robust} & 12.0 & 29.2 & 1.2240 & 0.0214 & 0.0560 & 0.0310 \\
			SPNv2($\phi$=6 GN) \cite{park2024robust} & 52.5 & 148.5 & 0.8850 & 0.0154 & \textbf{0.0310} & 0.0210 \\
			YOLOv8n-pose\cite{yolov8_ultralytics} & \textbf{3.2} & \textbf{9.1} & 2.1882 & 0.0382 & 0.0857 & 0.0546 \\
			YOLOv8s-pose\cite{yolov8_ultralytics} & 11.6 & 30.2 & 1.5764 & 0.0275 & 0.0524 & 0.0395 \\
			Ours small & 4.3 & 10.4 & 1.0760 & 0.0188 & 0.0459 & 0.0273 \\
			Ours medium & 18.7 & 53.5 & 0.8078 & 0.0141 & 0.0533 & 0.0231 \\
			Ours large & 47.2 & 145.5 & \textbf{0.6750} & \textbf{0.0118} & 0.0418 & \textbf{0.0189} \\
			\bottomrule
		\end{tabular}
	}
\end{table*}


\subsection{估计结果可视化}

本研究特别选择了SPEED+和SPEED数据集的示例来可视化本研究的最终结果。估计结果通过关键点和带有轴箭头的边界框来展示。绿色表示真实值,而红色表示预测结果。在合成数据集中,本研究首先可视化简单示例的预测结果。在图~\ref{fig:easy_demo}中,预测结果基本与真实值一致。在图~\ref{fig:hard_demo}中,展示了更困难的案例,其中一些表面难以辨别,遮挡更为明显。尽管如此,估计结果仍然与真实值非常接近。这证明了本研究方法的鲁棒性。

SPEED数据集还提供了真实的Tango卫星照片,但没有姿态标注。本研究在图~\ref{fig:real_demo}中也可视化了本研究在这些图像上的估计结果。尽管精度不如在合成数据中观察到的高,这是一个有待改进的点。

\begin{figure*}[!t] \centering \includegraphics[width=1\textwidth]{Img/easy_demo.pdf} \caption{SPEED+合成图像估计结果可视化的简单示例} \label{fig:easy_demo} \end{figure*}

\begin{figure*}[!t] \centering \includegraphics[width=1\textwidth]{Img/hard_demo.pdf} \caption{SPEED+合成图像估计结果可视化的困难示例} \label{fig:hard_demo} \end{figure*}

\begin{figure*}[htbp] \centering \includegraphics[width=1\textwidth]{Img/real_demo.pdf} \caption{在SPEED真实图像上的6D估计结果的实际示例} \label{fig:real_demo} \end{figure*}


\section{本章小结}
\label{sec:RANSAC-TRO-SQPnP:summary}
本章围绕通过关键点检测与基于关键点重投影误差优化的PnP算法相结合来求解非合作目标的6D姿态展开,首先在标准针孔模型和畸变模型的基础上推导了相机对3D关键点的投影关系;然后提出了结合正交约束的SQPnP算法,将旋转矩阵的求解视为带约束的非线性二次规划问题,通过序列二次规划方法迭代优化得到精确的旋转矩阵,并在此基础上线性求解平移向量;针对外点干扰,为了使估计对异常关键点更具鲁棒性,引入RANSAC随机采样与一致性检测以剔除外点;进而在获得精初值后,采用信赖域优化(TRO)对旋转和平移在局部区域内进一步非线性迭代精细求解,从而有效缓解旋转-平移之间的耦合影响;最后通过在SPEED与SPEED+等数据集上的实验,详细对比了RANSAC-TRO SQPnP与其他PnP算法及典型航天器姿态估计方法的精度与模型复杂度,结果表明RANSAC-TRO SQPnP在处理关键点噪声、剔除外点以及局部优化收敛性等方面均具有较高的鲁棒性和优异的精度表现。
