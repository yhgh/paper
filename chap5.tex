\chapter{基于关键点观测的空间自由翻滚非合作目标6D姿态滤波方法}
\label{chap:SpaceFreeMotionEKF}
\section{引言}
在上一章中,在关键点检测模型检测出的关键点像素坐标的基础上,利用RANSAC-TRO SQPnP解算出了空间非合作目标6D姿态。该方法的综合精度较高。然而,当目标较为“极端”的姿态时,其检测容易出现较大的误差。为了解决这一问题,本章将考虑空间非合作目标运动模型。在轨飞行中,考虑最为复杂的自由翻滚情形,非合作目标航天器往往经历较长时间且带有显著转动与平动耦合的自由运动。结合这个自由运动的模型可以进行卡尔曼滤波,从而充分利用时序信息,修正误差较大甚至是错误的估计。

\section{空间非合作目标自由运动模型假设}
在空间在轨服务及空间目标交会过程中,若空间非合作目标的推进系统已失效,当追踪者航天器接近时,可将其视为同一惯性系下,目标近似视为自由运动刚体。在全局惯性坐标系下,目标的平动与转动方程可分别表示为

\begin{equation}
	m\,\ddot{\mathbf{r}} = \mathbf{F}_\mathrm{ext},
\end{equation}

\begin{equation}
	\mathbf{I}\,\dot{\boldsymbol{\omega}}
	\;+\;
	\boldsymbol{\omega} \times \bigl(\mathbf{I}\,\boldsymbol{\omega}\bigr)
	= \mathbf{M}_\mathrm{ext},
\end{equation}

其中 \(m\) 是目标的总质量,\(\mathbf{r}\) 为目标质心在惯性系中的位置向量,\(\boldsymbol{\omega}\) 为目标的角速度向量,\(\mathbf{I}\) 是目标绕质心的转动惯量矩阵;\(\mathbf{F}_\mathrm{ext}\) 和 \(\mathbf{M}_\mathrm{ext}\) 分别为合外力和合外力矩。

由于目标在轨飞行时与追踪者航天器所受的重力加速度近似一致,二者在相对运动的局部参考系(如相机坐标系)中可将主要的地球引力视为“等效抵消”。如果 \(\mathbf{F}_\mathrm{ext} \approx \mathbf{0}\),则有

\begin{equation}
	\ddot{\mathbf{r}} \approx \mathbf{0},
\end{equation}

意味着目标质心做近似匀速或静止的平动;若同时 \(\mathbf{M}_\mathrm{ext} \approx \mathbf{0}\),则欧拉方程可近似写成

\begin{equation}
	\mathbf{I}\,\dot{\boldsymbol{\omega}}
	\;+\;
	\boldsymbol{\omega} \times \bigl(\mathbf{I}\,\boldsymbol{\omega}\bigr)
	\approx \mathbf{0}.
\end{equation}

在此条件下,目标的角动量近似保持不变,使其呈现出自由翻滚或不规则翻滚的运动形态。

\section{空间运动目标数据集的生成}
本节针对Starlink航天器在无外力与无外力矩条件下的自由运动过程进行仿真,并进一步利用BlenderProc对获得的时序位姿标注进行渲染,生成“空间运动目标”数据集。与前述仅包含静态位姿的数据集相比,本数据集更贴近真实在轨交会与捕获任务中非合作目标的运动特征,能够有效检验6D位姿估计算法在动态场景下的稳健性和泛化能力。

在空间非合作目标的近进捕获场景中,目标通常处于失控状态,即质心不受外力且无外力矩作用,容易发生自由翻滚;与此同时,追踪者航天器接近目标时,其相对轨迹可近似视为匀速靠近。因此,在同一惯性系下,目标的运动可被建模为“匀速平动 + 自由旋转”的刚体运动。

\subsection{Starlink几何模型与转动惯量的计算}
为了进行刚体动力学仿真,首先需要获取目标航天器的几何结构和质量属性。本文选取Starlink航天器的三维网格模型(通常以 \texttt{.ply} 格式存储),该模型包含离散顶点集合 \(\{\mathbf{p}_i\}_{i=1}^{N}\)。对模型进行预处理与坐标对齐后,可将其转换为稠密点云,并计算质心和惯量特性。具体步骤如下:

首先,将网格顶点在三维空间中的坐标求平均,得到质心
\begin{equation}
	\mathbf{c} = \frac{1}{N} \sum_{i=1}^{N} \mathbf{p}_i.
\end{equation}
随后,为简化转动惯量的表达,将所有顶点平移至质心坐标系,即 
\(\mathbf{p}_i^{*} = \mathbf{p}_i - \mathbf{c}\)。
必要时,还可对网格作小角度旋转,使其主体形态处于更便于后续分析的基准方位。

接着,在质心坐标系下计算离散情形的几何转动惯量张量
\begin{equation}
	I =
	\sum_{i=1}^{N}
	\begin{bmatrix}
		y_i^2 + z_i^2 & -x_i y_i      & -x_i z_i \\
		-x_i y_i      & x_i^2 + z_i^2 & -y_i z_i \\
		-x_i z_i      & -y_i z_i      & x_i^2 + y_i^2
	\end{bmatrix},
\end{equation}
其中 \(\mathbf{p}_i^{*} = (x_i,\,y_i,\,z_i)\) 表示第 \(i\) 个顶点在质心坐标系下的坐标。若航天器的实际质量为 \(m\),则通过以下比例系数对转动惯量张量进行缩放:
\begin{equation}
	\widetilde{I} = \alpha \, I, 
	\quad 
	\alpha = \frac{m}{N},
\end{equation}
从而得到更接近真实尺度的刚体转动惯量张量 \(\widetilde{I}\)。最终,将质心坐标 \(\mathbf{c}\) 和转动惯量张量 \(\widetilde{I}\) 保存,用于后续运动方程的数值积分。

\subsection{随机初始参数生成与覆盖率检查}
在进行自由运动仿真之前,需要为目标航天器指定初始位置 \(\mathbf{r}(0)\)、末端位置 \(\mathbf{r}(T)\)、平动速度 \(v\) 以及初始角速度 \(\boldsymbol{\omega}_{\mathrm{body}}(0)\) 等参数。为了确保数据集多样性并保证目标在大多数姿态下仍在相机视野内,本文设计了随机采样与覆盖率检查策略。

首先,将航天器的质心放置在合理的深度区间 \(\bigl(Z_{\min},\,Z_{\max}\bigr)\) 内,并限定初末位置在水平方向上的分布范围。限制初末位置之间的直线距离
\begin{equation}
d = \|\mathbf{r}(T) - \mathbf{r}(0)\|
\end{equation}
在 \(\bigl[d_{\min},\,d_{\max}\bigr]\) 范围内,以确保运动覆盖的充分性同时避免目标跑出视野。

然后,在给定的速度区间 \(\bigl(v_{\min},\,v_{\max}\bigr)\) 内随机采样平动速度 \(v\),以确定运动总时长 \(T\)。角速度方向则通过随机球面分布采样,并赋予 \(0.1 \sim 2.0\,\mathrm{rad/s}\) 的随机幅度,以模拟目标失控翻滚状态。为了排除目标越出视野的情况,对每组候选参数进行覆盖率检查:针对该组参数再随机若干次(如50次)姿态方向,将航天器网格点云与关键点投影到相机平面;若在绝大多数(如90\%以上)情况下投影点均落在图像内,则该组参数通过检查,否则丢弃。

通过上述步骤,可筛选出多组“合格”的初末位置、平动速度与初始角速度组合。


\subsection{无外力与无外力矩下的运动学与动力学模型}

在空间近似真空条件下,若目标航天器处于失控状态,则其质心作匀速直线运动,角动量保持守恒。设质心位置为 \(\mathbf{r}(t)\),质量为 \(m\),速度为 \(\mathbf{v}(t) = \dot{\mathbf{r}}(t)\)。在无外力作用下,满足

\begin{equation}
	m \,\dot{\mathbf{v}}(t) = 0, 
	\quad 
	\dot{\mathbf{r}}(t) = \mathbf{v}(t).
\end{equation}

因此,目标质心仅进行匀速直线运动或保持静止。若给定初始位置 \(\mathbf{r}(0)\) 与末端位置 \(\mathbf{r}(T)\),并设定平动速度 \(\mathbf{v}\),则质心位置更新可离散表示为

\begin{equation}
	\mathbf{r}(t_{k+1}) = \mathbf{r}(t_k) + \mathbf{v}\,\Delta t,
\end{equation}

其中 \(\Delta t\) 为时间步长,通常与渲染帧率相匹配,以生成连续序列图像。

对于旋转部分,设目标在本体坐标系下的角速度为 \(\boldsymbol{\omega}(t)\),转动惯量张量为 \(\widetilde{I}\)。在无外力矩条件下,刚体的角动量守恒方程为

\begin{equation}
	\widetilde{I}\,\dot{\boldsymbol{\omega}}(t) \;+\; \boldsymbol{\omega}(t) \,\times\, \bigl(\widetilde{I}\,\boldsymbol{\omega}(t)\bigr) \;=\; 0.
\end{equation}

该非线性方程的数值解在很多情形下都具有显著的难度,需要选取稳定且精度高的方法来对 \(\boldsymbol{\omega}(t)\) 随时间的演化进行积分。随后,再由旋转矩阵 \(R(t)\) 或四元数 \(q(t)\) 表征整体姿态的变化,满足

\begin{equation}
	\dot{R}(t) = R(t)\,[\,\boldsymbol{\omega}(t)\,]_{\times},
\end{equation}

其中 \([\boldsymbol{\omega}]_{\times}\) 为 \(\boldsymbol{\omega}\) 的反对称矩阵。

\subsubsection{改进的自适应Runge-Kutta积分方法}

为了解决刚体自由旋转方程中“角速度 \(\boldsymbol{\omega}\)”和“惯性耦合”所带来的数值困难,避免由于时间步长无法避免的离散化导致动量或能量出现漂移,本文在实现上采用了改进的自适应Runge-Kutta方法。传统的四阶Runge-Kutta(RK4)法通常可用于刚体动力学的数值积分,但对于刚体旋转方程

\begin{equation}
	\dot{\boldsymbol{\omega}}(t) 
	= \widetilde{I}^{-1}\Bigl[\,-\,\boldsymbol{\omega}(t) \times \bigl(\widetilde{I}\,\boldsymbol{\omega}(t)\bigr)\Bigr],
\end{equation}

若直接应用显式的RK4且时间步长过大,很容易出现数值爆炸或在长时间模拟中产生显著累积误差。

\subsubsection{改进的自适应Runge-Kutta积分方法}

为了解决刚体自由旋转方程中“角速度 \(\boldsymbol{\omega}\)”和“惯性耦合”所带来的数值困难,避免由于时间步长选择不当而导致动量或能量出现漂移,本文在实现上采用了改进的自适应Runge-Kutta方法。传统的四阶Runge-Kutta(RK4)法通常可用于刚体动力学的数值积分,但对于刚体旋转方程
\begin{equation}
	\dot{\boldsymbol{\omega}}(t) 
	= \widetilde{I}^{-1}\Bigl[-\,\boldsymbol{\omega}(t) \times \bigl(\widetilde{I}\,\boldsymbol{\omega}(t)\bigr)\Bigr],
\end{equation}
若直接应用显式的RK4且时间步长过大,很容易出现数值爆炸或在长时间模拟中产生显著累积误差。

改进的自适应Runge-Kutta积分方法在每一步计算中综合运用了误差估计、步长调节、Newton-Raphson校正以及正交化修正等策略。具体而言,首先通过对同一步长下不同阶次或两个步长大小的计算结果进行比对,来估计当前步的积分误差大小;若误差超过给定阈值,则缩小步长并重复本步运算,而若误差远小于阈值,则可适当放宽步长以提高计算效率。接下来,对刚体旋转中的角速度更新方程引入Newton-Raphson迭代,以显式处理 \(\boldsymbol{\omega} \times \widetilde{I}\,\boldsymbol{\omega}\) 这一非线性耦合项,降低显式积分带来的数值不稳定风险;最后,为防止数值累积误差破坏旋转矩阵的正交性,在每一次更新完 \(R(t)\) 后都利用奇异值分解(SVD)或QR分解进行正交化修正,从而保持姿态描述的准确性与稳定性。



\subsection{离散时间步仿真与位姿标注}

综合上述平动与旋转方程,本文采用离散时间步的方式对目标航天器进行刚体运动仿真,并生成对应的位姿标注。具体流程
如图\ref{fig:pos_gen_process}所示,首先进行初始化。随后进行RK4预测:根据上一时刻的 \(\boldsymbol{\omega}(t_k)\),采用Runge-Kutta方法在步长 \(\Delta t\) 下进行显式积分,得到 \(\boldsymbol{\omega}_{\mathrm{pred}}(t_{k+1})\)。接下来进行误差测度:将 \(\Delta t\) 在不改变初值的前提下拆分为两步 \(\Delta t/2\),分别进行两次RK积分,得到 \(\boldsymbol{\omega}_{\mathrm{two-step}}(t_{k+1})\),并比较 \(\|\boldsymbol{\omega}_{\mathrm{pred}} - \boldsymbol{\omega}_{\mathrm{two-step}}\|\) 来估计误差。若该误差大于阈值 \(\varepsilon\),则将 \(\Delta t \leftarrow \Delta t / 2\) 并重复本步;若误差远低于 \(\varepsilon\),则可在允许范围内增大步长。

在此基础上,为进一步降低显式积分带来的数值不稳定风险,需要对预测结果进行Newton-Raphson校正。具体做法是,在预测结果 \(\boldsymbol{\omega}_{\mathrm{pred}}(t_{k+1})\) 的基础上,构建带有耦合项的残差方程,并对 \(\boldsymbol{\omega}(t_{k+1})\) 进行Newton迭代求解,使之满足守恒方程要求。之后,根据更新后的角速度 \(\boldsymbol{\omega}(t_{k+1})\),对旋转矩阵 \(R(t_{k+1})\) 进行指数映射(或其他方法)以完成姿态更新。为保证在数值迭代过程中不会破坏旋转矩阵的正交性,还需对 \(R(t_{k+1})\) 做SVD或QR分解并修正,以保持正交性。若尚未达到终止条件,则返回并继续上述过程,直至完成所有时间步的位姿计算。



通过引入自适应步长、Newton-Raphson校正以及正交性维持等措施,可显著降低刚体自由旋转数值积分过程中的累积误差,从而保证动量、能量等守恒量的误差被控制在可接受范围内。

最终,可获取目标航天器在自由旋转与匀速平动条件下的时序位姿标注,用于后续BlenderProc的高质量渲染。

\subsection{仿真有效性验证}
\label{sec:simulation_validity_measure}

刚体自由旋转仿真往往对数值方法较为敏感:一方面,初始姿态与角速度的微小变化可能在长时间后累积为明显偏差;另一方面,由于无外力矩环境下应保持角动量恒定,数值积分若稍有不慎便会出现能量或动量漂移。为实现高精度数值仿真,需要在积分方法的稳定性、旋转参数化方式以及时间步长的选取等方面综合权衡。本研究在实现中参考了上述改进自适应Runge-Kutta积分方法,并在每一步迭代中进行舍入误差估计与Newton-Raphson修正,从而有效抑制累积误差对角动量守恒的影响,保证了长期仿真结果的可靠性。
实验表明该仿真的最大角动量误差仅为1.04e-4。精度足够理想。


\subsection{基于BlenderProc的动态渲染}

完成运动学仿真与位姿标注后,可将Starlink网格模型与上述时序平移、旋转信息一同加载至BlenderProc。与静态渲染类似,只是在每帧渲染前将目标位置设为$r(t_k)$,姿态设为$R(t_k)$。具体步骤如下:

\begin{enumerate}
	\item 载入Starlink模型及其材质纹理;
	\item 设置相机内外参、光照和环境等渲染参数;
	\item 逐帧将模型平移并旋转到$r(t_k)$与$R(t_k)$,调用BlenderProc渲染输出图像;
	\item 将生成图像与时刻$t_k$对应的位姿记录相匹配,保存为连续序列数据集。
\end{enumerate}

最终可得到包含多帧时序图像及相应6D位姿标注的“空间运动目标”数据集。如图\ref{fig:starlink_render_dyn}所示。该数据集更贴近实际在轨情况下的动态特征,为6D位姿估计算法在空间非合作目标场景下的研究与验证提供了更测试依据。
\begin{figure}[htbp]
	\centering
	\includegraphics[width=0.5\textwidth]{Img/starlink_motion.png}
	\caption{空间自由翻滚目标位姿图像}
	\label{fig:starlink_render_dyn}
	\vspace{-3ex}
\end{figure}

\begin{figure}[htbp]
	\centering
	\includegraphics[width=0.5\textwidth]{Img/pos_gen_process.png}
	\caption{位姿标注计算的流程图}
	\label{fig:pos_gen_process}
	\vspace{-3ex}
\end{figure}

\section{扩展卡尔曼滤波}
\label{sec:ekf_6dof_derivation}

\subsection{状态定义及系统模型}

本文以刚体在空间中的平动和转动作为研究对象,采用如下 13 维状态向量:
\begin{equation}
	\label{eq:xstate}
	\mathbf{x}_k =
	\begin{bmatrix}
		\mathbf{r}_k \\[6pt]
		\mathbf{q}_k \\[6pt]
		\mathbf{v}_k \\[6pt]
		\boldsymbol{\omega}_k
	\end{bmatrix}
	=
	\begin{bmatrix}
		\mathbf{r}_{k} \in \mathbb{R}^3 \\
		\mathbf{q}_{k} \in \mathbb{R}^4 \\
		\mathbf{v}_{k} \in \mathbb{R}^3 \\
		\boldsymbol{\omega}_{k} \in \mathbb{R}^3
	\end{bmatrix},
\end{equation}
其中
\begin{itemize}
	\item \(\mathbf{r}_k = \bigl[x,y,z\bigr]^T\) 表示刚体质心在世界坐标系中的位置;
	\item \(\mathbf{q}_k = \bigl[q_w,q_x,q_y,q_z\bigr]^T\) 表示刚体姿态的四元数(采用 \(wxyz\) 顺序);
	\item \(\mathbf{v}_k = \bigl[v_x,v_y,v_z\bigr]^T\) 表示刚体平动速度;
	\item \(\boldsymbol{\omega}_k = \bigl[\omega_x,\omega_y,\omega_z\bigr]^T\) 表示刚体的角速度。
\end{itemize}

考虑到空间非合作目标自由翻滚运动的复杂性,尤其是非恒定的角速度,采用“恒定速度–恒定角速度”的简化模型,即在一个采样周期 \(\Delta t\) 内,速度和角速度视为不变,位置和姿态作相应的线性/指数积分。其离散过程模型可写为:
\begin{equation}
	\label{eq:r_update}
	\mathbf{r}_{k+1} = \mathbf{r}_k + \mathbf{v}_k \,\Delta t,
\end{equation}
\begin{equation}
	\label{eq:q_update}
	\mathbf{q}_{k+1} = \mathbf{q}_k \,\otimes\, \exp\bigl(\boldsymbol{\omega}_k\,\Delta t\bigr),
\end{equation}
\begin{equation}
	\label{eq:v_update}
	\mathbf{v}_{k+1} = \mathbf{v}_k,
\end{equation}
\begin{equation}
	\label{eq:omega_update}
	\boldsymbol{\omega}_{k+1} = \boldsymbol{\omega}_k,
\end{equation}
其中 \(\otimes\) 表示四元数乘法,\(\exp(\cdot)\) 表示将角速度矢量转化为等效的“旋转向量”后再转换为四元数的指数映射,并对结果做归一化以抑制数值误差。

为在 EKF 中使用,需要将非线性的过程方程
\begin{equation}
	\mathbf{x}_{k+1} = f\bigl(\mathbf{x}_k,\,\mathbf{w}_k\bigr)
\end{equation}
在线性化(一阶近似)。由于本文示例中将过程噪声 \(\mathbf{w}_k\) 直接合并到离散系统中的 \(Q_k\) 矩阵,线性化时主要对 \(f(\cdot)\) 关于状态 \(\mathbf{x}_k\) 做一阶展开,得到状态转移雅可比矩阵 \(F_k\):
\begin{equation}
	\label{eq:Fk_jac}
	F_k = \frac{\partial f}{\partial \mathbf{x}}\Bigg|_{\hat{\mathbf{x}}_{k|k}}
	\approx
	\begin{bmatrix}
		I_{3\times 3} & 0 & I_{3\times 3}\Delta t & 0 \\
		0 & A_{q}      & 0 & B_{\omega} \\
		0 & 0          & I_{3\times 3} & 0 \\
		0 & 0          & 0 & I_{3\times 3}
	\end{bmatrix},
\end{equation}
其中\(I_{3\times 3}\) 表示 \(3\times 3\) 单位矩阵;上式中 \(A_{q}\) 和 \(B_{\omega}\) 则与四元数的更新过程(\ref{eq:q_update})的微分相关,可简化处理或数值求解。对于小角速度下,\(A_{q}\) 通常可近似为 \(I_{4\times 4}\),但更加精确的做法是基于 \(\Delta t\) 及 \(\boldsymbol{\omega}_k\) 的方向计算局部线性化;由于此示例中 \(\boldsymbol{\omega}_k\) 在一帧内不变,\(B_{\omega}\) 可视情况简化处理,也可认为在小量 \(\Delta t\) 范围内为常数的局部近似雅可比。

在具体实现中,为了算法更为简洁,很多场合只对平动部分的状态转移做精确雅可比(即位置对速度的偏导为 \(\Delta t\)),而姿态部分的微分直接在预测步骤中用数值方法(Rodrigues/库函数)更新并保持较小的线性化近似。

过程噪声协方差记为
\begin{equation}
	Q_k = \mathrm{diag}(\sigma_{r}^2, \sigma_{q}^2, \sigma_{v}^2, \sigma_{\omega}^2),
\end{equation}
实际可根据系统经验或调参对位置、姿态、速度、角速度部分分别赋值(代码中的 \texttt{q\_scale} 即在此基础上做整体缩放)。

\subsection{观测模型}

在本例中,测量由外部 PnP 或者其他视觉算法给出刚体的“观测位姿”\(\bigl(\mathbf{r}_{\mathrm{meas}}, \mathbf{q}_{\mathrm{meas}}\bigr)\),即测量向量 \(\mathbf{z}_k \in \mathbb{R}^7\):
\begin{equation}
	\mathbf{z}_k =
	\begin{bmatrix}
		\mathbf{r}_{\text{meas},k} \\[6pt]
		\mathbf{q}_{\text{meas},k}
	\end{bmatrix}
	+
	\boldsymbol{\nu}_k,
\end{equation}
其中 \(\boldsymbol{\nu}_k\) 为测量噪声,满足协方差 \(R_k\)。此时可将测量方程形式化为
\begin{equation}
	\mathbf{z}_k = h\bigl(\mathbf{x}_k\bigr) + \boldsymbol{\nu}_k,
\end{equation}
而
\begin{equation}
	h\bigl(\mathbf{x}_k\bigr) =
	\begin{bmatrix}
		\mathbf{r}_k \\
		\mathbf{q}_k
	\end{bmatrix}.
\end{equation}
由于这里直接测得了状态向量的一部分(位置和姿态),\(h(\cdot)\) 对 \(\mathbf{x}_k\) 的偏导即在相应分块为单位矩阵、其余为零:
\begin{equation}
	\label{eq:Hk_jac}
	H_k = \frac{\partial h}{\partial \mathbf{x}}\Bigg|_{\hat{\mathbf{x}}_{k|k-1}}
	=
	\begin{bmatrix}
		I_{3\times 3} & 0 & 0 & 0 \\
		0 & I_{4\times 4} & 0 & 0
	\end{bmatrix}.
\end{equation}

\subsection{EKF递推过程}

\noindent
EKF的预测步骤为预测步骤为:

\begin{equation}
	\hat{\mathbf{x}}_{k|k-1} = f\bigl(\hat{\mathbf{x}}_{k-1|k-1}\bigr),
\end{equation}
\begin{equation}
	P_{k|k-1} = F_k \, P_{k-1|k-1} \, F_k^T + Q_k,
\end{equation}
其中 \(F_k\) 为线性化雅可比,\(\hat{\mathbf{x}}_{k|k-1}\) 和 \(P_{k|k-1}\) 分别是先验状态估计及先验协方差。

\noindent
EKF的更新(Update)步骤为:

\begin{equation}
	\hat{\mathbf{z}}_{k|k-1} = h\bigl(\hat{\mathbf{x}}_{k|k-1}\bigr),
\end{equation}
\begin{equation}
	\label{eq:Hk}
	H_k = \frac{\partial h}{\partial \mathbf{x}}\Bigg|_{\hat{\mathbf{x}}_{k|k-1}},
\end{equation}
\begin{equation}
	S_k = H_k\, P_{k|k-1}\, H_k^T + R_k,
\end{equation}
\begin{equation}
	K_k = P_{k|k-1} \, H_k^T \, S_k^{-1},
\end{equation}
\begin{equation}
	\hat{\mathbf{x}}_{k|k} = \hat{\mathbf{x}}_{k|k-1} + K_k \bigl(\mathbf{z}_k - \hat{\mathbf{z}}_{k|k-1}\bigr),
\end{equation}
\begin{equation}
	\label{eq:P_update}
	P_{k|k} = \bigl(I - K_k\,H_k\bigr)\,P_{k|k-1}.
\end{equation}

在本代码实现中,更新时的残差为
\begin{equation}
	\mathbf{z}_k - h\bigl(\hat{\mathbf{x}}_{k|k-1}\bigr)
	=
	\begin{bmatrix}
		\mathbf{r}_{\text{meas},k} - \hat{\mathbf{r}}_{k|k-1}\\
		\mathbf{q}_{\text{meas},k} - \hat{\mathbf{q}}_{k|k-1}
	\end{bmatrix},
\end{equation}


\subsection{单纯位姿量测EKF的局限性}
\label{sec:limitations_pose_ekf}

在上述做法中,每帧图像依赖PnP先行估计出位姿$(\mathbf{r}_{\mathrm{meas}}, \mathbf{q}_{\mathrm{meas}})$,再将其当作EKF的观测量,这种流程在实现上虽然简单,却存在多方面缺陷:首先,PnP算法本身通常依赖关键点匹配质量和RANSAC剔除外点的稳定性,一旦特征退化或遮挡严重,PnP可能失效或者解算结果噪声巨大,致使随后进入EKF的测量量失真甚至中断;其次,这种“二次线性化”会带来额外误差,即PnP已对投影模型做了一次非线性优化或近似,EKF在更新时又会线性化$(\mathbf{r}_{\mathrm{meas}}, \mathbf{q}_{\mathrm{meas}})$与状态的残差关系,等于在同一组2D-3D观测上重复独立地做两次线性化,从而难以获得全局一致的最优估计;此外,由于每一帧滤波器获得的观测量仅限于单个位姿七维向量,很多潜在的信息(例如部分可用的关键点约束)无法被充分利用,一旦PnP在某些帧并未成功返回可靠解,滤波器就无法在该帧完成有效更新;最终,当相机运动与场景变化导致视觉解算精度下滑时,单纯依赖PnP位姿的EKF往往无力提高系统的整体稳健性与精度。为克服上述不足,后文将进一步探讨直接将关键点的图像观测纳入EKF过程,通过在滤波器内部对2D-3D投影残差进行整体求解,从而减少对PnP阶段性解算的依赖,并在理论上实现对非线性误差的一体化处理。

\subsection{基于关键点观测的测量模型}
\label{sec:keypoints_in_ekf}

为克服上述局限,进一步将视觉关键点的像素坐标$(u_i, v_i)$直接纳入EKF的测量方程,以替代对PnP位姿的二次线性化与合并。其核心思路在于:令滤波器的观测矢量包含多对$\{(\mathbf{X}_i, \mathbf{z}_i)\}$,其中$\mathbf{X}_i\in\mathbb{R}^3$为物体坐标系下已知的三维特征点,$\mathbf{z}_i=[\,u_i,\,v_i\,]^T$为相应图像平面测量。测量模型可写为
\begin{equation}
	\mathbf{z}_i = \Pi\bigl(\mathbf{R}(\mathbf{q}),\,\mathbf{t},\,\mathbf{X}_i\bigr) + \boldsymbol{\nu},
\end{equation}
其中$\mathbf{R}(\mathbf{q})$由四元数$\mathbf{q}$转换为旋转矩阵,$\mathbf{t}$为平移向量,$\Pi(\cdot)$表示内参与畸变校正的投影函数,$\boldsymbol{\nu}$为测量噪声。通过数值差分或解析方法计算该非线性投影对状态$x$的雅可比,可直接在EKF更新时最小化投影残差$\Delta \mathbf{z} = \mathbf{z}_\text{meas}-\Pi(\dots)$,从而在滤波器内部一体化地处理“由姿态到图像关键点”的关系,减少对外部PnP解算的依赖。这样的做法除了能更充分地利用部分可用的关键点观测,还可在遇到特征退化时,通过对多帧多点的冗余约束提高估计鲁棒性与精度;并且与前文仅有七维位姿观测相比,这种多关键点输入的EKF测量维度更高,能够显著提升滤波器的可观测性。后续将给出具体的数值实现与实验结果,以说明在多特征点场景下,该方法能有效地改进目标6DoF运动估计的稳定性与精度。





\section{基于SE(3)的误差状态扩展卡尔曼滤波}
在六自由度刚体运动中,系统状态一般位于李群SE(3)上。若直接将姿态用欧拉角或旋转矩阵作线性加法运算,往往易导致数值不稳定或精度丢失。基于SE(3)的误差状态EKF利用李群上的小扰动形式更新姿态,可更好地保持估计的数值稳定性和精度。

设名义状态
\begin{equation}
	x_{\text{nominal}} = 
	\begin{bmatrix}
		r \\
		v \\
		q \\
		\omega
	\end{bmatrix},
\end{equation}
其中$r \in \mathbb{R}^3$为位置,$v \in \mathbb{R}^3$为线速度,$q$为四元数姿态,$\omega \in \mathbb{R}^3$为角速度。定义误差状态
\begin{equation}
	\delta x = 
	\begin{bmatrix}
		\delta r \\
		\delta v \\
		\delta \phi \\
		\delta \omega
	\end{bmatrix},
\end{equation}
其中$\delta \phi \in \mathbb{R}^3$为姿态上的小扰动,通过李群上的指数映射修正姿态。预测阶段通常假设常速度和常角速度:
\begin{equation}
	r_{k+1} = r_k + v_k \,\Delta t,
\end{equation}
\begin{equation}
	q_{k+1} = \exp\!\Bigl(\tfrac{\omega_k \,\Delta t}{2}\Bigr)\otimes q_k,
\end{equation}
\begin{equation}
	v_{k+1} = v_k,
\end{equation}
\begin{equation}
	\omega_{k+1} = \omega_k.
\end{equation}
其中$\exp(\cdot)$表示$\mathrm{so}(3)$到$\mathrm{SO}(3)$的指数映射,$\otimes$为四元数乘法。量测更新阶段结合视觉关键点的投影,运用数值差分或解析方法获得观测方程对误差状态的雅可比矩阵,再由EKF更新方程进行状态修正。



\subsection{算法流程简述}
基于SE(3)的误差状态EKF可概括如下:先根据目标初始状态建立名义状态和协方差;在预测阶段假设常速度与常角速度并进行外推;在量测更新阶段通过关键点投影与名义状态的差构建残差,再以数值差分或解析方法获取观测对误差状态的雅可比矩阵,最后借助EKF的更新公式修正名义状态并更新协方差。由于对旋转部分采用李群上的增量形式,可减少直接加法带来的数值误差累积并提高姿态估计精度。

\section{实验分析}
为验证本章所述方法,分别设定三种滤波方案并进行对比:
\begin{itemize}
	\item \textbf{EKF}:不使用视觉关键点的EKF(仅依赖惯性预测);
	\item \textbf{EKF + 关键点观测}:在预测基础上纳入关键点量测更新的常规EKF;
	\item \textbf{EKF + 关键点观测 + SE(3)}:在上述基础上进一步采用基于李群SE(3)的误差状态表达。
\end{itemize}

实验通过模拟环境产生目标的自由翻滚运动并叠加一定噪声,以仿真相机模型生成关键点投影。对预测与测量噪声的协方差矩阵加以不同配置以模拟不同情形,并比较各方案在姿态估计误差随时间的变化情况。

\begin{figure}[htbp]
	\centering
	\includegraphics[width=1.0\textwidth]{Img/EKF_err.png}
	\caption{EKF滤波在角度误差与相对位置误差上的随帧数的变化}
	\label{fig:PoseErrorComparison_EKF}
\end{figure}

\begin{figure}[htbp]
	\centering
	\includegraphics[width=1.0\textwidth]{Img/EKF_kpt_err.png}
	\caption{EKF+关键点观测在角度误差与相对位置误差上随帧数的变化}
	\label{fig:PoseErrorComparison_KPT}
\end{figure}

\begin{figure}[htbp]
	\centering
	\includegraphics[width=1.0\textwidth]{Img/EKF_kpt_lie_err.png}
	\caption{EKF+关键点观测+SE(3)在角度误差与相对位置误差上随帧数的变化}
	\label{fig:PoseErrorComparison_LIE}
\end{figure}

如图\ref{fig:PoseErrorComparison_EKF}、\ref{fig:PoseErrorComparison_KPT} 以及\ref{fig:PoseErrorComparison_LIE}所示,三种方案在角度误差和位置误差上的时间序列曲线呈现以下规律:
仅依赖EKF与位姿态时序的方案(图\ref{fig:PoseErrorComparison_EKF})误差相对于滤波前没有明显改进;将视觉关键点纳入观测后(图\ref{fig:PoseErrorComparison_KPT}),角度与位置误差均显著收敛,脉冲幅度下降;再进一步结合基于SE(3)的误差状态方法(图\ref{fig:PoseErrorComparison_LIE}),则对旋转估计的稳健性和精度提升最为明显,误差曲线整体最为平稳。


为了量化对比三种方法在整个时域的平均误差,表\ref{tab:filter_compare}给出了角度误差与位置误差的数值统计。可以看出,EKF的角度误差为 $1.23^\circ$,位置误差为 $0.12\,\mathrm{m}$;EKF + 关键点观测角度误差降低到 $1.15^\circ$,位置误差降至 $0.10\,\mathrm{m}$;EKF + 关键点观测 + SE(3)方案进一步提升至$1.05^\circ$ 和 $0.08\,\mathrm{m}$,效果最佳。






从表\ref{tab:filter_compare}中可以看出,未滤波情形下的角度误差和相对位置误差分别为 \(1.4918^\circ\) 和 \(0.01341\,\mathrm{m}\)。采用EKF仅进行惯性预测的滤波后,这两项指标仅略微下降至 \(1.4911^\circ\) 与 \(0.01340\,\mathrm{m}\),说明在此仿真环境和噪声水平下,单纯依赖惯性信息并不能带来明显的效果提升。

当将关键点观测引入滤波过程后(即EKF + 关键点 方案),角度误差从 \(1.4918^\circ\) 下降至 \(1.2033^\circ\),位置误差则由 \(0.01341\,\mathrm{m}\) 显著降低到 \(0.00635\,\mathrm{m}\),相比未滤波时分别提升约 \(19\%\) 和 \(53\%\)。可见加入视觉关键点观测后,姿态与位置估计的稳健性和准确度均有明显改善。

在此基础上进一步采用\textbf{基于 SE(3) 的误差状态表达}(即EKF + 关键点 + SE(3)方案),角度误差降低至 \(1.1912^\circ\),位置误差进一步收敛至 \(0.00523\,\mathrm{m}\),较未滤波相比提升幅度分别约为 \(20\%\) 和 \(61\%\),比单纯使用关键点的方案更进一步。该结果说明:在纳入视觉关键点的基础上,通过对旋转群 \(SE(3)\) 上的误差状态精细化建模与更新,能显著增强滤波在非线性姿态估计场景中的数值稳定性与精度。
\begin{table}[htbp]
	\centering
	\caption{各方法在角度误差与相对位置误差上的对比}
	\label{tab:filter_compare}
	\begin{tabular}{lcc}
		\toprule
		\textbf{方法} & \textbf{角度误差($^\circ$)} & \textbf{相对位置误差} \\
		\midrule
		\textbf{未滤波} & 1.4918 & 0.01341 \\
		\textbf{EKF} & 1.4911 & 0.01340 \\
		\textbf{EKF + 关键点} & 1.2033 & 0.00635 \\
		\textbf{EKF + 关键点 + SE(3)} & 1.1912 & 0.00523 \\
		\bottomrule
	\end{tabular}
\end{table}
\vspace{5pt}

本章针对空间在轨服务及空间目标交会过程中非合作目标的自由运动情形,提出了结合刚体动力学模型和关键点观测的6D姿态EKF滤波方法。首先,在假设无外力与无外力矩条件下,对Starlink目标航天器进行刚体动力学仿真,并利用BlenderProc工具对其渲染生成了近似真实在轨动态场景下的“空间运动目标”数据集。随后,为克服PnP解算在极端姿态或特征退化条件下易出现失效的问题,本章将关键点像素坐标直接纳入EKF的量测方程,使滤波器可以在时序域内融合关键点投影与刚体状态,从而有效修正单帧估计的偏差。接着,针对传统欧拉角或四元数直接加法所带来的数值不稳定与精度损失问题,本章提出采用李群SE(3)上的误差状态表达方式,将姿态增量作为小扰动进行修正,显著提升了转动部分的稳定性与估计精度。最后,通过三种滤波方案(仅EKF预测、EKF+关键点观测、EKF+关键点+SE(3))的对比实验可以看出,引入关键点观测后,角度误差和位置误差均能大幅降低,进而在李群SE(3)上处理姿态增量能够在非线性姿态估计场景中获得更优的估计效果,从而为在复杂在轨时序环境中实现稳健且精确的空间目标6D位姿追踪奠定了方法基础。





