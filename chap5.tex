\chapter{基于关键点观测的空间自由翻滚非合作目标6D姿态滤波方法}
\label{chap:SpaceFreeMotionEKF}
\section{引言}
近年来,针对空间非合作目标的6D姿态估计研究已取得一定进展,但更多是局限于单帧静态图像的6D姿态估计。但在实际应用中,仍面临由于目标姿态复杂变化及测量噪声等因素导致的精度与稳定性不足问题。前文(第\ref{chap:attention_kpt}章与第\ref{chap:RANSAC-TRO-SQPnP}章)从6D姿态估计的两个阶段出发,提出了基于关键点检测与PnP优化的完整估计流程,并在静态图像数据(SPEED/SPEED+与两个自建的仿真的数据集)上验证了其有效性。然而,根据第~\ref{chap:RANSAC-TRO-SQPnP}章图\ref{fig:hard_demo},包括第~\ref{chap:attention_kpt}的消融实验的可视化图,也会存在较高的姿态估计误差,对要求稳定性和连续性的应用场景构成挑战。考虑到空间非合作目标在在轨失效时,常表现出自由翻滚不规则运动,若此时对目标实施6D姿态估计,则必须应对其剧烈且连续的转动和位移。尽管自由翻滚在直觉上给测量带来难度,但正是这类运动的时序信息为解决高误差估计提供了可能。基于此思想,若能对目标的运动学规律施加合理的先验假设,并通过卡尔曼滤波在时间序列上对估计结果进行平滑,则有望抑制孤立帧所带来的高误差。

卡尔曼滤波在航天器姿态估计中已有广泛应用,但很多目标是合作目标,能够主动进行交互或者有额外的其他类型传感器作为协助判断\cite{DQWX202212007,CHWZ202205009,hudoba2024distributed}。针对非合作目标,也有相关研究的研究\cite{s24061811,zhang2023pose,XDFJ20250225001},EKF作为一款经典的处理非线性的滤波器也被用于解决空间非合作目的6D姿态滤波问题,然而对空间非合作目标,在强非线性且复杂的自由翻滚运动仍存在局限。为此,本章在基于第\ref{chap:attention_kpt}章与第\ref{chap:RANSAC-TRO-SQPnP}章估计方案估计出的6D姿态的基础上,引入滤波过程,提出将像素关键点作为观测量融入EKF滤波过程;同时采用基于SE(3)李群的四元数状态更新策略,以在非线性旋转空间中获得更具鲁棒性的更新结果。然而前文的研究数据样本是空间非合作目标单帧静态图像的合集,没有能够构成动态连续帧的样本序列。因此本章首先基于空间目标自由翻滚运动的动力学特性,计算出相应的运动帧的6D姿态,然后借助BlenderProc渲染生成空间非合作目标自由翻滚运动的连续帧序列图像,对滤波算法进行验证和性能评估。相关实验结果将在关键点观测与SE(3)优化的联合加持下,滤波方法能够在多帧信息融合中抑制较高离群估计,提高姿态跟踪的整体精度与稳定性。



\section{空间非合作目标自由运动模型假设}
在空间在轨服务及空间目标交会过程中,若空间非合作目标的推进系统已失效,当追踪者航天器接近时,可将其视为同一惯性系下,目标近似视为自由运动刚体。在全局惯性坐标系下,目标的平动与转动方程可分别表示为

\begin{equation}
	m\,\ddot{\mathbf{r}} = \mathbf{F}_\mathrm{ext}
\end{equation}

\begin{equation}
	\label{eq:eluar_eq}
	\mathbf{I}\,\dot{\boldsymbol{\omega}}
	\;+\;
	\boldsymbol{\omega} \times \bigl(\mathbf{I}\,\boldsymbol{\omega}\bigr)
	= \mathbf{M}_\mathrm{ext}
\end{equation}

其中 \(m\) 是目标的总质量,\(\mathbf{r}\) 为目标质心在惯性系中的位置向量,\(\boldsymbol{\omega}\) 为目标的角速度向量,\(\mathbf{I}\) 是目标绕质心的转动惯量矩阵;\(\mathbf{F}_\mathrm{ext}\) 和 \(\mathbf{M}_\mathrm{ext}\) 分别为合外力和合外力矩。

由于目标在轨飞行时与追踪者航天器所受的重力加速度近似一致,二者在相对运动的局部参考系(如相机坐标系)中可将主要的地球引力视为“等效抵消”。如果 \(\mathbf{F}_\mathrm{ext} \approx \mathbf{0}\),则有

\begin{equation}
	\ddot{\mathbf{r}} \approx \mathbf{0}
\end{equation}

意味着目标质心做近似匀速或静止的平动;若同时 \(\mathbf{M}_\mathrm{ext} \approx \mathbf{0}\),则欧拉方程可近似写成

\begin{equation}
	\mathbf{I}\,\dot{\boldsymbol{\omega}}
	\;+\;
	\boldsymbol{\omega} \times \bigl(\mathbf{I}\,\boldsymbol{\omega}\bigr)
	\approx \mathbf{0}
\end{equation}

在此条件下,目标的角动量近似保持不变,使其呈现出自由翻滚的运动形态。





\section{空间非合作目标自由翻滚目标的运动模拟}
为了得到空间自由翻滚目标的连续运动帧的6D姿态标注,本节以Starlink模型为研究对象,并设计了如图~\ref{fig:motion_sim}所示的流程,介绍空间自由翻滚目标从初始化到最终输出的模拟过程。针对无外力、无外力矩作用下的刚体,分别对转动与平动进行数值积分,并在每个离散时间步进行更新。下面将按流程图~\ref{fig:motion_sim}顺序依次说明各运动模拟的各步骤。


\subsection{运动模拟的初始化设置}
如图~\ref{fig:motion_sim}所示,首先进行一个环节的初始化操作,首先进行初始化。在开始仿真前,需要设定如表~\ref{tab:simulation_parameters}所示的初始参数。
\begin{table}[htbp]
	\centering
	\caption{运动模拟的参数设置}
	\label{tab:simulation_parameters}
	\begin{tabular}{lc}
		\toprule
		\textbf{参数说明} & \textbf{参数形式} \\
		\midrule
		初始位置(m) & $(x_0, y_0, z_0)$ \\
		末端位置(m) & $(x_T, y_T, z_T)$ \\
		平动速度大小(m/s) & $v_0$ \\
		角速度(rad/s) & $(\omega_{x0}, \omega_{y0}, \omega_{z0})$ \\
		初始姿态四元数 & $(1, 0, 0, 0)$ \\
		转动惯量 & $\mathbf{I}$ \\
		模拟帧率(fps)& $30$ \\
		时间步长(s)& $0.033$\\
		\bottomrule
	\end{tabular}
\end{table}

初始位置、末端位置和平动速度大小,角速度采用随机生成的3组数据,如表~\ref{tab:inial_cond_motion}所示。初始姿态按照四元数(1,0,0,0),表示没有任何旋转。而平动速度的方向则可以通过初始位置到末端位置的方向确定。帧率设置为30帧,可推导出相邻两帧的时间为$\frac{1}{30}$s约为$0.033$s,这既是时间步长$\Delta t$的值。


根据欧拉方程~\ref{eq:eluar_eq},为了完成转动运动的计算,需要目标的转动惯量,而转动惯量与目标的质量分布有关系。这里按照Starlink点云的分来近似Starlink的质量分布。
\begin{equation}
	I =
	\sum_{i=1}^{N}
	\begin{bmatrix}
		y_i^2 + z_i^2 & -x_i y_i      & -x_i z_i \\
		-x_i y_i      & x_i^2 + z_i^2 & -y_i z_i \\
		-x_i z_i      & -y_i z_i      & x_i^2 + y_i^2
	\end{bmatrix},
\end{equation}
可根据其点云坐标计算转动惯量,其中$N$为点云的总点数,$(x_i, y_i, z_i)$为第$i$个点的坐标。但是为了使得转动惯量更接近Starlink的真实转动惯量,这里假设Starlink的质量为300kg,通过公式~\ref{eq:scale_interia}与~\ref{eq:scale_interia_factor}按照比例系数对转动惯量张量进行缩放:
\begin{equation}
	\label{eq:scale_interia}
	\widetilde{I} = \alpha \, I, 
\end{equation}
\begin{equation}
	\label{eq:scale_interia_factor}
	\alpha = \frac{m}{N},
\end{equation}
其中$m$为航天器质量,这样就可以得到更接近真实尺度的刚体转动惯量张量 $\widetilde{I}$。

计算得到的Starlink的转动惯量如下:
\begin{equation}
	\widetilde{I} = 
	\begin{bmatrix}
		665.220 & -9.326 & 13.295 \\
		-9.326 & 321.837 & 70.596 \\
		13.295 & 70.596 & 896.139
	\end{bmatrix}
\end{equation}
\subsection{转动运动计算}

如图~\ref{fig:motion_sim}的转动更新环节所示,在前一次迭代的结果的输入下,通过转动更新求出第$n+1$的角速度$\boldsymbol{\omega}_{n+1}$从而进一步得到第$n+1$帧的四元数$\mathbf{q}_{n+1}$作为更新。但是考虑到角速度随欧拉方程演化的复杂性。需要设计一个更优化的迭代方式来完成角速度$\boldsymbol{\omega}_{n+1}$的计算。

考虑到在无外力矩作用下,刚体的角动量在惯性系中守恒,对应的刚体转动满足经典欧拉刚体方程。若以$\boldsymbol{\omega}$表示刚体绕其质心的角速度矢量(在刚体本体坐标系下表达),则连续形式的欧拉方程可写为
\begin{equation}\label{eq:euler}
	\mathbf{I}\,\dot{\boldsymbol{\omega}} + \boldsymbol{\omega} \times \bigl(\mathbf{I}\,\boldsymbol{\omega}\bigr) = \mathbf{0}
\end{equation}
不考虑数值误差时,$\mathbf{I}\boldsymbol{\omega}$(即刚体的角动量)在惯性系中应保持恒定,从而刚体的转动动能也随时间保持不变。


在无外力矩作用下,刚体的角动量在惯性系中守恒,对应的刚体转动满足经典欧拉刚体方程。如果以 $\boldsymbol{\omega}$ 表示刚体绕其质心的角速度矢量(在刚体本体坐标系下表达),则有
\begin{equation}\label{eq:euler}
	\mathbf{I}\,\dot{\boldsymbol{\omega}}
	\;+\;
	\boldsymbol{\omega} \times \bigl(\mathbf{I}\,\boldsymbol{\omega}\bigr)
	\;=\;
	\mathbf{0}
\end{equation}
这里 $\mathbf{I}$ 为惯性张量(在刚体本体坐标系下),$\mathbf{I}\,\boldsymbol{\omega}$ 即为刚体在惯性系中的角动量。由于无外力矩,刚体角动量与转动动能均应保持不变。

这里采用隐式中点法对该方程离散。令时间步长为 $\Delta t$,记 $t_{n+1} = t_n + \Delta t$,在时刻 $t_n$ 有角速度 $\boldsymbol{\omega}_n$,在时刻 $t_{n+1}$ 的未知量为 $\boldsymbol{\omega}_{n+1}$,则离散格式为
\begin{equation}\label{eq:midpoint_formula}
	\boldsymbol{\omega}_{n+1}
	\;=\;
	\boldsymbol{\omega}_{n}
	\;-\;
	\Delta t\,\mathbf{I}^{-1}\Bigl[
	\Bigl(\tfrac12\bigl(\boldsymbol{\omega}_{n}+\boldsymbol{\omega}_{n+1}\bigr)\Bigr)
	\,\times\,
	\mathbf{I}\,\Bigl(\tfrac12\bigl(\boldsymbol{\omega}_{n}+\boldsymbol{\omega}_{n+1}\bigr)\Bigr)
	\Bigr]
\end{equation}
由于右侧同时包含 $\boldsymbol{\omega}_{n+1}$ 的非线性项,需要通过 Newton-Raphson 迭代来求解。设迭代中间量为 $\boldsymbol{\omega}^{(k)}$,则迭代格式为
\begin{equation}
	\boldsymbol{\omega}^{(k+1)}
	\;=\;
	\boldsymbol{\omega}^{(k)}
	\;-\;
	\bigl[J_f\bigl(\boldsymbol{\omega}^{(k)}\bigr)\bigr]^{-1}
	\,f\bigl(\boldsymbol{\omega}^{(k)}\bigr),
\end{equation}
其中
\begin{equation}
	f(\boldsymbol{\omega})
	\;=\;
	\boldsymbol{\omega}
	\;-\;
	\boldsymbol{\omega}_n
	\;+\;
	\Delta t\,\mathbf{I}^{-1}\biggl[
	\Bigl(\tfrac12\bigl(\boldsymbol{\omega}_n + \boldsymbol{\omega}\bigr)\Bigr)
	\;\times\;
	\mathbf{I}\,
	\Bigl(\tfrac12\bigl(\boldsymbol{\omega}_n + \boldsymbol{\omega}\bigr)\Bigr)
	\biggr].
\end{equation}
通过数次迭代使 $f(\boldsymbol{\omega}^{(k)}) \approx \mathbf{0}$,即可得到满足隐式中点公式的 $\boldsymbol{\omega}_{n+1}$。该方法在长时间仿真下相比显式格式能更好地保持数值稳定性与角动量守恒。


在获得新的角速度 $\boldsymbol{\omega}_{n+1}$ 后,代码对刚体姿态矩阵 $R_n$ 进行更新。记中点角速度为
\begin{equation}
	\boldsymbol{\omega}_{\mathrm{avg}}
	\;=\;
	\tfrac12\bigl(
	\boldsymbol{\omega}_n + \boldsymbol{\omega}_{n+1}
	\bigr).
\end{equation}
则在 $\Delta t$ 的时间间隔内,刚体的旋转增量可表示为
\begin{equation}
	\Delta \theta
	\;=\;
	\boldsymbol{\omega}_{\mathrm{avg}}\,\Delta t,
\end{equation}
并令
\begin{equation}
	\Delta R
	\;=\;
	\exp\bigl(\widehat{\Delta \theta}\bigr),
\end{equation}
其中 $\exp(\cdot)$ 是从 $\mathfrak{so}(3)$ 到 $SO(3)$ 的指数映射,$\widehat{\cdot}$ 表示将矢量转换为反对称矩阵。	

然后更新姿态矩阵:
\begin{equation}
	R_{n+1}
	\;=\;
	R_{n}\,\Delta R.
\end{equation}
数值计算中,为避免在多步迭代后因舍入误差破坏 $R_{n+1}$ 的正交性,会在每一步后通过 SVD 或 QR 分解对 $R_{n+1}$ 做一次正交化,使其严格落在正交群 $SO(3)$ 上。

从表~\ref{tab:inial_cond_motion}中角动量误差$h_{\mathrm{rel}}$可以看出数据都在$1e-04$的量级以下, 说明该隐式中点法结合牛顿迭代与矩阵指数姿态更新,在无外力矩条件下能较好地保持角动量数值稳定,生成的6D姿态数据能够很好的贴近真实的空间非合作目标自由翻滚的场景。



\subsection{平动运动计算}

如图~\ref{fig:motion_sim}所示,接下来进行平动更新操作
对于不受外力作用的刚体,质心做匀速直线运动,其经典运动方程为
\begin{equation}\label{eq:translation}
	\mathbf{r}(t) \;=\;
	\mathbf{r}_{0}
	\;+\;
	\mathbf{v}_{0}\,t.
\end{equation}
在数值仿真中,若令平动在每个时间步也以显式欧拉或相同步长更新,则
\begin{equation}
\mathbf{r}_{n+1} 
\;=\;
\mathbf{r}_{n} 
\;+\;
\mathbf{v}_{0}\,\Delta t.
\end{equation}
由于$\mathbf{v}_0$恒定,平动部分不会产生数值累积误差,能够精准保持匀速运动规律。

\subsection{模拟过程的推进与终止}
如图~\ref{fig:motion_sim}所示, 仿真一直持续到时刻T,时刻T由表~\ref{tab:inial_cond_motion}中的每个运动过程的初始位置,末端位置以及平动速度大小所决定。在时刻T以内,进行不断的循环迭代,计算出该过程中每帧的6D姿态,达到时刻T时,该流程结束,从而完成空间自由翻滚目标的运动模拟得到该运动过程的6D姿态标注。

\begin{figure}[H]
	\centering
	\begin{tikzpicture}[
		node distance=1.6cm,
		>=stealth,
		auto,
		thick,
		scale=0.95,
		transform shape
		]
		% 定义样式
		\tikzstyle{startstop} = [draw, ellipse, minimum width=2.5cm, minimum height=1.2cm, align=center, line width=0.8pt]
		\tikzstyle{process} = [draw, rectangle, minimum width=10cm, minimum height=1.5cm, align=center, line width=0.8pt]
		\tikzstyle{smallprocess} = [draw, rectangle, minimum width=7cm, minimum height=1.5cm, align=center, line width=0.8pt]
		\tikzstyle{decision} = [draw, diamond, aspect=2.5, minimum width=8cm, minimum height=2.8cm, align=center, inner sep=1pt, line width=0.8pt]
		\tikzstyle{io} = [draw, trapezium, trapezium left angle=70, trapezium right angle=110, minimum width=10cm, minimum height=1.5cm, align=center, line width=0.8pt]
		\tikzstyle{arrow} = [thick, ->, >=stealth]
		
		% 节点
		\node (start) [startstop] {开始};
		\node (init) [process, below=of start] {初始化: 设定初始状态};
		\node (rot) [smallprocess, below=of init] {转动更新: 迭代求 $\boldsymbol{\omega}_{n+1}$, 更新 $\mathbf{q}_{n+1}$};
		\node (trans) [smallprocess, below=of rot] {平动更新: $\mathbf{r}_{n+1} = \mathbf{r}_n + \mathbf{v}_0\,\Delta t$};
		\node (time) [smallprocess, below=of trans] {时间推进: $t \leftarrow t + \Delta t$};
		\node (dec) [decision, below=of time, yshift=-0.5cm] {是否达到\\终止时刻 $T$?};
		\node (output) [io, below=of dec, yshift=-0.5cm] {输出末状态: $\mathbf{r}_T$, $\mathbf{q}_T$, $\mathbf{v}_T$, $\boldsymbol{\omega}_T$};
		\node (end) [startstop, below=of output] {结束};
		
		% 连接箭头
		\draw [arrow] (start) -- (init);
		\draw [arrow] (init) -- (rot);
		\draw [arrow] (rot) -- (trans);
		\draw [arrow] (trans) -- (time);
		\draw [arrow] (time) -- (dec);
		\draw [arrow] (dec) -- node[right, pos=0.4] {是} (output);
		\draw [arrow] (output) -- (end);
		\draw [arrow] (dec.west) -- +(-1.5,0) |- node[left, pos=0.25] {否} (rot.west);
		
	\end{tikzpicture}
	\caption{空间自由翻滚目标运动模拟的计算流程}
	\label{fig:motion_sim}
\end{figure}


%如表~\ref{tab:inial_cond_motion}所示,下面三组数据为仿真生成,表中规定了仿真运动的初末尾位置,速度$v$和角速度$\omega$,以及仿真误差的关键指标最大动量误差$\Delta h_{\mathrm{rel}}$。由角于动量守恒,其$\Delta h_{\mathrm{rel}}$应该趋近于0,从表中可以看出$\Delta h_{\mathrm{rel}}$的值的量级均在 1e-04 或者 1e-05,其仿真精度符合要求。刚体自由旋转仿真往往对数值方法较为敏感:一方面,初始姿态与角速度的微小变化可能在长时间后累积为明显偏差;另一方面,由于无外力矩环境下应保持角动量恒定,数值积分若稍有不慎便会出现能量或动量漂移。为实现高精度数值仿真,需要在积分方法的稳定性、旋转参数化方式以及时间步长的选取等方面综合权衡。本章利用了上述改进自适应Runge-Kutta积分方法,并在每一步迭代中进行舍入误差估计与Newton-Raphson修正,从而有效抑制累积误差对角动量守恒的影响。



\begin{table}[htbp]
	\centering
	\caption{运动模拟的初始参数}
	\label{tab:inial_cond_motion}
	{%
		\zihao{5}% 将表内字号设为五号
		\setlength{\tabcolsep}{4pt}% 减小列间距
		\begin{tabular}{cccccc}
			\toprule[1.5pt]
			Case 
			& \begin{tabular}{c}初始位置\\$\bigl(x,y,z\bigr)$ (m)\end{tabular}
			& \begin{tabular}{c}末端位置\\$\bigl(x,y,z\bigr)$ (m)\end{tabular}
			& $v$ (m/s) 
			& \begin{tabular}{c}$\bigl(\omega_x,\ \omega_y,\ \omega_z\bigr)$\\(rad/s)\end{tabular}
			& $\Delta h_{\mathrm{rel}}$\\
			\midrule[1pt]
			1 
			& (11.04,\,-2.93,\,63.11) 
			& (2.18,\,0.07,\,30.16) 
			& 0.5247 
			& (0.846,\,0.390,\,0.427) 
			& 1.05e-04\\
			\midrule[1pt]
			2 
			& (-3.06,\,-4.35,\,60.55) 
			& (1.45,\,0.02,\,29.99) 
			& 1.4887 
			& (-1.210,\,0.578,\,-0.612) 
			& 2.19e-04\\
			\midrule[1pt]
			3 
			& (-2.20,\,0.76,\,62.78) 
			& (1.16,\,-0.01,\,30.02) 
			& 0.9299 
			& (-0.115,\,-0.396,\,0.533) 
			& 4.85e-05\\
			\bottomrule[1.5pt]
		\end{tabular}
	}
\end{table}


\section{自由翻滚的空间非合作目标图像渲染}

完成运动学仿真得到6D位姿标注后,可将Starlink模型与相应的6D姿态标注一同加载至BlenderProc。与静态渲染类似,按照第~\ref{chap:attention_kpt}表~\ref{tab:camera-param}与表~\ref{tab:env-param}设置好渲染参数后,就可以根据6D姿态标注文件,设置Starlink的旋转和平移来得到空间非合作目标相应的自由翻滚运动过程中的连续帧6D姿态图像。如图\ref{fig:starlink_render_dyn}所示,这十张图片生成的数据集中抽取的部分图像的展示,可以看出目标的自由翻滚的变化状态。
\begin{figure}[htbp]
	\centering
	\includegraphics[width=0.8\textwidth]{Img/starlink_motion.png}
	\caption{空间自由翻滚目标位姿图像}
	\label{fig:starlink_render_dyn}
	\vspace{-3ex}
\end{figure}

\section{6D姿态的EKF滤波方法}

针对空间自由翻滚目标的6维姿态(位置和姿态)跟踪,本文采用无外力作用下的恒定速度运动学模型作为状态方程。设目标相对于相机坐标系的状态向量为:
\begin{equation}
	x = \begin{bmatrix}
		p^T & q^T & v^T & \omega^T
	\end{bmatrix}^T
\end{equation}
其中$p \in \mathbb{R}^3$表示目标质心的位置,$q \in \mathbb{R}^4$为描述目标姿态的单位四元数(采用${w, x, y, z}$参数化形式,且$|q|=1$),$v \in \mathbb{R}^3$和$\omega \in \mathbb{R}^3$分别表示目标的线速度和角速度。在无外力和无外力矩条件下,线速度和角速度可近似视为常量不变。因而,其连续时间运动学方程可表示为: \begin{equation}\label{eq:continuous_model} \dot{p} = v, \qquad \dot{v} = 0, \qquad \dot{q} = \frac{1}{2}\Omega(\omega)q, \qquad \dot{\omega} = 0, \end{equation} 其中$\Omega(\omega)$为由角速度$\omega=[\omega_x,\omega_y,\omega_z]^T$构成的四元数乘积矩阵:
\begin{equation}
	\Omega(\omega) = 
	\begin{bmatrix}
		0        & -\omega_x & -\omega_y & -\omega_z \\
		\omega_x & 0         & \omega_z  & -\omega_y \\
		\omega_y & -\omega_z & 0         & \omega_x  \\
		\omega_z & \omega_y  & -\omega_x & 0 
	\end{bmatrix}
\end{equation}
使得四元数微分$\dot{q} = \frac{1}{2}\Omega(\omega)q$与姿态角速度满足等价关系。式\eqref{eq:continuous_model}表明位置和姿态以当前速度进行匀速变化,而速度本身保持恒定。对上述连续模型进行离散化(时间采样间隔为$\Delta t$),可获得离散时间状态转移方程: \begin{equation}\label{eq:discrete_model} \begin{aligned} p_{k} &= p_{k-1} + v_{k-1}\Delta t,\\ v_{k} &= v_{k-1},\\ q_{k} &= q_{k-1} \otimes \exp\left( \frac{1}{2}\omega_{k-1}\Delta t \right),\\ \omega_{k} &= \omega_{k-1}, \end{aligned} \end{equation} 其中$\exp(\frac{1}{2}\omega \Delta t)$表示将角速度$\omega$经$\frac{1}{2}\omega \Delta t$轴角表示后转换为对应增量四元数,$\otimes$表示四元数乘法运算。需要注意,为保持四元数为单位模,更新后需对$q_k$进行归一化处理。式\eqref{eq:discrete_model}定义了滤波的状态演化函数$f(\cdot)$。在实际实现中,可根据需要在上述理想运动模型中加入过程噪声$w_k$以建模未确定的加速度或模型误差,此处过程噪声协方差记为$Q$。

有了状态方程后,本文采用扩展卡尔曼滤波(EKF)对目标姿态进行递推估计。EKF可递归地融合先验运动模型预测和新帧观测信息,从而在存在噪声的情况下得到对状态真实值的最优估计。滤波的两个主要步骤如下:

时间更新(预测):根据运动模型\eqref{eq:discrete_model}对上一时刻的状态估计进行外推,预测当前时刻的先验状态均值和协方差: \begin{align} \hat{x}_{k|k-1} &= f(\hat{x}_{k-1|k-1}), \label{eq:predict_state}\\ P_{k|k-1} &= F_{k-1}P_{k-1|k-1}F_{k-1}^T + Q. \label{eq:predict_cov} \end{align} 其中$\hat{x}_{k-1|k-1}$为上一时刻后验状态估计,$\hat{x}_{k|k-1}$为依据模型得到的当前先验状态估计;$P_{k-1|k-1}$为上一时刻后验协方差矩阵,$P_{k|k-1}$为当前先验协方差矩阵;$F_{k-1} = \frac{\partial f}{\partial x}\big|_{x=\hat{x}_{k-1|k-1}}$是对状态转移函数在上一估计点的雅可比矩阵。

测量更新(校正):获取当前时刻的观测$z_k$后,利用观测模型$h(\cdot)$计算预测观测值,并结合实际观测进行校正,得到后验状态估计和协方差: \begin{align} y_k &= z_k - h(\hat{x}_{k|k-1}), \label{eq:innov}\\ K_k &= P_{k|k-1}H_k^T \big(H_kP_{k|k-1}H_k^T + R\big)^{-1}, \label{eq:kgain}\\ \hat{x}_{k|k} &= \hat{x}_{k|k-1} + K_ky_k, \label{eq:update_state}\\ P_{k|k} &= (I - K_kH_k)P_{k|k-1}. \label{eq:update_cov} \end{align} 其中$y_k$称为创新量,为观测与先验预测之间的残差;$H_k = \frac{\partial h}{\partial x}\big|_{x=\hat{x}_{k|k-1}}$是观测函数在先验状态估计处的雅可比矩阵;$K_k$为Kalman增益;$R$为观测噪声协方差。式\eqref{eq:update_state}利用观测残差按增益矩阵对先验状态进行校正修正,\eqref{eq:update_cov}则给出了相应的协方差更新公式。通过上述预测-校正过程,滤波器能够在每一时刻融合模型预测和量测信息,不断逼近目标的真实状态。

值得一提的是,在滤波初始时刻需要提供初始状态$\hat{x}_{0|0}$及初始协方差$P_{0|0}$。其中姿态初值$(p_0, q_0)$可由前文第\ref{chap:attention_kpt}章与第\ref{chap:RANSAC-TRO-SQPnP}章的方法从首帧图像估计得到,初始速度$(v_0, \omega_0)$可以根据先验知识设定或初始化为零,协方差$P_{0|0}$则反映初始估计的不确定性大小。

\section{基于像素关键点的观测模型}

在姿态滤波的观测模型设计上,本文将前文提取的图像像素关键点作为观测量,引入EKF滤波过程。这一做法的核心是利用多帧图像中稳定观测到的目标特征点,提高单帧姿态估计的鲁棒性和稳定性。具体而言,假设目标物体上定义了若干静止的3D关键点,其在目标自身坐标系下的坐标已知(通过先验模型获取,例如目标的特征结构点)。记第$i$个关键点的物体坐标为
\begin{equation}
	\mathbf{P}_i^{\mathrm{obj}}=[X_i^{\mathrm{obj}}, Y_i^{\mathrm{obj}}, Z_i^{\mathrm{obj}}]^T
\end{equation}
对于相机在第$k$帧的姿态状态$(p_k,q_k)$,该3D点在相机坐标系下的坐标为:

\begin{equation}
	\mathbf{P}_i^{\mathrm{cam}} = R\bigl(q_k\bigr)\,\mathbf{P}_i^{\mathrm{obj}} + \mathbf{p}_k
\end{equation}
其中$R(q_k)\in SO(3)$表示由四元数$q_k$对应的旋转矩阵。再利用相机成像模型(假设针孔相机模型及已知内参),可将$\mathbf{P}_i^{\mathrm{cam}}=[X_i, Y_i, Z_i]^T$投影到图像平面得到像素坐标: 
\begin{equation}
	\label{eq:projection}
	 u_i = f_x \frac{X_i}{Z_i} + c_x, \qquad v_i = f_y \frac{Y_i}{Z_i} + c_y, 
\end{equation} 其中$(u_i,v_i)$为第$i$个关键点的像素坐标观测,$(f_x, f_y)$和$(c_x, c_y)$分别为相机焦距和主点坐标内参。将所有可观测的$n$个关键点像素坐标加以堆叠,形成观测向量:
\begin{equation}
	z_k = \bigl[u_1,\;v_1,\;u_2,\;v_2,\;\dots,\;u_n,\;v_n\bigr]^T \;\in\; \mathbb{R}^{2n}.
\end{equation}

这样,观测模型可定义为
\begin{equation}
	z_k = h(x_k) + v_k
\end{equation}
其中$h(x_k)$表示由状态$x_k$计算得到的所有关键点投影坐标(对应式\eqref{eq:projection}),$v_k$是观测噪声向量(假设为零均值高斯噪声,协方差为$R$)。

由于投影模型$h(\cdot)$具有显著的非线性,EKF在执行观测更新时需要对其进行一阶线性化处理,求取观测函数关于状态的雅可比矩阵$H_k$。根据定义,
\begin{equation}
	H_k = \frac{\partial h}{\partial x}\big|_{x=\hat{x}_{k|k-1}}
\end{equation}
为一个$2n \times (3+4+3+3)$维矩阵(本方法状态量为$3$维位置+$4$维四元数+$3$维线速度+$3$维角速度,共13维)。其中,与位置和姿态相关的雅可比子块为主要部分,而由于像素坐标观测对速度分量不直接敏感,$H_k$中相应列将为零。以下结合单个关键点$i$推导其观测方程的雅可比矩阵。

对第$i$个关键点,有
\begin{equation}
	z_{i} = [u_i, v_i]^T = h_i(p,q),
\end{equation}
其中$p$和$q$分别影响$\mathbf{P}_i^{\mathrm{cam}}$的位置和取向。对姿态的小扰动可用李代数形式描述:设存在一个微小旋转矢量$\delta \boldsymbol{\phi}\in \mathbb{R}^3$和微小平移$\delta \mathbf{t}\in\mathbb{R}^3$,作用在相机姿态$(p,q)$上引起相机坐标系下点坐标的变化$\delta \mathbf{P}_i^{\mathrm{cam}}$。利用旋转向量的反对称矩阵表示$[\delta \boldsymbol{\phi}]_\times$,有近似关系: \begin{equation}\label{eq:pcam_perturb} \delta \mathbf{P}_i^{\mathrm{cam}} = [\delta \boldsymbol{\phi}]_{\times}\mathbf{P}_i^{\mathrm{cam}} + \delta \mathbf{t}, \end{equation} 其中$[\delta \boldsymbol{\phi}]_{\times}$表示$\delta \boldsymbol{\phi}$对应的$3\times 3$反对称矩阵(满足$[\delta \boldsymbol{\phi}]_{\times} \mathbf{a} = \delta \boldsymbol{\phi} \times \mathbf{a}$)。将$\mathbf{P}_i^{\mathrm{cam}}=[X_i, Y_i, Z_i]^T$代入,可将\eqref{eq:pcam_perturb}按分量表示为:

\begin{align}
	\delta X_i &= (\delta \boldsymbol{\phi}\times \mathbf{P}_i^{\mathrm{cam}})_x + \delta t_x = \beta Z_i - \gamma Y_i + \delta t_x,\\ 
	\delta Y_i &= (\delta \boldsymbol{\phi}\times \mathbf{P}_i^{\mathrm{cam}})_y + \delta t_y = \gamma X_i - \alpha Z_i + \delta t_y,\\ 
	\delta Z_i &= (\delta \boldsymbol{\phi}\times \mathbf{P}_i^{\mathrm{cam}})_z + \delta t_z = \alpha Y_i - \beta X_i + \delta t_z,
\end{align}

其中$\delta \boldsymbol{\phi}=[\alpha,\beta,\gamma]^T$、$\delta \mathbf{t}=[\delta t_x,\delta t_y,\delta t_z]^T$。随后,将上述$\mathbf{P}_i^{\mathrm{cam}}$的扰动传播到成像平面,根据透视投影模型\eqref{eq:projection}对$(X_i,Y_i,Z_i)$求导,可得像素坐标对相机坐标变化的偏导数:
\begin{equation}
	\begin{aligned}
		\frac{\partial u_i}{\partial [X_i,\,Y_i,\,Z_i]} 
		&= \biggl[\frac{f_x}{Z_i},\;0,\;-\frac{f_x\,X_i}{Z_i^2}\biggr],\\[6pt]
		\frac{\partial v_i}{\partial [X_i,\,Y_i,\,Z_i]} 
		&= \biggl[0,\;\frac{f_y}{Z_i},\;-\frac{f_y\,Y_i}{Z_i^2}\biggr].
	\end{aligned}
\end{equation}
结合$\delta \mathbf{P}_i^{\mathrm{cam}}$,可线性近似得到像素观测的变化: \begin{equation}\label{eq:uv_perturb} \begin{aligned} \delta u_i &\approx \frac{f_x}{Z_i}\delta X_i - \frac{f_x X_i}{Z_i^2}\delta Z_i,\\ \delta v_i &\approx \frac{f_y}{Z_i}\delta Y_i - \frac{f_y Y_i}{Z_i^2}\delta Z_i. \end{aligned} \end{equation} 将\eqref{eq:pcam_perturb}代入\eqref{eq:uv_perturb},并分别取$\delta \boldsymbol{\phi}$和$\delta \mathbf{t}$的系数,就可得到观测方程对小旋转和小平移扰动的雅可比矩阵块$\frac{\partial (u_i,v_i)}{\partial \delta \boldsymbol{\phi}}$和$\frac{\partial (u_i,v_i)}{\partial \delta \mathbf{t}}$。将所有关键点$i=1,\dots,n$的雅可比子块按对应状态变量堆叠起来,最终获得$H_k$矩阵,用于滤波的观测更新步骤。

在实现过程中,每帧图像的关键点观测$z_k$来自于前两章方法的检测结果;第\ref{chap:RANSAC-TRO-SQPnP}章。在滤波时,应当利用检测算法输出的匹配关系,将对应的3D模型点$\mathbf{P}_i^{\mathrm{obj}}$与2D观测$[u_i, v_i]$正确对应,方能保证观测模型$h(x)$的准确性。如果某些关键点在当前帧未被检测到或被判为外点,应从观测向量中剔除,以免误差传入滤波器。在本文的实验中,假定匹配和外点剔除已由前序算法可靠完成,滤波器使用的观测为匹配良好的像素关键点集合。

通过以上方法,将原本单帧估计的6D姿态问题转化为在滤波框架下的多帧优化问题:滤波器通过不断校正投影关键点与观测关键点之间的偏差来修正姿态估计,从而平滑掉单帧解算中的噪声和异常。

\section{基于SE(3)李群的四元数姿态更新}

在扩展卡尔曼滤波的状态更新过程中,姿态参数的表示和更新方式对滤波精度有重要影响。由于旋转空间的非线性,直接对四元数分量进行加法更新并归一化的做法可能在大角度误差时引入不必要的偏差。为此,考察了近年与姿态滤波相关的研究,发现采用了基于李群$SE(3)$的四元数乘法更新策略,即在滤波校正阶段利用李代数上的小扰动来更新姿态,可以有效提高姿态估计的鲁棒性\cite{9780112,Fang2024Kinematic}。因此本章将其引入,作为EKF滤波器中四元数的更新策略以尽可能降低四元数更新的误差。

李群$SE(3)$是刚体在三维空间的欧式运动群,其元素可表示为位置和平移的组合$(R, t)$,其中$R\in SO(3)$为旋转矩阵(或等价的四元数$q$),$t\in \mathbb{R}^3$为平移向量。$SE(3)$对应的李代数$\mathfrak{se}(3)$由$\mathbb{R}^6$表示,可写作$\boldsymbol{\xi} = \begin{bmatrix} \boldsymbol{\phi}^T & \mathbf{u}^T \end{bmatrix}^T$,其中$\boldsymbol{\phi}\in \mathbb{R}^3$表示旋转部分的微小轴角(其反对称矩阵$[\boldsymbol{\phi}]_\times$属于$\mathfrak{so}(3)$),$\mathbf{u}\in \mathbb{R}^3$表示平移部分的小量。李群到李代数的指数映射$\exp: \mathfrak{se}(3) \to SE(3)$可将一个小扰动$\boldsymbol{\xi}$映射为对应的群增量

\begin{equation}
	\exp(\boldsymbol{\xi}) = (\exp(\boldsymbol{\phi}), \mathbf{u}')
\end{equation}

,其中$\exp(\boldsymbol{\phi})$是$\mathfrak{so}(3)$到$SO(3)$的指数映射(即 Rodrigues 公式或等价的轴角到旋转矩阵/四元数的转换),$\mathbf{u}'$则是考虑旋转影响后的平移增量。对于纯旋转情况,$\exp(\boldsymbol{\phi})$可直观地表示为四元数增量

\begin{equation}
	\delta q = \begin{bmatrix} \cos(\frac{|\boldsymbol{\phi}|}{2}) & \sin(\frac{|\boldsymbol{\phi}|}{2})\frac{\boldsymbol{\phi}}{|\boldsymbol{\phi}|} \end{bmatrix}^T
\end{equation}

。当$|\boldsymbol{\phi}|$很小时,有近似

\begin{equation}
	\delta q \approx \begin{bmatrix} 1 & \frac{1}{2}\boldsymbol{\phi}^T \end{bmatrix}^T
\end{equation}

。

基于上述李群理论,在EKF的状态校正阶段,将姿态的误差以李代数形式来表征,并通过群运算完成姿态的更新。具体来说,观测更新步骤\eqref{eq:update_state}得到的增量$\Delta \hat{x}_k = K_ky_k$中,包括了姿态的修正量。设滤波计算得到的小旋转修正为$\delta \boldsymbol{\phi}_k \in \mathbb{R}^3$,对应的小平移修正为$\delta \mathbf{t}_k \in \mathbb{R}^3$。将$\delta \boldsymbol{\phi}_k$转换为等价的增量四元数$\delta q_k = \exp(\delta \boldsymbol{\phi}_k)$,并采用四元数乘法来更新先验姿态四元数: \begin{equation}
	\label{eq:q_update} q_{k|k} = \delta q_k \otimes q_{k|k-1},
\end{equation} 同时平移部分按通常的向量加法更新: 
\begin{equation}
\label{eq:t_update}
 p_{k|k} = p_{k|k-1} + \delta \mathbf{t}_k. 
 \end{equation} 
 上述\eqref{eq:q_update}--\eqref{eq:t_update}一起完成了将状态在李群上的更新,可近似表示为:
\begin{equation}
	\label{eq:X_update}
	X_{k|k} \;=\; \exp\bigl([\delta \phi_k,\, \delta t_k]\bigr)\;\cdot\;X_{k|k-1}.
\end{equation}
其中$X = (R, t) \in SE(3)$表示姿态的群元素。需要注意,每次更新后应对$q_{k|k}$进行归一化处理,以确保其仍为单位四元数。至此,滤波算法在更新阶段采用乘法形式修正姿态,使得即使面对较大的姿态误差,更新结果依然能够保证在旋转空间上的准确性和有效性。这种基于李群的四元数更新策略本质上等价于航天领域常用的乘法扩展卡尔曼滤波(Multiplicative EKF)方法,在保持姿态参数合法性的同时,提高了滤波对非线性姿态变化的鲁棒性。

\section{算法流程与实现}

综合以上运动模型、观测模型和李群更新策略,本章设计的基于关键点观测的姿态EKF滤波算法流程如图\ref{fig:flowchart}所示。首先,利用初始若干帧图像(通常为第一帧)的姿态估计结果来初始化滤波器状态,包括目标初始位姿$(p_0, q_0)$及初始化协方差$P_{0|0}$等;线速度和角速度初值$(v_0, \omega_0)$可根据任务先验设定(若未知则可设为零)。随后,对每一个新的图像帧,按照以下步骤循环执行:(1) 输入当前帧图像,运行关键点检测网络提取像素关键点,并利用匹配算法获取观测集合$z_k$;(2) 根据运动学模型对上一时刻姿态状态进行外推,预测当前状态先验$\hat{x}_{k|k-1}$及协方差$P_{k|k-1}$;(3) 利用先验姿态$\hat{x}_{k|k-1}$将目标3D关键点投影到图像平面,计算预测观测$h(\hat{x}_{k|k-1})$,并与实际观测$z_k$比较,得到创新量$y_k$;(4) 计算卡尔曼增益$K_k$,据此修正状态估计,其中姿态部分采用四元数SE(3)更新实现,得到后验状态$\hat{x}_{k|k}$和协方差$P_{k|k}$;(5) 将校正后的姿态估计输出,并作为下一时刻预测的起点,进入下一帧滤波循环。重复上述过程,直至序列帧结束。通过在时间序列上融合多帧信息,算法能够平滑掉单帧估计中的噪声波动,提高6D姿态跟踪的整体精度和稳定性。


\begin{figure}[htbp] 
	\centering 
	\begin{tikzpicture}[
		scale=1.2, % 这里添加了缩放因子,0.8表示缩小到原来的80%
		transform shape, % 确保节点和文本也相应缩放
		node distance=2cm,auto,>=latex', every node/.style={font=\small}] 
		\tikzstyle{block} = [rectangle, draw, align=center, fill=white, minimum height=1cm, minimum width=5.5cm]; 
		\tikzstyle{decision} = [diamond, draw, aspect=2, inner sep=1pt, align=center, fill=white]; 
		\tikzstyle{line} = [->, thick]; 
		\tikzstyle{terminator} = [ellipse, draw, fill=white, minimum height=0.8cm, minimum width=2cm];
		
		\node [terminator] (start) {开始};
		\node [block, below of=start, node distance=2cm] (init) {初始化状态 $\hat{x}_{0|0}=(p_0,q_0,v_0,\omega_0)$ \\ 初始化协方差 $P_{0|0}$};
		\node [block, below of=init, node distance=2cm] (predict) {基于运动模型进行状态预测 \\ 计算 $\hat{x}_{k|k-1}$ 和 $P_{k|k-1}$};
		\node [block, below of=predict, node distance=2cm] (observe) {获取第$k$帧图像并检测关键点 \\ 得到观测 $z_k$ (关键点 像素坐标集合)};
		\node [block, below of=observe, node distance=2cm] (update) {计算创新 $y_k = z_k - h(\hat{x}_{k|k-1})$ \\ 计算增益 $K_k$ 并更新状态 $\hat{x}_{k|k}$、$P_{k|k}$ \\ (姿态SE(3)更新)};
		\node [decision, below of=update, node distance=2.5cm] (decide) {是否还有下一帧?};
		\node [terminator, below of=decide, node distance=2cm] (end) {结束};
		
		\draw [line] (start) -- (init);
		\draw [line] (init) -- (predict);
		\draw [line] (predict) -- (observe);
		\draw [line] (observe) -- (update);
		\draw [line] (update) -- (decide);
		\draw [line] (decide) -- node [right] {否} (end);
		
		% 改进的回路箭头,使用更平滑的路径
		\draw [line] (decide) -- node[above, fill=white] {是} ++(-3.5,0) |- (predict.west);
	\end{tikzpicture}
	\caption{基于关键点观测的6D姿态EKF滤波算法流程图}
	\label{fig:flowchart}
\end{figure}
\section{实验分析}
为验证本章所述方法,本节从上述生成的三个 case 中进行验证,设定三种滤波方案并进行对比,分别是EKF方案、EKF结合关键点观测方案,以及EKF结合关键点观测和基于李群SE(3)的误差状态表达方案。

从表~\ref{tab:caseekfcmp}中可以看出,EKF滤波方案无论在$err_{\text{ort}}^{\circ}$还是$score_{\text{pst}}$指标,以及综合指标 $score$ 上均没有明显改进,甚至在当前精度下难以观察到显著变化。这进一步说明在空间非合作目标的自由翻滚这种复杂情境下,传统滤波方案存在局限性。而通过引入关键点观测的基于SE(3)EKF滤波方案效果得到了显著提高,再结合误差状态表达,其滤波效果得到进一步改善。从表中数据可以看出,基于关键点观测的SE(3)EKF滤波方案在$err_{\text{ort}}^{\circ}$指标上三个运动过程分别超越了EKF滤波 8.18\%、2.88\%和2.54\%;在$score_{\text{pst}}$指标上三个运动过程分别超越了EKF滤波 53.85\%、52.17\%和40\%;在综合指标$score$上三个运动过程分别超越了EKF滤波 25.21\%、17.30\%和11.20\%。三个运动过程平均下来,基于关键点观测的SE(3)EKF滤波方案的 $score$ 指标达到了0.0230,超越EKF方案$score$ 18.7\%。

\begin{table}[htbp]
	\centering
	\caption{精度结果对比}
	\label{tab:caseekfcmp}
	{%
		\zihao{5}% 将表内字号设为五号
		\begin{tabular}{c c c c c}
			\toprule[1.5pt]
			运动过程 & Method 
			& $err_{\text{ort}}^{\circ}$ 
			& $score_{\text{pst}}$ 
			& $score$ \\
			\midrule[1pt]
			\multirow{4}{*}{1} 
			& 未滤波                       & 1.2793 & 0.0130 & 0.0353 \\
			& EKF滤波                       & 1.2782 & 0.0130 & 0.0353 \\
			& 基于关键点观测的EKF滤波            & 1.2001 & 0.0060 & 0.0268 \\
			& 基于关键点观测的SE(3)EKF滤波       & 1.1737 & 0.0060 & 0.0264 \\
			\midrule[1pt]
			\multirow{4}{*}{2}
			& 未滤波                       & 0.9679 & 0.0069 & 0.0238 \\
			& EKF滤波                       & 0.9604 & 0.0069 & 0.0237 \\
			& 基于关键点观测的EKF            & 0.9478 & 0.0031 & 0.0197 \\
			& 基于关键点观测的SE(3)EKF      & 0.9327 & 0.0033 & 0.0196 \\
			\midrule[1pt]
			\multirow{4}{*}{3}
			& 未滤波                       & 1.1463 & 0.0060 & 0.0260 \\
			& EKF滤波                       & 1.1418 & 0.0060 & 0.0259 \\
			& 基于关键点观测的EKF            & 1.1196 & 0.0035 & 0.0231 \\
			& 基于关键点观测的SE(3)EKF      & 1.1128 & 0.0036 & 0.0230 \\
			\bottomrule[1.5pt]
		\end{tabular}
	}
\end{table}

如图\ref{fig:case1efk}、\ref{fig:case2efk}以及\ref{fig:case3efk}所示,三个运动过程的误差曲线随帧率变化图中,子图(a)的EKF方案中滤波前后的曲线几乎重合,这说明该方案几乎不能在空间非合作目标自由翻滚运行的情形中发挥滤波作用。而子图(b)的方案,滤波后的曲线有相当多的部分均位于滤波前的曲线之下,这进一步说明了基于关键点观测的SE(3)EKF滤波方法具有显著的滤波效果。

\begin{figure}[htbp]
	\centering
	\includegraphics[width=1.0\textwidth]{Img/case1ekf.png}
	\caption{(a) 运动过程1的EKF滤波方法的误差随帧号的变化曲线;(b) 运动过程1的基于关键点检测的SE(3)EKF滤波方法的误差随帧号的变化曲线}
	\label{fig:case1efk}
\end{figure}

\begin{figure}[htbp]
	\centering
	\includegraphics[width=1.0\textwidth]{Img/case2ekf.png}
	\caption{(a) 运动过程2的EKF滤波方法的误差随帧号的变化曲线;(b) 运动过程2的基于关键点检测的SE(3)EKF滤波方法的误差随帧号的变化曲线}
	\label{fig:case2efk}
\end{figure}

\begin{figure}[htbp]
	\centering
	\includegraphics[width=1.0\textwidth]{Img/case3ekf.png}
	\caption{(a) 运动过程3的EKF滤波方法的误差随帧号的变化曲线;(b) 运动过程3的基于关键点检测的SE(3)EKF滤波方法的误差随帧号的变化曲线}
	\label{fig:case3efk}
\end{figure}

为了量化对比三种方法在整个时域的平均误差,表\ref{tab:caseekfcmp}给出了角度误差与位置误差的数值统计。可以看出,EKF的角度误差为 $1.23^\circ$,位置误差为 $0.12\,\mathrm{m}$;EKF + 关键点观测角度误差降低到 $1.15^\circ$,位置误差降至 $0.10\,\mathrm{m}$;EKF + 关键点观测 + SE(3)方案进一步提升至$1.05^\circ$ 和 $0.08\,\mathrm{m}$,效果最佳。

从表~\ref{tab:caseekfcmp}中可以看出,未滤波情形下的角度误差和相对位置误差分别为 $1.4918^\circ$ 和 $0.01341\,\mathrm{m}$。采用EKF仅进行惯性预测的滤波后,这两项指标仅略微下降至 $1.4911^\circ$ 与 $0.01340\,\mathrm{m}$,说明在此仿真环境和噪声水平下,单纯依赖惯性信息并不能带来明显的效果提升。

当将关键点观测引入滤波过程后(即EKF + 关键点方案),角度误差从 $1.4918^\circ$ 下降至 $1.2033^\circ$,位置误差则由 $0.01341\,\mathrm{m}$ 显著降低到 $0.00635\,\mathrm{m}$,相比未滤波时分别提升约 $19\%$ 和 $53\%$。可见加入视觉关键点观测后,姿态与位置估计的稳健性和准确度均有明显改善。

在此基础上进一步采用基于 SE(3) 的误差状态表达(即EKF + 关键点 + SE(3)方案),角度误差降低至 $1.1912^\circ$,位置误差进一步收敛至 $0.00523\,\mathrm{m}$,较未滤波相比提升幅度分别约为 $20\%$ 和 $61\%$,比单纯使用关键点的方案更进一步。该结果说明:在纳入视觉关键点的基础上,通过对旋转群 $SE(3)$ 上的误差状态精细化建模与更新,能显著增强滤波在非线性姿态估计场景中的数值稳定性与精度。
\vspace{5pt}


\section{本章小结}

本章针对空间非合作目标自由翻滚运动时的6D姿态估计问题,提出了一种结合关键点观测与SE(3)李群更新的扩展卡尔曼滤波方法。首先,基于刚体力学原理构建了空间目标无外力无外力矩条件下的自由翻滚运动模型,并通过隐式中点法结合Newton-Raphson迭代保证了角动量守恒的数值稳定性,进而生成了具有高精度标注的连续运动序列数据。在此基础上,针对目标的关键特征点,设计了像素坐标观测模型,将其直接引入EKF滤波过程,实现了对姿态估计的直接修正;同时,通过在SE(3)李群上进行四元数乘法更新,有效地处理了旋转空间的非线性特性,提高了算法对大角度误差的鲁棒性。通过对比实验显示,传统EKF滤波在空间非合作目标复杂运动场景下几乎无效,而结合关键点观测与SE(3)李群更新的方案则显著提升了估计精度,角度误差平均降低约20\%,位置误差平均降低约61\%。研究表明,在空间非合作目标的姿态估计中,通过引入多帧信息融合并结合适当的几何约束,能够有效抑制单帧估计中的离群值,提高整体跟踪的稳定性和精度。这一成果为空间目标交会任务中的姿态估计和控制提供了更为可靠的技术支持。




