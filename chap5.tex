\chapter{基于关键点观测的空间自由翻滚非合作目标6D姿态滤波方法}
\label{chap:SpaceFreeMotionEKF}
\section{引言}
近年来,针对空间非合作目标的6D姿态估计研究已取得一定进展,但更多是局限于单帧静态图像的6D姿态估计。但在实际应用中,仍面临由于目标姿态复杂变化及测量噪声等因素导致的精度与稳定性不足问题。前文(第\ref{chap:attention_kpt}章与第\ref{chap:RANSAC-TRO-SQPnP}章)从6D姿态估计的两个阶段出发,提出了基于关键点检测与高鲁棒PnP优化的完整估计流程,并在静态图像数据上验证了其有效性。然而,实验结果(如图\ref{fig:hard_demo}所示)也显示,局部场景中会出现较高的姿态估计误差,对要求稳定性和连续性的应用场景构成挑战。
空间非合作目标在在轨失效时,常表现出自由翻滚不规则运动,若此时对目标实施姿态感知和捕获,则必须应对其剧烈且连续的转动和位移。尽管自由翻滚在直觉上给测量带来不确定性,但正是这类运动的时序信息为解决高误差估计提供了可能。基于此思想,若能对目标的运动学规律施加合理的先验假设,并通过卡尔曼滤波在时间序列上对估计结果进行平滑,则有望抑制孤立帧所带来的高误差。
卡尔曼滤波在航天器姿态估计中已有广泛应用,但传统扩展卡尔曼滤波(EKF)对空间非合作目标强非线性且复杂的自由翻滚运动仍存在局限。为此,本章基于第\ref{chap:attention_kpt}章与第\ref{chap:RANSAC-TRO-SQPnP}章估计方案估计出6D姿态的基础上,引入滤波过程,提出将像素关键点作为观测量融入EKF滤波过程;同时采用基于SE(3)李群的四元数状态更新策略,以在非线性旋转空间中获得更具鲁棒性的更新结果。然而前文的研究数据样本是空间非合作目标单帧静态图像的合集,没有能够构成动态连续帧的样本序列。因此本章首先基于空间目标自由翻滚运动的动力学特性,计算出相应的运动帧的6D姿态,然后借助BlenderProc渲染生成空间非合作目标自由翻滚运动的连续帧序列图像,对滤波算法进行验证和性能评估。相关实验结果将在关键点观测与SE(3)优化的联合加持下,滤波方法能够在多帧信息融合中抑制较高离群估计,提高姿态跟踪的整体精度与稳定性。
