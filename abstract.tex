随着人类对太空探索的不断深入,越来越多的航天器被送入地球轨道,同时也导致失效航天器或者残骸与日俱增,在占据轨道资源的同时也影响着正常的在轨运行航天器。维修这些航天器或者清理这些航天器成为了当前的一个潜在需求,为了达成这一目标。通常需要获取目标航天器或空间物体的相对位姿信息,为捕获这些航天器从而完成维修或者清理操作创造条件。然而,由于目标在轨的非合作特征,给位姿测量技术带来较大的挑战。为此本文采用了基于单目图像关键点检测的方案进行空间非合作目标的6D姿态估计。然而,由于单目相机只能获取二维图像,缺少直接深度信息,加之光照剧变、弱纹理和高亮背景等因素影响,其测量精度和鲁棒性仍面临较大挑战。同时航天器复杂多样的形状叠加复杂的运动使得其在单目图像上的投影存在明显差异,这也使得其位姿估计存在较大挑战

针对上述问题,本文提出了一种基于单目图像关键点检测的空间非合作目标6D姿态估计方法。主要工作如下:

	
(1)为提升关键点检测的效率、精度以及在遮挡和欠曝等情况下的鲁棒性,在YOLOv8架构基础上,提出将EfficientViT和Triplet Attention相结合的关键点检测网络,充分利用了注意力机制的优越性,通时也考虑了模块的轻量化,有效提高了检测的精度。针对3D纵深明显的目标在成像后,其关键点的像素坐标损失存在平衡性问题,本研究在YOLOv8的(Object Keypoint Similarity,OKS)损失函数的基础上,进一步提出了基于针孔相机模型改进的近似3D尺度(A3DKS, Approximate Keypoint Similarity)损失。消融实验表明,上述改进均能显著提升关键点检测的精度。

(2)在由关键点估计6D位姿的过程中,关键点检测模型所预测的噪声或异常关键点会对解算精度产生不利影响。为此,本研究引入了RANSAC-TRO SQPnP求解方法,能有效处理SQPnP基于重投影误差的优化目标以及二次规划问题的构造,得到相较传统PnP算法更为精确的解。针对异常关键点,则利用RANSAC算法将所有关键点划分为若干子集并分别输入SQPnP,根据重投影误差最小化的标准,选出使SQPnP误差最小的关键点子集,并将子集之外的点识别为异常点排除,从而有效提升位姿求解的精度和稳定性。通过在SPEED+数据集上验证第(1)点与第(2)点的6D姿态方法能够以	4.3M的参数了实现了1.0760°的旋转误差以及0.0459m的绝对位置误差。

(3)对于空间自由翻滚的运动目标,因其具有非线性运动特性,常规的卡尔曼滤波方案存在局限。本文选用针对四元数的SE(3)扩展卡尔曼滤波器(Extended Kalman Filter,EKF)。同时考虑到目标角速度非恒定的特性,传统仅对四元数进行滤波(如假设角速度恒定或仅作小修正)的做法往往会产生明显累积误差,因此将关键点检测网络输出的关键点作为EKF的观测向量,为滤波器提供更丰富的信息以进行修正,从而有效提升了自由翻滚运动目标的滤波性能。

综上所述,本研究针对硬件系统要求简易的单目视觉测量场景,提出了一种用于非合作空间目标的6D姿态测量算法。该算法能以相对高效且实时的方式提供目标位姿估计,并借助卡尔曼滤波充分利用目标运动信息对结果进行平滑,使姿态估计更具稳定性。该方法在轨服务以及空间碎片清理等应用中具有应用的潜力。
\keywords{6D姿态估计,空间非合作目标,关键点检测,卡尔曼滤波}% 中文关键词
