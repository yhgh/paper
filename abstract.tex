随着人类空间活动日益频繁,轨道上失效航天器和空间残骸不断增多,不仅占用了在轨资源,还对在轨航天器构成潜在威胁。对这些航天器的维修或残骸清理已成为当前需求,而准确获取空间目标的6D姿态信息是实现该目标的关键。然而,此类目标通常属于空间非合作目标,缺乏可利用的特征标志或交互手段,光照条件复杂、视角投影变化显著以及一些弱纹理的情况,这使得空间非合作目标的6D姿态估计面临挑战。

为有效解决上述难题,本文提出一种基于单目图像关键点检测的空间非合作目标6D姿态估计算法。主要工作包括:改进的关键点检测网络、鲁棒的位姿解算策略,以及适应非线性运动的滤波方法。具体如下:

(1)为提升关键点检测的效率和精度并增强其在目标截断、欠曝等情况下的鲁棒性,本文在YOLOv8架构基础上融合EfficientViT(Efficient Vision Transformer)和Triplet Attention,构建了轻量高效的关键点检测网络,充分利用注意力机制提升检测性能。针对具有明显3D纵深的目标在成像后关键点像素坐标损失存在平衡性问题,本文在YOLOv8的目标关键点相似度损失(Object Keypoint Similarity,OKS)基础上,引入了基于针孔相机模型改进的近似3D尺度关键点相似度损失(Approximate 3D Keypoint Similarity,A3DKS)。消融实验结果表明,上述改进均使综合指标$score^+$得到提升。

(2)在由关键点解算6D姿态的过程的PnP(Perspective-n-Point)问题中,关键点检测模型产生的噪声或异常关键点会影响6D姿态解算精度。因此,本文利用了基于重投影误差优化的SQPnP的求解算法,并结合随机采样一致性算法(Random Sample Consensus,RANSAC)剔除异常点,考虑到SQPnP算法优化的局限性,引入了信赖域优化(Trust Region Optimization,TRO)的方法来进一步优化,从而得到了更优的解。

RANSAC结合SQPnP的求解策略,有效处理了SQPnP的重投影误差优化目标及二次规划问题,获得相较传统PnP算法更精确的解。为剔除异常关键点,采用RANSAC将所有关键点划分为若干子集并分别进行SQPnP求解;根据重投影误差最小化准则选择误差最小的子集,将子集之外的点识别为异常点并剔除,从而提高6D姿态解算的精度和稳定性。基于SPEED+数据集的实验验证表明,融合了改进关键点检测和RANSAC-SQPnP求解的6D姿态估计算法在仅4.3M参数量和10.4 GFLOPs计算复杂度的条件下,实现了平均旋转误差1.0760°和绝对平移误差0.0459 m的精度。

(3)针对具有非线性运动特性的空间自由翻滚目标,常规卡尔曼滤波方法存在局限。本文采用基于四元数的SE(3)扩展卡尔曼滤波器(EKF)。鉴于目标角速度并非恒定,传统仅对四元数进行滤波(假设角速度恒定或仅作微小修正)容易产生累积误差。为此,本文将关键点检测网络输出的关键点作为EKF的观测向量,为滤波提供更丰富的信息以校正估计。在多个模拟空间自由翻滚目标的测试数据上,上述方案取得了平均旋转误差1.0730°和平均相对位置误差0.43\%的性能表现。与未融合关键点观测的EKF方案相比,该方案的综合得分提高了18.73\%,有效提升了自由翻滚目标姿态估计的稳定性。

综上所述,本研究面向硬件条件受限的单目视觉测量场景,提出了一种针对空间非合作目标的6D姿态测量算法。该算法在较低计算复杂度下即可估计空间非合作目标的6D姿态,并借助卡尔曼滤波充分利用目标运动信息对结果进行平滑,使姿态估计更加稳定。该方法在在轨服务与空间残骸清理等任务中具有潜在的应用价值。
