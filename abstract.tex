随着人类的空间活动的增加,在轨道的失效航天器和空间残骸日益增多,占用在轨资源的同时也对在轨航天器构成一定的威胁。维修这些航天器或者清理这些残骸成为了一个潜在需求,为实现空间目标的维修或清理,准确获取空间目标的6D姿态信息至关重要。然而,由于这类目标往往属于空间非合作目标,缺乏可利用的特征标志或者交互方式,且其表面材质,光长条件以及视角的变异性大,这使得传统的基于主动信标或者靶标的测量手段行不通,同时由于目标自身可能存在的弱纹理,光照过明或者过暗,以及遮挡等情况,这使得空间非合作目标的6D姿态成为挑战。

为了有效解决得空间非合作目标的6D姿态估计这一难题,同时也考虑到低成本可靠性,本文提出了一种基于单目图像关键点检测的空间非合作目标6D姿态估计方法,主要内容包括:改进的关键点检测网络、鲁棒的位姿解算策略以及适应非线性运动的滤波方法。具体工作如下:

(1)为提升关键点检测的效率、精度以及在遮挡和欠曝等情况下的鲁棒性,在YOLOv8架构基础上,提出将EfficientViT和Triplet Attention相结合的关键点检测网络,充分利用了注意力机制的优越性,同时也考虑了模块的轻量化,有效提高了检测的精度。针对3D纵深明显的目标在成像后,其关键点的像素坐标损失存在平衡性问题,本研究在YOLOv8的(Object Keypoint Similarity,OKS)损失函数的基础上,进一步提出了基于针孔相机模型改进的近似3D尺度(A3DKS, Approximate Keypoint Similarity)损失。消融实验表明,上述改进均在综合指标$score^+$上取得了提升。

(2)在由关键点估计6D位姿的过程中,关键点检测模型所预测的噪声或异常关键点会对解算精度产生不利影响。因此,本研究引入了RANSAC-TRO SQPnP求解方法,能有效处理SQPnP基于重投影误差的优化目标以及二次规划问题的构造,得到相较传统PnP算法更为精确的解。针对异常关键点,则利用RANSAC算法将所有关键点划分为若干子集并分别输入SQPnP,根据重投影误差最小化的标准,选出使SQPnP误差最小的关键点子集,并将子集之外的点识别为异常点排除,从而有效提升位姿求解的精度和稳定性。通过在SPEED+数据集上验证第(1)点与第(2)点结合的6D姿态方法能够以4.3M的参数量、10.4GFLOPs的计算复杂度实现了旋转误差$1.0760^\circ$以及绝对平移误差$0.0459$ m的精度。

(3)对于空间自由翻滚的运动目标,因其具有非线性运动特性,常规的卡尔曼滤波方案存在局限。本文选用针对四元数的SE(3)扩展卡尔曼滤波器(Extended Kalman Filter,EKF)。同时考虑到目标角速度非恒定的特性,传统仅对四元数进行滤波(如假设角速度恒定或仅作小修正)的做法往往会产生明显累积误差,因此将关键点检测网络输出的关键点作为EKF的观测向量,为滤波器提供更丰富的信息以进行修正,该方案在合成的多个空间自由翻滚目标的测试数据上取得了平均旋转误差$1.0730^\circ$,以及平均相对位置误差$0.43\%$。相对于EKF方案,其综合得分提高了$18.73\%$精度从而有效提升了自由翻滚运动目标的滤波性能。

综上所述,本研究针对硬件系统要求简易的单目视觉测量场景,提出了一种用于非合作空间目标的6D姿态测量算法。该算法能以j较低的计算复杂度估计空间非合作目标的6D姿态,并借助卡尔曼滤波充分利用目标运动信息对结果进行平滑,使姿态估计更具稳定性。该方法在轨服务以及空间残骸的清理等应用中具有潜在的应用价值。
