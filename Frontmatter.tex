%---------------------------------------------------------------------------%
%->> Frontmatter
%---------------------------------------------------------------------------%
%-
%-> 生成封面
%-
\maketitle% 生成中文封面
\MAKETITLE% 生成英文封面
%-
%-> 作者声明
%-
\makedeclaration% 生成声明页
%-
%-> 中文摘要
%-

\intotoc\chapter*{摘\qquad 要}% 显示在书签,目录,3.0修改2汉字字符间距
\setcounter{page}{1}% 开始页码
\pagenumbering{Roman}% 页码符号

随着人类的空间活动的增加,在轨道的失效航天器和空间残骸日益增多,占用在轨资源的同时也对在轨航天器构成一定的威胁。维修或者清理这些航天器成为了当前的一个潜在需求,为实现空间目标的维修或清理,准确获取非合作目标的6D姿态信息至关重要。然而,由于单目视觉系统缺乏直接深度信息,且面临光照剧变、弱纹理、高亮背景等挑战,加之航天器复杂多样的形状和运动,使得非合作目标的位姿估计极具难度。

本文提出了一种基于单目图像关键点检测的空间非合作目标6D姿态估计方法,主要内容包括:改进的关键点检测网络、鲁棒的位姿解算策略以及适应非线性运动的滤波方法。具体工作如下:

	
(1)为提升关键点检测的效率、精度以及在遮挡和欠曝等情况下的鲁棒性,在YOLOv8架构基础上,提出将EfficientViT和Triplet Attention相结合的关键点检测网络,充分利用了注意力机制的优越性,通时也考虑了模块的轻量化,有效提高了检测的精度。针对3D纵深明显的目标在成像后,其关键点的像素坐标损失存在平衡性问题,本研究在YOLOv8的(Object Keypoint Similarity,OKS)损失函数的基础上,进一步提出了基于针孔相机模型改进的近似3D尺度(A3DKS, Approximate Keypoint Similarity)损失。消融实验表明,上述改进均能显著提升关键点检测的精度。

(2)在由关键点估计6D位姿的过程中,关键点检测模型所预测的噪声或异常关键点会对解算精度产生不利影响。为此,本研究引入了RANSAC-TRO SQPnP求解方法,能有效处理SQPnP基于重投影误差的优化目标以及二次规划问题的构造,得到相较传统PnP算法更为精确的解。针对异常关键点,则利用RANSAC算法将所有关键点划分为若干子集并分别输入SQPnP,根据重投影误差最小化的标准,选出使SQPnP误差最小的关键点子集,并将子集之外的点识别为异常点排除,从而有效提升位姿求解的精度和稳定性。通过在SPEED+数据集上验证第(1)点与第(2)点结合的6D姿态方法能够以4.3M这样小的参数量实现1.0760 °的旋转误差以及0.0459 m的绝对位置误差这样较高的精度。

(3)对于空间自由翻滚的运动目标,因其具有非线性运动特性,常规的卡尔曼滤波方案存在局限。本文选用针对四元数的SE(3)扩展卡尔曼滤波器(Extended Kalman Filter,EKF)。同时考虑到目标角速度非恒定的特性,传统仅对四元数进行滤波(如假设角速度恒定或仅作小修正)的做法往往会产生明显累积误差,因此将关键点检测网络输出的关键点作为EKF的观测向量,为滤波器提供更丰富的信息以进行修正,从而有效提升了自由翻滚运动目标的滤波性能。

综上所述,本研究针对硬件系统要求简易的单目视觉测量场景,提出了一种用于非合作空间目标的6D姿态测量算法。该算法能以相对高效且实时的方式提供目标位姿估计,并借助卡尔曼滤波充分利用目标运动信息对结果进行平滑,使姿态估计更具稳定性。该方法在轨服务以及空间残骸的清理等应用中具有潜在的应用价值。

\keywords{6D姿态估计,空间非合作目标,关键点检测,卡尔曼滤波}% 中文关键词
%-
%-> 英文摘要
%-
\intotoc\chapter*{\textbf{Abstract}}% 显示在书签,目录,3.0修改加粗

With the increase in human space activities, the number of defunct spacecraft and space debris in orbit continues to grow, not only occupying orbital resources but also posing significant threats to operational spacecraft. Servicing or removing these objects has become an emerging requirement. To accomplish the maintenance or removal of space targets, accurately obtaining 6D pose information of non-cooperative targets is crucial. However, pose estimation of non-cooperative targets is extremely challenging due to monocular vision systems lacking direct depth information, facing issues like dramatic lighting changes, weak textures, bright backgrounds, and the complex and diverse shapes and movements of spacecraft.
This paper proposes a 6D pose estimation method for non-cooperative space targets based on monocular image keypoint detection. The main contributions include: an improved keypoint detection network, a robust pose solving strategy, and a filtering method adapted to nonlinear motion. The specific work includes:

(1) To enhance the efficiency, accuracy, and robustness of keypoint detection under occlusion and underexposure conditions, we propose a keypoint detection network combining EfficientViT and Triplet Attention based on the YOLOv8 architecture. This approach fully leverages the advantages of attention mechanisms while considering module lightweight design, effectively improving detection accuracy. For targets with significant 3D depth, pixel coordinate loss of keypoints after imaging presents balance issues. Building upon YOLOv8's Object Keypoint Similarity (OKS) loss function, we further propose an Approximate 3D Keypoint Similarity (A3DKS) loss based on the pinhole camera model. Ablation experiments demonstrate that these improvements significantly enhance keypoint detection accuracy.

(2) During 6D pose estimation from keypoints, noise or anomalous keypoints predicted by the detection model adversely affect solution accuracy. To address this, we introduce the RANSAC-TRO SQPnP solving method, which effectively handles SQPnP's optimization target based on reprojection error and quadratic programming problem construction, yielding more accurate solutions compared to traditional PnP algorithms. For anomalous keypoints, the RANSAC algorithm divides all keypoints into several subsets for separate SQPnP processing. Based on minimizing reprojection error, the keypoint subset producing the minimum SQPnP error is selected, while points outside this subset are identified as anomalies and excluded, thereby effectively improving pose solution accuracy and stability. Validation on the SPEED+ dataset shows our combined approach achieves 1.0760 ° rotational error and 0.0459 m absolute position error with only 4.3M parameters.

(3) For free-tumbling space targets with nonlinear motion characteristics, conventional Kalman filtering schemes have limitations. We employ the SE(3) Extended Kalman Filter (EKF) for quaternions. Considering non-constant angular velocity, traditional methods that only filter quaternions (such as assuming constant angular velocity or applying minor corrections) often produce significant cumulative errors. Therefore, we use keypoints output by the detection network as EKF observation vectors, providing richer information for correction and effectively improving filtering performance for free-tumbling motion targets.
In summary, this research proposes a 6D pose measurement algorithm for non-cooperative space targets in monocular vision scenarios with simple hardware requirements. The algorithm provides target pose estimation efficiently and in real-time, using Kalman filtering to leverage target motion information for result smoothing, enhancing pose estimation stability. This method has potential applications in on-orbit servicing and space debris removal operations.

\KEYWORDS{6D pose estimation, non-cooperative space targets, keypoint detection, Kalman filtering}% 英文关键词
%---------------------------------------------------------------------------%
